% Created 2017-11-19 Sun 20:51
% Intended LaTeX compiler: pdflatex
\documentclass[11pt]{article}
\usepackage[utf8]{inputenc}
\usepackage[T1]{fontenc}
\usepackage{graphicx}
\usepackage{grffile}
\usepackage{longtable}
\usepackage{wrapfig}
\usepackage{rotating}
\usepackage[normalem]{ulem}
\usepackage{amsmath}
\usepackage{textcomp}
\usepackage{amssymb}
\usepackage{capt-of}
\usepackage{hyperref}
\usepackage[a4paper, left=2cm, right=2cm, top=2cm, bottom=2cm]{geometry}
\setlength{\parfillskip}{0pt plus 1fil}
\usepackage{german}
\usepackage{fancyhdr}
\pagestyle{fancy}
\fancyhead{}
\fancyfoot{}
\fancyhead[L]{\rightmark}
\fancyhead[R]{\thepage}
\renewcommand{\headrulewidth}{0.4pt}
\renewcommand{\footrulewidth}{0pt}
\usepackage{tcolorbox}
\tcbuselibrary{theorems}
\newtcbtheorem[number within=section]{definition}{Definition}%
{colback=green!5,colframe=green!35!black,fonttitle=\bfseries}{th}
\newtcbtheorem[number within=section]{axiom}{Axiom}%
{colback=orange!5,colframe=orange!35!black,fonttitle=\bfseries}{th}
\newtcbtheorem[number within=section]{theo}{Satz}%
{colback=blue!5,colframe=blue!35!black,fonttitle=\bfseries}{th}
\newtcbtheorem[number within=section]{satz}{Satz}%
{colback=orange!5,colframe=orange!35!black,fonttitle=\bfseries}{th}
\newtcolorbox{comm}[1][]
{title=Kommentar,colback=black!5,colframe=black!35!black,fonttitle=\bfseries}
\newtcolorbox{relation}[1][]
{
colframe = red!25,
colback  = red!10,
halign = center,
#1,
}
\usepackage{etoolbox}
\usepackage{amsthm}
\usepackage{amssymb}
\usepackage{gauss}
\usepackage{stmaryrd}
\newtheorem{prof}{Beweis}[section]
\newtheorem{exa}{Beispiel}[section]
\newtheorem{expe}{experiment}[section]
\newtheorem*{notte}{Beachte}
\newtheorem*{notation}{Notation}
\newtheorem*{proposition}{Proposition}
\author{Valentin Boettcher}
\date{\today}
\title{Lineare Algebra (f"ur Physiker) I}
\hypersetup{
 pdfauthor={Valentin Boettcher},
 pdftitle={Lineare Algebra (f"ur Physiker) I},
 pdfkeywords={},
 pdfsubject={},
 pdfcreator={Emacs 25.3.1 (Org mode 9.1.2)}, 
 pdflang={English}}
\begin{document}

\maketitle
\tableofcontents

\maketitle
\newpage

\section{Vorwort}
\label{sec:org8d9fa1c}
Ihr habt hier die Mitschriften Valentin Boettchers vor euch. Er teilt eben Diese
"ausserst gern mit euch und freut sich "uber Feedback, Fehlerkorrekturen und
Verbesserungsvorschl"age. Kontaktiert ihn am besten via \href{mailto:valentin.boettcher@mailbox.tu-dresden.de}{Email} :).

Vor allem aber ist es wichtig zu verstehen, dass das Format dieses Skriptes
kein allumfassendes Kompendium ist und nur den Inhalt der Vorlesung abdeckt.
Wenn Valentin einmal ein Paar interessante Gedanken kommen, packt er sie
wohlm"oglich auch hinein, versucht aber immer deren Korrektheit zu
gew"ahrleisten. Auch Kommentare des Lesenden k"onnen Teil dieses Skriptes
werden.

Wie ihr bestimmt bis hierher bemerkt habt, ist Valentins Rechtschreibung
grausig: Also frisch ans Werk und Feedback geben.


Viel Vergn"ugen. \textbf{Mathe ist sch"on.}

\section{Mengenlehre}
\label{sec:orgd4be270}
In der modernen Mathematik fasst man Strukturen (R"aume, Fl"achen,
Zahlensysteme) als \emph{Mengen} und \emph{Abbildungen} auf.

\begin{definition}{Menge}{def-meng}
Eine Zusammenfassung von Objekten die \textbf{Elemente} der heissen. Eine Menge ist
also eindeutig dadurch bestimmt, welche Elemente sie enth"alt.
\end{definition}

\begin{notation}\
\begin{itemize}
\item \(M=\{m_1,m_2,m_3,...\}\) - Aufzeahlung
\begin{itemize}
\item \(\{...\}\) - Mengenklammern
\end{itemize}
\item \(M=\{x| P(x)\}\) - Eigenschaft
\begin{itemize}
\item Alle \(x\) mit der Eigenschaft \(P(x)\)
\end{itemize}
\end{itemize}
\end{notation}

\begin{exa}\
\begin{itemize}
\item \(n=\{\text{Nat"urliche Zahlen}\} =(add-hook 'La(add-hook 'LaTeX-mode-hook 'LaTeX-math-mode)TeX-mode-hook 'LaTeX-math-mode) \{0,1,2,...\}\)
\item \(E=\{x|\text{x hat die Eigenschaft } P(x)\}\)
\end{itemize}
\end{exa}

\subsection{Wichtige Mengen}
\label{sec:orga565e26}
\begin{itemize}
\item \(\mathbb{N}=\{\text{nat"urliche Zahlen}\} = \{1,2,...\}\)
\item \(\mathbb{Z}=\{\text{ganze Zahlen}\} = \{...,-2,-1,0,1,2,...\}\)
\item \(Q=\{\text{Rationale
   Zahlen}\}=\left\{\left.\displaystyle\frac{p}{q}\;\right\vert\begin{array}{c}p \in \mathbb{Z} \\ q \in N \setminus \{0\}\end{array}\right\}\)
\item \(\mathbb{R}=\{\text{reelle Zahlen}\}\)
\end{itemize}

\subsection{Beziehungen zwischen Mengen}
\label{sec:orgcffbbc3}
\begin{definition}{Mengenbeziehungen}{def-teilmenge}
Seien \(A,B\) zwei Mengen. 
\begin{enumerate}
\item \(A\) heisst \textbf{Teilmenge} von B, wenn f"ur jedes Element \(a\in A\) gilt: \(a\in B\).
\item Es sei die Menge \(C = \{a|a\in A \text{ und } b\in B\}\), so heisst \(C\) \textbf{Durchschnitt} von \(A\) und \(B\).
\item Es sei die Menge \(C = \{a|a\in A \text{ oder } b\in B\}\), so heisst \(C\) \textbf{Vereinigung} von \(A\) u(add-hook 'c++-mode-hook 'clang-format-bindings)nd \(B\).
\end{enumerate}
\end{definition}

\begin{notation}\
\begin{itemize}
\item \(\in\) ``Element von'': \(x\in X\) - ''x ist Element von X''
\item \(\subseteq\) Teilmenge: \(A\subseteq B\) - ''A ist eine Teilmenge von B''
\item \(\cap\) Durchschnitt: \(A\cap B = \{a|a\in A \text{ und } b\in B\}\)
\item \(\cup\) Vereinigung \(A\cup B = \{a|a\in A \text{ oder } b\in B\}\)
\item \(\varnothing\) - Leere Menge
\item \(A\setminus B\) - Mengendifferenz
\item \(A\times B\) - Direktes Produkt
\begin{itemize}
\item \((a,b)\) - geordentes Paar mit dem ersten Element \(a\) und dem zweiten
Element \(b\).
\end{itemize}
\end{itemize}
\end{notation}

\begin{exa}
\(N\subseteq \mathbb{Z}\), aber \(Q \not\subset \mathbb{Z}\): \(\frac{1}{2} \not\in \mathbb{Z}\)
\end{exa}

\begin{exa}
F"ur \(A = \{1,2,3,4,5\}\) und \(B = \{2,3,10\}\):
\begin{itemize}
\item \(A\cap B = \{2,3\}\)
\item \(A\cup B = \{1,2,3,5,10\}\)
\end{itemize}
\end{exa}

\begin{definition}{Leere Menge}{}
Die leere Menge \(\varnothing\) ist die (eindeutig bestimmte) Menge, die kein Element enth"alt.
\end{definition}

\begin{exa}
\(\{\pi\} \cap Q = \varnothing\)
\end{exa}

\begin{definition}{Differenz}{}
Die Differenz zweier Mengen \(A, B\) wird definiert als \(A\setminus B = \{a\in A | a\not\in
B\}\) (Elemente aus \(A\), die nicht in \(B\) liegen). 
\end{definition}

\begin{definition}{Direktes/Kartesisches Produkt}{}
Wenn \(A,B\) zwei Mengen sind dann ist die Menge der Paare \((a,b)\) und \(a\in A,
b\in B\) das direkte (kartesische) Produkt von \(A\) und \(B\) (\(A\times B\)).
\end{definition}

Analog gilt: \(A_1\times A_2\times ... \times A_n = \{(a_1,...,a_n)| a_1\in A_1,...,a_n\in A_n\}\)

\begin{exa}
\(\mathbb{R}^n=\mathbb{R}\times ... \times \mathbb{R} =  \{(x_1,...,x_n)| x_1\in \mathbb{R},...,x_n\in \mathbb{R}\}\)
\end{exa}

Geometrie \(m\) der Ebene mit Koordinaten \(=\) Untersuchung von Konstruktionen in
\(\mathbb{R}^2=\mathbb{R}\cdot\mathbb{R}\).

\begin{definition}{Komplemen"armenge}
Seien \(A,M\) Mengen und \(A\subseteq B\) so ist \(A^c = M\setminus A\) und heisst
\textbf{Komplement"armenge} zu \(M\).
\end{definition}

Seien \(A,B,M\) Mengen und \(A\subseteq M\) und \(B\subseteq M\), so gilt:
\begin{relation}
\begin{enumerate}
\item \((A\cup B)^c = A^c \cap B^c\)
\item \((A\cap B)^c = A^c \cup B^c\)
\item \((A^c)^c = A\)
\item \(A\cup A^c = M\)
\end{enumerate}
\end{relation}

\begin{notte}
Es gelten auch alle Identit"aten f"ur Mengen.
\end{notte}


\subsection{Abbildungen zwischen Mengen}
\label{sec:org4ef8946}
\begin{definition}{Abbildung}{}
Seien \(X,Y\) Mengen. Eine Abbildung \(f\) von \(X\) nach \(Y\) (Bez: \(f:X\rightarrow
Y\)) ist eine Vorschrift, die jedem Element \(x\in X\) ein Element von
\(y\in Y\) Zuordnet.
\end{definition}


\begin{notation}
Man schreibt: \(x\mapsto f(x)\) - ''x wird auf \(f(x)\) abgebildet'' = ''dem \(x\in
X\) wird ein \(f(x)\in Y\) zugeordnet.''
\end{notation}

\begin{exa}\
\begin{itemize}
\item \(f(t)=t^2+1\) definiert eine Abbildung \(f: \mathbb{R}\mapsto \mathbb{R}, t\mapsto f(t)=t^2+1\)
\item \(g(t)= \frac{t^2+1}{t-1}\) definiert eine Abbildung \(g: \mathbb{R}\setminus\{
   1\}\mapsto \mathbb{R}, t\mapsto  \frac{t^2+1}{t-1}\)
\item \(h: S=\{\text{Teilnehmer der Vorlesung}\}\mapsto N, s\mapsto Geburtsjahr(s)\)
\end{itemize}
\end{exa}

\subsubsection{Spezielle Abbildungen}
\label{sec:orge512a75}
\begin{relation}
\begin{enumerate}
\item F"ur jede Menge \(X\) ist die \textbf{Indentit"atsabbildung} auf \(X\) definiert durch \(Id_x:X\mapsto X, x\mapsto x\).
\item Gegeben seien Mengen \(A,B\). Die Abbildung \(\pi_A: A\times B \mapsto A, (a,b)
    \mapsto a\) heisst \textbf{Projektionsabbildung} von \(A\times B\) auf \(A\).
\item Seien \(X,Y\) Mengen, sei \(y_0 \in Y\). Dann heisst die Abbildung \(f: X\mapsto
    Y, x\mapsto y_0\) eine \textbf{konstante Abbildung} (mit dem wert \(y_0\)).
\end{enumerate}
\end{relation}

\begin{exa}\
\begin{itemize}
\item Identit"atsabbildung: \(f(x)=x\)
\item konstante Abbildung: \(f(x)=1\)
\item Projektionsabbildung: \(f(x,y)=x\)
\end{itemize}
\end{exa}

\subsubsection{Bild und Urbild}
\label{sec:org006b051}
\begin{definition}{Bild und Urbild einer Funktion}{}
Sei \(f: X\mapsto Y\) eine Abbildung.
\begin{itemize}
\item Sei \(A\subseteq X\). Dann heisst \(f(A):=\{f(a)|a\in A\}\) das Bild von A.
\item Sei \(B\subseteq Y\). Dann heisst \(f^{-1}(B):=\{a\in A|f(a)\in B\}\) das Urbild von \(B\).
\end{itemize}
\end{definition}

\begin{notte}
Das Bild und das Urbild f"ur eine \emph{Menge} einer Funktion ist wieder eine \emph{Menge}.
\end{notte}


\begin{notte}
\(f^{-1}\) ist keine Abbildung, sonder nur ein formales Symbol!
\end{notte}

\subsubsection{Einige Eigenschaften von Funktionen}
\label{sec:org1909a84}
Seien \(X,Y\) Mengen, \(f: X\mapsto Y\) eine Abbildung. \(f\) heist:
\begin{relation}
\begin{enumerate}
\item \textbf{Injektiv}, wenn f"ur \(x\in X\not = x' \in X\) gilt: \(f(x) \not = f(x')\)
\begin{itemize}
\item Keine Verklebung von Punkten!
\end{itemize}
\item \textbf{Surjektiv}, wenn f"ur \(y\in Y\) ein \(x\in X\) existiert mit \(f(x)=y\).
\begin{itemize}
\item Keine Abbildung auf eine echte Teilmenge von \(Y\)!
\end{itemize}
\item \textbf{Bijektiv}, wenn \(f\) injektiv und surjektiv ist.
\end{enumerate}
\end{relation}

\begin{exa}\
\begin{enumerate}
\item \(f: \mathbb{R} \implies \mathbb{R}, t\mapsto t^2\) 
\begin{itemize}
\item ist nicht injektiv:
\end{itemize}
\end{enumerate}
\(-1\mapsto 1\)
\begin{itemize}
\item ist nicht surjektiv: f"ur \(-1\in \mathbb{R}\) gibt es kein \(t\in\mathbb{R}\)
mit \(t^2=-1\)
\end{itemize}
\begin{enumerate}
\item \(g: \mathbb{N}\mapsto\mathbb{Z}, n\mapsto-n\)
\begin{itemize}
\item ist injektiv: \(m\ne n\implies g(m)=-m \ne -n = g(n)\)
\item ist nicht surjektiv: f"ur \(1\in \mathbb{Z}\) gibt es kein \(n\in \mathbb{N}\)
mit \(-n=1\)
\end{itemize}
\item \(h: \mathbb{R}\mapsto\mathbb{R},t\mapsto t^3\) ist Bijektiv ("Ubung)
\end{enumerate}
\end{exa}

\subsubsection{Inverse Abbildung zu einer bijektiven Abbildung}
\label{sec:orgaae7124}
\begin{definition}{Inverse Abbildung}{}
Sei \(f:X\mapsto Y\) bijektiv. Sei \(y\in Y\). Definiere eine Abbildung \(f^{-1}:
Y\mapsto X\) so: \(f^{-1}(y)=x\) mit der Eigenschaft \(f(x)=y\).
\end{definition}

Dies ist wohldefiniert (diese Vorschrift definiert tats"achlich eine Abbildung)
weil:

\begin{relation}
\begin{itemize}
\item Das x mit der gew"unschten Eigenschaft existiert f"ur jedes \(y\in Y\), weil
\(f\) surjectiv ist.
\item F"ur jedes \(y\in Y\) existiert h"ochstens ein \(x\in X\) mit der gew"unschten
Eigenschaft, weil \(f\) injektiv ist.
\end{itemize}
\end{relation}

\begin{notte}
Wenn die Abbildung \(f\) bijektiv ist, hat \(f^{-1}(A)\) f"ur ein \(A\subseteq Y\) a
priori zwei Bedeutungen:
\begin{itemize}
\item Urbild von \(A\) unter f
\item Bild von \(A\) von \(f^{-1}\)
\end{itemize}

Wenn \(f\) bijektiv ist, stimmen aber diese Mengen "uberein. (Bew. "Ubung)

\textbf{Aber}: Wenn \(f\) nicht bijektiv ist, hat \(f^{-1}\) nur einen Sinn: Urbild!
\end{notte}

\subsubsection{Verkn"upfung von Abbildungen}
\label{sec:org540b965}
\begin{definition}{Verkn"upfung}{}
\(f: X\mapsto Y, g: Y\mapsto Z\) ist die verkn"upfung \(g\circ: X\mapsto Z\) definiert
als \(g\circ f(x)=g(f(x))\). Diagramme Siehe V2\(_{\text{1}}\).
\end{definition}


Die Verkn"upfung hat folgende Eigenschaften:
\begin{relation}
\begin{enumerate}
\item Sie ist Assoziativ: \(h\circ (g\circ f) = (h \circ g) \circ f\) f"ur alle Abb. \(f: X\mapsto Y, g:Y\mapsto Z\), \(h:Z\mapsto V\)
\item F"ur jede abbildung \(f: X\mapsto Y\) gilt: \(f\circ id_X=id_Y\circ f = f\).
\item Wenn \(f:X\mapsto Y\) bijektiv ist, dann gilt: \(f\circ f^{-1}=id_Y\):
\begin{itemize}
\item \(f^{-1}\circ f=id_X\) weil: \(f(f^{-1}(y))=y\):
\item \(f^{-1}(f(x))=x'\) mit \(f(x')=f(x)\implies x=x'\) wenn \emph{Bijektiv}
\end{itemize}
\end{enumerate}
\end{relation}

\subsubsection{Kommutative Diagramme}
\label{sec:org429b8d6}
Siehe V2\(_{\text{2}}\): 
\begin{enumerate}
\item Dieses Diagramm heist kommutativ, wenn \(h=g\circ f\).
\item kommutativ wenn \(g\circ f=h\circ k\)
\end{enumerate}

\subsubsection{Eingeschr"ankte Abbildungen}
\label{sec:org60b2559}
\begin{definition}{Einschr"ankung}{}
Sei \(f: X\mapsto Y\) eine Abbildung.\\
Die Einschr"ankung von \(f\) auf eine Teilmenge \(A\subseteq X\) ist die  Abbildung:
\(f|_A:\begin{matrix}A\mapsto Y\\ a\mapsto f(a)\end{matrix}\)
\end{definition}

\begin{exa}
\(f: \mathbb{R}\mapsto \mathbb{R}, t\mapsto t^{2}\) ist nicht injektiv, \(f|_{[0,
\infty)}\) ist injektiv.
\end{exa}

\subsubsection{Quantoren}
\label{sec:orgd2b2557}
\begin{definition}{Quantoren}{}
\begin{itemize}
\item f"ur alle \(x\) in \(X\) - \(\forall x \in X\)
\item es existiert \(x \in X\) - \(\exists x \in X\)
\end{itemize}
\end{definition}

\begin{exa}
\(f:X\mapsto Y\) ist surjektiv, wenn \(\forall y \in Y \exists x\in X\) mit \(f(x)=y\).
\end{exa}

F"ur die Negation der Quantoren gilt:
\begin{relation}
\begin{itemize}
\item \(\neg(\forall x\in X : A(x)) = \exists x\in X : \neq A(x)\)
\item \(\neg(\exists x\in X : A(x)) = \forall x\in X : \neq A(x)\)
\end{itemize}
\end{relation}

\begin{exa}[] \label{} \
\(f: X\mapsto Y\) ist surjektiv \(\iff \forall y\in Y \exists x\in X : f(x)=y\).\\
Also: \(f: X\mapsto Y\) ist \textbf{nicht} surjektiv \(\iff \exists y\in Y \forall x\in X : f(x)\not=y\).
\end{exa}

\subsection{Schlagworte}
\label{sec:org0cab6a2}
\begin{itemize}
\item Venn Diagram - Kreise und Schnittmengen
\item Zeigen von "Aquivalenz zweier Zusammengeseten Mengen:
\begin{itemize}
\item Wahrheitstafel
\item Zur"uckf"uhren auf Aussagenlogik
\end{itemize}
\item Zeigen das \(p,q,r\) "aquivalent sind:
\begin{itemize}
\item \(p\implies q \implies r \implies q\)
\end{itemize}
\item \emph{Injektivit"at} zeigen:
\begin{itemize}
\item nicht I. wenn Gegenbeispiel existiert
\item Zeigen das Funktion streng monoton steigt.
\end{itemize}
\item \emph{Surjektivit"at} zeigen:
\begin{itemize}
\item nicht S. wenn Gegenbeispiel existiert
\item Zeigen das Funktion streng monoton steigt und gegen \(+-\infty\) strebt.
\end{itemize}
\item \(A\setminus (A\setminus B) = A \cap B\)
\item Beweise mit Abbildungen \(M\) sei Menge, \(f\) sei Abbildung:
\begin{itemize}
\item \(y \in f(M) \implies \exists x \in M : f(x)=y\)
\end{itemize}
\end{itemize}

\section{Logik und Beweisf"uhrung}
\label{sec:orgc083250}
Mathematik operiert mit \textbf{Aussagen}.

\begin{definition}{Aussage}{}
Eine Aussage ist eine Behauptung, die Wahr oder Falsch sein kann.
\end{definition}

\begin{notation}\
\begin{description}
\item[{1}] wahr
\item[{0}] falsche
\end{description}
\end{notation}

\(A,B\) seien Aussagen, dann kann man folgende Aussagen betrachten:
\begin{relation}
\begin{itemize}
\item ''nicht \(A\)'': \(\neg A\)
\end{itemize}
\begin{center}
\begin{tabular}{lrr}
\(A\) & 0 & 1\\
\hline
\(\neg A\) & 1 & 0\\
\end{tabular}
\end{center}

\begin{itemize}
\item Vernk"upfungen
\end{itemize}
\begin{center}
\begin{tabular}{rrrrrr}
\(A\) & \(B\) & \(\neg A\) & \(A\wedge B\) & \(A \vee B\) & \(A\implies B\)\\
\hline
0 & 0 & 1 & 0 & 0 & 1\\
0 & 1 & 1 & 0 & 1 & 1\\
1 & 0 & 0 & 0 & 1 & 0\\
1 & 1 & 0 & 1 & 1 & 1\\
\end{tabular}
\end{center}

\begin{itemize}
\item ''A "aquivalent zu B'':  \(A\iff B\)
\end{itemize}
\begin{center}
\begin{tabular}{rrr}
\(A\) & \(B\) & \(\iff A\)\\
\hline
0 & 0 & 1\\
0 & 1 & 0\\
1 & 0 & 0\\
1 & 1 & 1\\
\end{tabular}
\end{center}
\end{relation}

\begin{exa}
F"ur ein Element \(x\in X\) k"onnen wir Aussagen betrachten:
\begin{enumerate}
\item \(A(x)=x\in A\)
\item \(B(x)=x\in B\)
\end{enumerate}
\(A(x)\wedge B(x)=x\in (A\cap B)\)
\end{exa}

\subsection{Identit"aten der Aussagenlogik}
\label{sec:orgd743b6e}
\begin{relation}
\begin{enumerate}
\item Direkter Beweis 
\begin{itemize}
\item \((A\implies B) = (\neg A)\vee B\)
\item Vorraussetzung \(\rightarrow\) logische Aussage \(\rightarrow\) Behauptung
\end{itemize}
\item Beweis in Schritten
\begin{itemize}
\item \(((A\implies B)\wedge (B\implies C))\implies (A\implies C)\) \\
\(\rightarrow\) Konstant \(=1\) (\emph{Tautologie})
\end{itemize}
\item Beweis durch Kontraposition
\begin{itemize}
\item \((A\implies B) \iff (\neg B \implies \neg A)\) - \emph{Tautologie}
\end{itemize}
\end{enumerate}
\end{relation}

\subsection{Widerspruchsbeweis}
\label{sec:org54c9d02}
Wenn wir die Konsequenz aus der Negation der zu beweisenden Aussage und die
Pr"amisse zu einem widerspruch f"uhren so ist die Aussage bewiesen, denn:
\begin{relation}
\[(A\wedge \neg A)=0\]
\end{relation}


Wir wollen \(A\implies B\) zeigen.
Nehmen an \(\neg B\) und leiten her:\\
\begin{relation}
\((\neg B \wedge A)\implies 0\), also \(\neg B\wedge A = 0\), und daher \(A\implies
B\).
\end{relation}

\begin{theo}{Satz von Euklid}{}
Es gibt unendlich viele Primzahlen.
\end{theo}

\begin{proof}\
\begin{enumerate}
\item Nehmen wie an, es gibt nur endlich viele Primzahlen. \(p_1, ..., p_n\).
\item Betrachte \(n=p_1\cdot p_2\cdot ... \cdot p_n + 1\). \(n\) geteilt durch jede
von den Primzahlen \(p_1, ..., p_n\) gibt Rest \(1\).
\item Also ist \(n\) eine Primzahl, aber \(n\not=p_1 ... p_n\) weil gr"osser.
\item Folglich enth"alt die Menge \({p_1,...,p_n}\) nicht alle Primzahlen.
\end{enumerate}
\indent\indent \(\rightarrow\) Das ist ein \textbf{Widerspruch}. (\((A\wedge \neg A) = 0\))
\end{proof}


\begin{exa}
Wir werden die Aussage: wenn \(q\) eine gerade Primzahl ist \(\implies q=2\)
beweisen.

\begin{proof}[Direkter Beweis] \label{} \
\begin{enumerate}
\item \(q\) ist gerade \(\implies q\) ist durch \(2\) Teilbar f"ur \(k\in\mathbb{N}\)
\item \(q\) ist aber eine Primzahl \(\implies\) einer der Faktoren in \(2\cdot k\) ist
gerade \(1\), \(2\not= 1\)
\item \(\implies k=1, q=2\)
\end{enumerate}
\end{proof}

\begin{proof}[Kontraposition] \label{} \
Wir m"ussen zeigen: \(q\not= 2\implies\) (\(q\) ungerade) \(\vee\) (\(q\) keine
Primzahl). Es reicht zu zeigen: (\(q\not=2)\wedge(q\) ist eine Primzahl)
\(\implies q\) ist ungerade!
\begin{enumerate}
\item Wenn \(q\) gerade ist, \(q\cdot 2k\), also ist \(k>1\)
\item also \(q\not= 2\)
\end{enumerate}
\end{proof}

\begin{proof}[Widerspruchsbeweis] \label{} \
Annahme: \(q\) ist gerade, \(q\) ist eine Primzahl, \(q\not= 2\). Wir wollen einen
Widerspruch herleiten.

\begin{enumerate}
\item da \(q\) gerade ist, gilt \(q=2\cdot k\) f"ur ein \(k\in \mathbb{N}\)
\item da \(q\not= 2\), gilt \(k>1\)
\item aber \(q\) ist prim, also kann \(q\) kein Produkt von zwei Zahlen sein! \(\lightning\)
\end{enumerate}
\end{proof}
\end{exa}

\section{Komplexe Zahlen}
\label{sec:org73b0a26}
Idee: Man m"ochte Quadratische Gleichungen ohne reelle Nullstellen trotzdem
l"osen, also erweitert man die reellen Zahlen.

\begin{relation}
Die pototypische Quadratische Gleichungen ohne reelle L"osungen ist: \(x^2+1 =
-1\).\\
Man f"ugt K"unstlich die Zahl \(i\) hinzu mit \(i^2=-1\), m"oglichst unter
Beibehaltung der "ublichen Rechenregeln: man braucht also die Zahlen \(b\cdot i :
b\in \mathbb{R}\) und \(a+b\cdot i :  a,b\in \mathbb{R}\).
\end{relation}

Was passiert, wenn man solche Zahlen miteinander multipliziert ''als ob'' sie
normale Zahlen w"aren: 
\begin{relation}
\((a+bi)\cdot(c+di)=ac+bc\cdot i+ad\cdot i+(-bd)=(ac-bd)+(bc+ad)\cdot i\) f"ur \(a,b,c,d\in \mathbb{R}\)
\end{relation}

Addieren kann man solche Ausdr"ucke auch:
\begin{relation}
\((a+bi)+(c+di)=(a+c)+(b+d)\cdot i\)
\end{relation}

\begin{definition}{Komplexe Zahlen}{}
Die komplexen Zahlen \(\mathbb{C}\) sind die Menge der Paare \((a,b)\in
\mathbb{R}^2\) versehen mit der Addition \((a,b)+(c,d)=(a+c,b+d)\) und der
Multiplikation \((a,b)\cdot (c,d)=(ac-bd, bc+ad)\).
\end{definition}

\begin{notation}[] \label{} \
\begin{itemize}
\item Statt \((a,b)\) schreibt man auch \((a+bi)\in \mathbb{C}\).
\item \(i:=(0,1)=0+1\cdot i\):
\begin{itemize}
\item nach Multiplikation erf"ullt \(i^2=-1\)
\end{itemize}
\end{itemize}
\end{notation}

Man "uberpr"uft, dass die "ublichen Rechenregeln aus \(\mathbb{R}\) weiterhin
gelten (\emph{K"orperaxiome}): F"ur \(z_1, z_2, z_3 \in \mathbb{C}\) gilt, z.B.:
\begin{relation}
\begin{itemize}
\item \(z_1\cdot (z_2+z_3)=z_1z_2+z_1z_3\)
\item \(z_1\cdot (z_2\cdot z_3)=(z_1\cdot z_2)\cdot z_3\)
\item \(z_1 + (z_2 + z_3)=(z_1 + z_2) + z_3\)
\end{itemize}
\end{relation}

\begin{notte}[] \label{}
\((\mathbb{R},+,\cdot)\subsetneq (\mathbb{C},+,\cdot)\) auf nat"urliche Weise als
der der Form \(a+0\cdot i = (a,0)\), \(a\in \mathbb{R}\).
\end{notte}

\begin{definition}{Real- und Imagin"aranteil}{}
F"ur \(z=a+b\cdot i\in \mathbb{C}\) heisst:
\begin{itemize}
\item \(a:=:Re(z)\) Realanteil von \(z\)
\item \(b:=:Im(z)\) Imagin"aranteil von \(z\)
\end{itemize}

Also ist \(z=Re(z)+ Im(z)\cdot i\).
\end{definition}

\begin{definition}{Rein Imagin"are Zahlen}{}
Die Zahlen der Form \(b\cdot i : b\in \mathbb{R}\) heissen \textbf{rein Imagin"ar}.
\end{definition}

F"ur reele Zahlen wissen wir: \(\forall a\in \mathbb{R}\) mit \$a\textlnot{}= 0 \(\exists\)
\(a^{-1} \in \mathbb{R}\) mit \(a*a^{-1}=1\). Gilt das auch in \(\mathbb{C}\) ?

\begin{definition}{Komplexe Konjugation}{}
F"ur \(z\in \mathbb{C}\) heisst die Zahl \(\overline{z}:=a-bi\) die komplex
konjugierte Zahl zu \(a+bi\).
\end{definition}

\begin{exa}[] \label{}
\(\overline{1+i}=1-i\)
\end{exa}

\begin{relation}
\(z*\overline{z}=(a+bi)(a-bi)=a^2+b^2\geq -\) mit Gleichheit genau dann, wenn \(z=0\).
\end{relation}

\begin{definition}{Betrag der Komplexen Zahl}{}
\(|z|:=\sqrt{x\cdot \overline{z}}=\sqrt{a^2+b^2}\) mit \(z=a+bi\).
\end{definition}

\subsection{Inverses zu einer komplexen Zahl}
\label{sec:org0018acd}
Das Inverse zu \(z\not= 0\):
\begin{relation}
\(z\cdot \frac{\overline{z}}{|z|^2}=1\) \\
Also: \(\forall z\not= 0 \exists z^{-1} = \frac{\overline{z}}{|z|^2}\) mit \(z \cdot z^{-1}=1\)
\end{relation}

\begin{exa}[] \label{} \
\((1+i)^{-1}=\frac{1-i}{2}\)
\end{exa}

Mnemonische Rechenregel, Multipliziere mit dem Inversen: 
\begin{relation}
\(\frac{1}{1+i}=\frac{1-i}{(1+i)(1-i)}=\frac{1-i}{2}\)
\end{relation}

\subsection{Geometrische Interpretation von \(\mathbb{C}\)}
\label{sec:org992fe0c}
Siehe Zeichung \(C_1\).

\begin{relation}
\begin{itemize}
\item Addition: als Addition von Vektoren
\item Betrag: L"ange des Vektors
\item \(\varphi\) - Winkel zwischen der reellen Achse und dem Vektor der \(z\) entspricht,
gez"ahlt gegen den Urzeigersinn.
\end{itemize}
\end{relation}

Es folgt:
\begin{relation}
\(a=|z|\cdot \cos(\varphi)\) und \(b=|z|\cdot \sin(\varphi)\)
\end{relation}

\begin{notte}[] \label{}
\(\varphi\) ist nicht eindeutig bestimmt, sondern bis auf Addition von eines
vielfachen von \(2\pi\).
\end{notte}

\begin{exa}[] \label{}
\(\varphi=\frac{\pi}{4}\) und \(\varphi=-\frac{7\pi}{4}\) sind im geometrischen Bild von
\(\mathbb{C}\) "aquivalent.
\end{exa}

\begin{definition}{}{}
Der wert von \(\varphi\), welcher in \([0, 2\pi)\) liegt, heisst Hauptargument von \(z\),
\(arg(z)=\varphi\).\\
Das Argument von \(z\) ist die Menge von allen \(\varphi \in R\),4
\(z=|z|(\cos(\varphi)+i\cdot \sin(\varphi))\), \(Arg\, z = {\varphi \in R : |z|(\cos(\varphi)+i\cdot \sin(\varphi))}\).
\end{definition}

\begin{notte}[] \label{}
\(Arg\, z= {arg(z)+2\pi\cdot k : k\in \mathbb{Z}}\)
\end{notte}

\begin{exa}[] \label{} \
Seien \(z_1=|z_1|\cdot \cos(\varphi_1)+i\cdot \sin(\varphi_1)\), \(z_2=|z_2|\cdot
\cos(\varphi_2)+i\cdot \sin(\varphi_2)\) zwei komplexe Zahlen.\\

So gilt: \(z_1\cdot z_2 = |z_1|\cdot |z_2|(\cos(\varphi_1+\varphi_2)+i\cdot \sin(\varphi_1 +
\varphi_2))\)
\end{exa}

\begin{relation}
Bei der Multiplikation von komplexen Zahlen multiplizieren sich die Betr"age,
und die Argumente addieren sich.
\end{relation}

F"ur geometrische Interpretation: Siehe \(C_2\).

Besonders n"utzlich ist dies f"ur die Multiplikation einer komplexen Zahl vom
Betrag \(1\):
\begin{align*}
|z|=1\iff z=\cos(\varphi)+i\cdot \sin(\varphi) \text{f"ur ein} \varphi \in \mathbb{R}
\end{align*}

\begin{relation}
Es liegen \(\{z\in \mathbb{C} : |z|=1\}\) auf dem Einheitskreis.
Die Multiplikation mit von komplexen Zahlen Zahlen mit dem Betrag 1 entspricht
also der Rotation gegen den Urzeigersinn um \(\varphi\).
\end{relation}

\subsection{Exponentialform der komplexen Zahlen}
\label{sec:orgb4d9f14}
\begin{notation}[] \label{} \
\begin{itemize}
\item Exponentialform: \(\cos(\varphi)+i\cdot \sin(\varphi):=e^{i\cdot \varphi}\)
\item es gilt \(e^{i(\varphi_k)}, k\in\mathbb{R}\) sind die Zahlen auf dem Einheitskreis
\end{itemize}
\end{notation}

\begin{definition}{Exponentialform der komplexen Zahlen}{}
Die Exponentialform f"ur jede komplexe Zahl \(z\in\mathbb{C}\) lautet \(z=|z|e^{i\cdot arg\,z}\).
\end{definition}

Mit dieser Notation folgt:
\begin{relation}
\((e^{i\varphi})^n=(\cos(\varphi)+i\cdot\sin(\varphi))^2=e^{n\cdot i\cdot
 \varphi}=\cos(n\varphi)+i\cdot\sin(n\varphi)\) f"ur alle \(n\in\mathbb{N}\)
\end{relation}

\begin{exa}[] \label{}\
\begin{equation*}
%\begin{split}
(\cos(\varphi)+i\cdot \sin(\varphi))^2 =\cos^2(\varphi)-\sin^2(\varphi)+2\cdot\sin(\varphi)\cdot\cos(\varphi) 
 = \cos(2\varphi) + 2\sin(2\varphi) 
 \implies 
\begin{cases}
\cos(2\varphi)=\cos^2(\varphi)-\sin^2(\varphi) \\
\sin(2\varphi)=2\cdot\sin(\varphi)\cdot\cos(\varphi)
\end{cases}
%\end{split}
\end{equation*}
\end{exa}

\subsection{Einscheitswurzeln}
\label{sec:org0755105}
Sei die gleichung \(x^n=a\) "uber \(\mathbb{R}\) gegeben. Je nach Vorzeichen von
\(a\) und Parit"at von \(n\), gibt es Varianten f"ur die Anzahl der L"osungen.
\begin{relation}
In \(\mathbb{C}\) hat aber die Gleichung \(z^n=a\) f"ur ein \(a\in
\mathbb{C}\setminus \{0\}\) immer genau \(n\) L"osungen.
\end{relation}

Sei \(w\in \mathbb{C}\) mit \(w^n=a\). Dann gilt \((\frac{z}{w})^n=1\) f"ur jedes
\(z\in \mathbb{C}\) mit \(z^n=a\). \textbf{Also} l"osen wir erst mal die Gleichung \(z^n=1\),
und dann reduzieren wir den allgemeinen Fall darauf.

\begin{definition}{Einheitswurzel}{}
Eine Zahl \(z\in \mathbb{C}\) heisst \(n\text{-te}\) Einheitswurzel, wenn \(z^n=1\).
\end{definition}

\begin{proposition}[] \label{}
F"ur jedes \(n\geq, n\in\mathbb{N}\) existieren genau \(n\)
Einheitswurzeln in \(\mathbb{C}\). Sie sind durch die Formel
\(z_k=e^{\frac{2\pi\cdot k\cdot i}{n}},\quad k=0,1,...,n-1\) gegeben.
\end{proposition}

\begin{proof}[] \label{} \
\(z_k\) sind \(n\text{-te}\) Einheitswurzeln denn:
\begin{align*}
z_k^n & = (e^{\frac{2\cdot\pi\cdot k}{n}})^n \\ 
& = e^{2\pi\cdot k} \\
& = 1
\end{align*}


Wir m"ussen noch zeigen, dass jede \(n\text{-te}\) Einheitswurzel von dieser Form
ist. \\

Sei \(z\in\mathbb{C}\) mit \(z^n=1\). Es gilt:

\begin{align*}
|z|^n & =|z^n|=1 \\
& \implies |z|=1  \\
& \implies z=e^{i\cdot\varphi} \tag*{f"ur ein $\varphi\in[0, 2\pi)$}  \\ 
& \implies 1 = z^n \\
& = (e^{i\varphi})^n=e^{i\varphi\cdot n} \\
& =\cos(n\varphi)+i\cdot \sin(n\varphi)
\end{align*}

Also folgt:
\begin{gather*}
\cos(n\varphi)=1,\;\sin(n\varphi)=0 \\
\implies  n\cdot\varphi = 2\pi\cdot k \tag*{f"ur ein $k\in \mathbb{Z}$} \\ 
 \implies \varphi = \frac{2\pi\cdot k}{n} \tag*{f"ur ein $k\in \mathbb{Z}$}
\end{gather*}


Da \(\varphi\) in \([0,2\pi)\implies 0\leq k < n\).
\end{proof}

Wenn wir jetzt also eine Gleichung \(z^n=a\) l"osen wollen, reicht es, eine
L"osung \(w\) zu finden, die anderen L"osungen bekommt man als \(w\cdot z_k,\;
k=0,...,n-1\) mit \(z_k\), der \(n\text{-ten}\) Einheitswurzeln: \(z^n=a\iff
(\frac{z}{w})^n=1\).\\

Eine L"osung \(w\) kann man folgendermassen finden:
\begin{relation}


\begin{align*} 
\text{Schreiben wir a}\; & =|a|\cdot e^{i\cdot \psi}\; \text{f"ur ein $\psi\in \mathbb{R}$} \\
\text{Dann gilt: }
w & =\sqrt[n]{|a|}\cdot e^{\frac{i\cdot\psi}{n}} \text{ l"ost $w^n=a$} \\
& \\
\left(\sqrt[n]{|a|}\cdot e^{\frac{i\cdot\psi}{n}}\right)^n & = \sqrt[n]{|a|}\cdot e^{\frac{i\cdot\psi}{n}\cdot n} \\
& = |a|\cdot e^{i\cdot \psi} \\ 
& = a
\end{align*}
\end{relation}

Gemetrische Interpretation: regul"ares \(n\text{-Eck}\).

\newpage

\section{Lineare Gleichungsysteme}
\label{sec:org4c2c71d}
Wir werden die Bezeichung \(K\) f"ur \(\mathbb{R}\) oder \(\mathbb{C}\) verwenden.

\begin{definition}{Lineare Gleichung}{}
Eine Lineare Gleichung "uber \(K\) ist eine Gleichung der Form
\(a_1x_1+a_2x_2+...+a_nx_n=b\).\\
Hierbei sind \(x_1,...,x_n\) die Variablen und \(a_1,...,a_n,b \in K\), die Koeffizienten.
\end{definition}

\begin{definition}{Lineares Gleichunssystem}{}
Ein Lineares Gleichungsystem ist eine endliche Menge von Gleichungen:
\[{\displaystyle {\begin{matrix}a_{11}x_{1}+a_{12}x_{2}\,+&\cdots
&+\,a_{1n}x_{n}&=&b_{1}\\a_{21}x_{1}+a_{22}x_{2}\,+&\cdots
&+\,a_{2n}x_{n}&=&b_{2}\\&&&\vdots &\\a_{m1}x_{1}+a_{m2}x_{2}\,+&\cdots
&+\,a_{mn}x_{n}&=&b_{m}\\\end{matrix}}}\]
\end{definition}

Ein L"osung von diesem Gleichungssystem ist ein \[n\text{-Tupel }
\left( \begin{matrix} x_{1}\\ \vdots\\ x_{n}\end{matrix} \right) \in K^{n} \]
dass jede Gleichung erf"ullt. Ein lineares Gleichungssystem (LGS) zu l"osen,
heisst, alle L"osungen zu finden.

\begin{relation}
\textbf{Idee}: Man formt das LGS durch Operationen um, die die Menge der L"osungen nicht
ver"andern. Solche Operationen heissen "Aquivalenzumformungen. Diese sind unter
anderem: 
\begin{enumerate}
\item Multiplikation einer Gleichung mit einer zahl \(\alpha\in K\setminus \{0\}\)
\item Addierung von einer Gleichung zu der anderen (z.B. Ersetzen der zweiten
Gleichung durch die Summe der ersten und zweiten.)
\item Vertauschen von zwei Gleichungen; dies kann man auf Operationen von Typ eins
und Zwei zur"ukf"uhren
\end{enumerate}
\end{relation}

Wir werden ein LGS umformen, um es auf eine Form zu bringen, wo die L"osung
offensichtlich ist.

Wir beobachten:
\begin{relation}
Es ist "uberflu"ssig, die Variablen mitzuschleppen. Man k"onnte statdessen die
''Tabellen'' von Koeffizienten umformen.
\end{relation}

\begin{definition}{}{}
Eine \(M\times N\) Matrix \(A\) ist eine Tabelle der Gr"osse \(m\times n\), gef"ullt
mit Elementen aus \(K\).  
\[A=(a_{ij})_{\substack{i=1,\cdots,m \\ j=1,\cdots,n}}\]
\end{definition}

\begin{exa}[] \label{}
\[
  A=\left( \begin{matrix} 1& 1\\ 2& -3\end{matrix} \right)
\]

Wobei \(a_{11} = 1\), \(a_{21} = 2\), \(a_{12}=1\) und \(a_{22}=-3\).
\end{exa}

\begin{relation}
Gegeben ein LGS (\(*\)), k"onnen wir eine Matrix \[ A=\left( \begin{matrix}
a_{11}& \cdots & a_{1n}\\ \vdots & & \vdots \\ a_{n1}& \ldots &
a_{nn}\end{matrix} \right) \] aufstellen. Sie heisst Koeffizientenmatrix des
LGS. Auch stellen wir \[b=\left( \begin{matrix} b_{1}\\ \vdots
\\ b_{n}\end{matrix} \right)\]
eine \(m\times 1\) Matrix (Spalte) auf. (Sie
heisst rechter Teil des LGS). Die Matrix \(A'=(A\mid b)\) heisst erweiterte
Koeffizientenmatrix des LGS (\(*\)).
\end{relation}


\begin{definition}{Elementare Zeilenumforumungen}{}
Die "Aquivalenzumformungen des LGS, die wir vorhin betrachtet haben, entsprechen
dann folgenden Umformungen von der erweiterten Koeffizientenmatrix: 
\begin{itemize}
 \item[1'.] Multiplikation einer Zeile mit $\alpha \in K^\times$
 \item[2'.] Addieren von einer Zeile zu der anderen.
\end{itemize}
Wir werden dann versuchen, die (erweiterten koeffzienten-) Matrizen durch diese
Umformungen auf eine Form zu bringen, in der man die L"osung leicht ablesen
kann.

\(1'\) und \(2'\) heissen elementare Zeilenumforumungen.
\end{definition}

Weitere Zeilenumformungen, die man aus diesen erhalten kann:
\begin{relation}
\begin{itemize}
\item Vertauschen Zweier Zeilen
\item Addieren einer Zeile, Multipliziert mit \(\alpha \not= 0\)
\end{itemize}
\end{relation}

Ziel ist eine gegebe erweiterte Koeffizientenmatrix \((A\mid b)\), durch
Zeilenumformungen zu einer Matrix umzuformen, aus der man die L"osung leicht
ablesen kann.

\begin{definition}{Pivotelement}{}
Gegeben einer Zeile \(Z=(a_1,...,a_n)\in K^n\), nennen wir das erste Element
\(a\not= 0\) das Pivotelement.
Wenn \(Z=(0,...,0)\) ist dann gibt es kein Pivotelement.
\end{definition}

\begin{definition}{Zeilenstufenform}{}
Eine Matrix \(A\) hat Zeilenstufenform, wenn folgendes gilt:
\begin{enumerate}
\item Die Nummern von Pivotlementen der Zeilen von \(A\) bilden eine aufsteigende
Folge.
\item Die Nullzeilen, falls existent, stehen am Ende.
\end{enumerate}
\end{definition}

\begin{exa}[] \label{} \
\[
\begin{pmatrix}
 0 & a_{12} & a_{13} \\
 0 & 0 & a_{23} \\
 0 & 0 & 0 \\
\end{pmatrix}
\]
\end{exa}

\begin{theo}{Gauss}{}
Jede Matrix kann durch elementare Zeilenumformungen auf die Stufenform gebracht
werden.
\end{theo}

\begin{proof}[] \label{}
Sei \(A=\begin{matrix}a_{11}&...&a_{nn}\end{matrix}\). \\
Wenn \(A=0\) - Bewiesen. \\
Wenn \(A\not=0\), dann gibt es eine Spalte \(\not= 0\). Sei
\(j_1\) die Nummer dieser Spalte. Durch vertausche von Zeilen erreichen wir
zun"achst \(a_{1j_1}\not= 0\). Multiplaktion der ersten Zeule mit
\(\frac{1}{j_1}\). Jetzt Subtrahiere von jeder Zeile ab der Zweiten die erste
Zeile multipliziert mit \(a_{kj_1}\) (\(k=\) Nummer der Zeile). \\

Wir erhalten dann Restmatrix \(A_1<A\) und wir wenden das selbe Verfahren auf
\(A_1\) an. Da \(A_1\) weniger Zeilen hat, stoppt der ganze Prozess.

\begin{notte}
Nach diesem Verfahren gilt sogar: Pivotelemente sind alle \$=1\$1
\end{notte}
\end{proof}


\begin{exa}
\begin{align*}
  & \begin{gmatrix}[p]
      1 & 2 \\
      3 & 4
      \rowops
      \add[-3]{0}{1}
    \end{gmatrix} \\
  \Rightarrow & \begin{gmatrix}[p]
      1 & 2 \\
      0 & -6
      \rowops
      \mult{1}{\scriptstyle\cdot-\frac{1}{6}}
    \end{gmatrix} \\
  \Rightarrow & \begin{gmatrix}[p]
      1 & 2 \\
      0 & 1
      \rowops
      \add[-2]{1}{0}
    \end{gmatrix} \\
  \Rightarrow & \begin{gmatrix}[p]
      1 & 0 \\
      0 & 1
    \end{gmatrix}
\end{align*}
\end{exa}

\begin{definition}{Reduzierte Zeilenstufenform}{}
Nachdem wir die Zeilenstufenform mit Pivotelementen \(=1\) erreicht haben, k"onnen
wir durch weitere Zeilenumformungen die eintr"age zu Null f"uhren, die oberhalb
von Pivotelementen stehen; Die Finalform heisst dann \textbf{reduzierte
Zeilenstufenform}.
\end{definition}

Das entsprechende Verfahren zum L"osen von LGS sieht so aus:
\begin{relation}
\begin{enumerate}
\item Bringe die erweiterte Koeffizientenmatrix auf die reduzierte
Zeilenstufenform: \\
Die Spalten mit den Pivotelementen in dieser reduzierten Zeilenstufenform
nennen wir Basispalten.
\item Zwei F"alle:
\begin{enumerate}
\item Letzte Spalte des ist eine Basispalte - in diesem Fall hat das LGS keine
L"osungen, da eine Gleichung \(0=1\) entsteht.
\item Die letzte Spalte ist keine Basisspalte: \\
Das LGS in der reduzierten Zeilenstufenform dr"uckt die Variablen, die zu
Basisspalten geh"oren , durch die restlichen (freien) Variablen und den
rechten Teil des LGS aus. Alle L"osungen werden dadurch erhalten, dass
man f"ur die freien Variablen beliebige Werte in \(K\) ausw"ahlt. Die
Basisvariablen werden dann durch Freie Variablen ausgedr"uckt.
\end{enumerate}
\end{enumerate}
\end{relation}

In unserem Beispiel l"asst sich die L"osung so aufschreiben: 


Errinnerung: ein LGS hatte die erweiterte Koeffizientenmatrix \((A|b)\). Das LGS4
l"asst sich dann auch so aufschreiben:$\backslash$\ \(:=A\cdot x\), wobei \(x=\)


\subsection{Matrizenrechnung}
\label{sec:org0f3e63e}
\begin{definition}{Matrix-Spaltenvektor Produkt}{}
Das Produkt von einer \(m\times n\) Matrix \(A\) und einer Spalte (in dieser
Reihenfolge) wird definiert durch \(A\cdot x =\). In dieser Spalte wird das LGS
\(A\cdot b\).
\end{definition}

Die Menge von Matrizen der Gr"osse \(m\times n\) mit Eintr"agen in \(K\) wird durch
\(M(m\times n, k)\) oder \(K^{m\times n}\) bezeichnet. Matrizen der Gr"osse \(1\times
n\) heissen Spalten der L"ange \(n\). Matrizen der Gr"osse \(n\times 1\) heissen
Zeilen der L"ange \(n\).

\begin{definition}{Addition}{}
Matrizen gleicher Gr"osse kann man eintragsweise Addieren: \$A,B \(\in\) K\(^{\text{m}\texttimes{}\text{
n}}\) \(\rightarrow\) (A+B)\(_{\text{ij}}\):=\$A\(_{\text{ij}}\)+B\(_{\text{ij}}\)\$\$
\end{definition}

\begin{definition}{Multiplikation}{}
Matrizen kann man mit Zahlen multiplizieren (indem man jeden eintrag mit dieser
Zahl multipliziert).

\((\lambda \cdot A)_{ij}:=\lambda \cdot A_{ij}\).
\end{definition}

\begin{definition}{Produkt}{}
Wenn die Breite von \(A\) mit der H"ohe von \(B\) "ubereinstimmt, kann man das
Produkt \(A\cdot B\) definieren: \\
\(A\cdot B:=(A\cdot b_1\; \cdots \; A\cdot b_n)\) mit \(B=(b_1\; \cdots\; b_n)\) (Spalten)
mit \(A\cdot B \in K^{p\times n}\)
\end{definition}

\subsection{Eigenschaften der Matrix-Multiplikation}
\label{sec:org8eaaee0}
\begin{notation}[] \label{}
\begin{itemize}
\item wenn \(\alpha_1,...,\alpha_n\in K\) dann notieren wir \(\alpha_1+...+\alpha_n :=
   \sum_{i=1}^{n}{\alpha_i}\)
\item analog \(\alpha_1\cdot ...\cdot\alpha_n := \Pi_{i=1}^{n}{\alpha_i}\)
\end{itemize}
\end{notation}

\begin{relation}
Es gilt dann mit \(A=(a_{ij})_{\substack{i=1,p}}\) : \((A\cdot
x))_i=\sum_{j=1}^{m}{a_{ij}\cdot x_j,\, i=1,p}\) \\

Insbesondere gilt: \((A\cdot b_k)_i\) Aber \((A\cdot b_k)_i = (A\cdot B)_{ik}\) und \((b_k)_j=b_jk\)
\end{relation}

\begin{relation}
Matrixmultiplikation ist \emph{linear}: \(A\cdot (\lambda B_1 + \lambda_2 A B_2)\)
Analog:
\end{relation}

\begin{proof}[] \label{}
Sei \(C=A\cdot (\lambda_1 B_1 + \lambda_2 B_2)\)
\end{proof}

\begin{relation}
Die Matrixmultiplikation ist assoziativ: \(A\cdot (B\cdot C)=(A\cdot B)\cdot C\)
\end{relation}

\begin{proof}[] \label{}

\end{proof}

\begin{definition}{Einheitsmatrix}{}
Die Einheitsmatrix der gr"osse \(r\) ist die Matrix, die auf der Hauptdiagonale
(links-oben nach rechts unten) Einsen und sonnst Nullen hat. Beizeichnung \(E_r\)
oder \(1_r\).
\end{definition}

\begin{definition}{Kronecker-Symbol}{}
Das Kronecker-Symbol ist definiert als: 

Also gilt: \((Er)_{ij}=\delta_{ij}\).
\end{definition}

\begin{theo}{}{} \
F"ir alle \(A\in K^{p\times m}\) gilt:
\begin{itemize}
\item \(E_p\cdot A=A\)
\item \(A\cdot E_m =A\)
\end{itemize}
\end{theo}

\begin{proof}[] \label{}

\end{proof}


\begin{notation}[] \label{Vorsicht!}
Die Matrix Multiplikation ist nicht Konjunktiv: \(A\cdot B\not= B\cdot A\) im
Allgemeinen, selbst wenn beide Produkte definiert sind und die gleiche Gr"osse
haben.
\end{notation}

\begin{exa}[] \label{}

\end{exa}

\begin{notation}[] \label{}
Die \(i\text{te}\) Spalten der Einheitsmatrix wird durch \(e_i=()\) bezeichnet.
\end{notation}

\begin{definition}{Transposition}{}
Sei \(A\in K^{m\times n}\). Die transponierte Matrix \(A^{T}\in K^{n\times n}\) ist
definiert durch \((A^T)_{ij}:=A_{ji}\). Also ist die i-te Zeile der Einheitsmatrix \((e_i)^T\)
\end{definition}

Wie l"ost man nun das LGS \(A\cdot x=b\)? Man bringt die erweiterte
Koeffizientenmatrix in die reduzierten Zeilenstufenform.

\begin{relation}
Die Basisspalten in der reduzierten Zeilenstufenform sind von der Form, wo \(e_i\)
die i-te Spalte der Einheitsmatrix \(1_r\) ist. \(r <= m\).
\end{relation}

Wenn die ersten \(r\) Spalten Basisspalten sind, dann sieht die reduzierte
Zeilenstufenform so aus:


Dann sehen die L"osungen so aus:
\begin{relation}
\begin{enumerate}
\item Es gibt keine \(\iff\) \(b''\not= 0\)
\item Wenn \(b''=0\), dann sehen die L"osungen so aus: \[x=+\]
\end{enumerate}
\end{relation}

Proposition: Sei \(A\in k^{m\times n}\). Das homogene LGS der Form \(L=\{\phi
t \}\) fuer ein \(r\geq 0, \phi\)

\(\rightarrow\) es gibt \(n-r\) freie Parameter, die die Loesungsmenge beschreiben.

\textbf{Anmerkung} Ein homogenes LGS \(A\cdot x=0\) mit  hat immer eine L"osung \(x\not=
0\) (es gibt mindestens eine freie Variable).


\begin{exa}[] \label{}
Finde ein reelles Polynom von Grad 2.

Die Frage ist aequivalent zu dem LGS:
\end{exa}

\begin{definition}{}{}
Die Menge der Polynome vom Grad h"ochstens \(n\) mit Koeffizienten in \(K\) ist
durch \(K[t]_n\) berechnet.
\end{definition}

\section{Vektorra"ume}
\label{sec:org4906e00}
\begin{definition}{Vektorraum}{}
Ein \(k\) - Vektorraum \(V\) ist eine Menge zusammen mit den Operationen und mit
folgenden Eigenschaften:
\begin{itemize}
\item Addition \(+:\, V\times V \mapsto V, (v_1,v_2)\mapsto v_1+v_2\)
\begin{enumerate}
\item kommutativ
\item assoziativ
\item \(\exists 0 \in V\) mit \(0+v=v+0=v\) \(v \in V\)
\end{enumerate}
\item Skalarmultiplikation \(+:\, V\times V \mapsto V, (v_1,v_2)\mapsto v_1+v_2\)
\begin{enumerate}
\item assoziativ
\item distributiv bez. addition
\item \(1\cdot v = v\), \(v\in V\)
\end{enumerate}
\end{itemize}
\end{definition}

\begin{exa}[] \label{}
\begin{enumerate}
\item \(K\) ist selbst ein Vektorraum mit \(+\) und \(\cdot\)
\item \(K^{n}:=K^{n\times 1}\) ist ein K-Vektorraum mit:
\begin{enumerate}
\item Addition
\item Skalarmultiplikation
\end{enumerate}
\item \(K^{m \times n}\), eine Matrix der Gr"osse \(m\times n\) mit Eintr"agen in K,
ist ein K-Vektorraum mit Addition und Skalarmultiplikation von Matrizen:
\item \(K[t]_n\) ist ein K-Vektorraum:
\item \(K[t]:=\{a_n\cdot \}\) - alle Polynome mit Koeffizienten in \(K\) bilden einen
K-Vektorraum mit gleichen Operatoren.
\item Sei \(X\) X eine Menge (z.B. \(X=\mathbb{R}\)) \(Fun(X,K)=\{\}\) ist ein K-Vektorraum:
\begin{enumerate}
\item Addition \((f_1 + f_2)(x):= f_1(x)+ f_2(x)\), \(x\in X\)
\item Miltiplikation
\end{enumerate}
\item Sei \(A\in K^{m\times n}\). Die L"osungen von dem homogenen LGS bilden einen
Vektorraum:
\begin{itemize}
\item Wenn \(x_1,x_2\) L"osungen sind, dann gilt: Also ist die Menge der
L"osungen auch ein Vekorraum bzgl. der Operatoren aus \(K^n\)
\end{itemize}
\end{enumerate}
\end{exa}

\begin{notte}[] \label{}
Bei der Entwicklung der Vektorraumtheorie ist es oft n"utzlich, an das Beispiel
\(V=K^n\) zu denken.
\end{notte}

\subsection{Vektorraumtheorie}
\label{sec:org432e282}
Sei \(V\) ein K-Vektorraum.

\begin{definition}{Linearkombination}{}
Seien \(v_1, v_2\). Die Linearkombination mit Koeffizienten ist der Vektor.
\end{definition}

\begin{definition}{Triviale Linearkombination}{}
Eine Linearkombination heist trivial wenn \(\lambda_1 = \lambda_2 = ... =
\lambda_n = 0\). (\emph{Nichttrivial} wenn mindestens ein \(\lambda_i\not= 0\)).
\end{definition}


\begin{definition}{}{}
Die Menge der Vektoren heist linear Unabh"angig wenn: \$\$ (Nur die Triviale
linearkombination ergibt 0). Andernfalls heist die Menge linear abh"angig.
\end{definition}

\begin{exa}[] \label{}
\(\begin{pmatrix}1\\0\end{pmatrix}\begin{pmatrix}0\\1\end{pmatrix}\)
\end{exa}

\begin{exa}[] \label{}
\(\{v_1,v_2\}\) sind linear abhaengig wenn sie Proportional sind.

\begin{proof}[] \label{}

\end{proof}
\end{exa}

\textbf{Lemma} Die Menge ist linear abh"angig. \(v_i\) ist eine Linearkombination von \$\$

\begin{proof}[] \label{}
Wenn \(v_i=\lambda_1 v_1\), dann \(0=\) Denn \(-1\) ist ein nicht-trivialer
Linearfaktor.

\((\implies)\) Nach Definition gibt es eine nichttriviale Linearkombination: als
\(\exists i : \lambda_i \not= 0\) Also gilt  folglich 
\end{proof}

\begin{notte}[] \label{}
Eine L"osung des LGS \(Ax=b\) ist eine Spalte \$\$ mit 
Deis heisst, das LGS \(Ax=b\) zu l"osen ist genau linearkombinationen von spalten
zu finden, welche \(b\) ergeben.
\end{notte}

\textbf{Lemma} Seien die Vektoren linear unabh"angig. Ein Vektor \(v\) ist genau dann
eine Linearkombination von \(v_1,...,v_n\) wenn linear abh"angig ist.

\begin{proof}[] \label{}
(\(\implies\)) Sei Dann gilt , also ist linear ab"angig.
Sei .. linear abh"angig  Dann 

Es gilt: \(\lambda \not= 0\) (wenn  .. linearunabh"angig.) Also gilt \(v=-\lambda_1\)
\end{proof}

\textbf{Lemma} Sei \(v=\lambda\) eine Linearkombination von \(v_1,...,v_n\). Diese
Darstellung ist eindeutig ganau dann, wenn linear unabh"angig sind.

\begin{proof}[] \label{}
\((\implies)\) Sei die Darstellung eindeutig \$v=..\$ Wenn jetzt \$\$, dan gilt \(v=\)
Eindeutigkeit der Darstellung ergibt:

Seien \(v_1,..,v_n\) linear unabh"angig, sei 
Dann gilt: \(\rightarrow\) lineare Unabhaengigkeit von erzwingt Korrolar: Wenn die
Spalten von \(A\) linear unabhaenig sind, hat das LGS \(Ax=b\) h"ochstens eine
L"osung, folglich hat \(Ax=0\) genau eine L"osung x=0.
\end{proof}

\subsection{Geometrische Deutung der linearen Abh"angigket}
\label{sec:org95b9a1d}
\begin{notte}[zu geometrischer Interpretation] \label{}
Wichitige Beispiele von Vektorr"aumen sind: \(V=\mathbb{R}^2\) (Ebene),
\(V=\mathbb{R}^3\) (3D-Raum).
\end{notte}

Seien \(v_1, v_2\) nicht proportional.
In drei Dimensionen:
\begin{relation}
\begin{itemize}
\item wenn \(v_3\) in \(E\) liegt, dann ist \(v_3\) eine Linearkombination \(v_1, v_2\)
\item wenn nicht, dann linear Unabhaenig (sonst in Ebene.)
\end{itemize}
\end{relation}

\subsection{Lineare unabhangigkeit in R"aumenx}
\label{sec:orgea5b4b9}
\textbf{Proposition} Seien \(v_1,...,v_n \in \mathbb{V}\) linear unabhaenig, seien \(W_1,
..., W_n \in \mathbb{V}\) so dass jedes \(w_i\) eine Linearkombination von
\(v_1,...,v_n\) ist. Wenn \(m>n\), dann sind \(w_1,...,w_n\) linear abhaengig.

\begin{proof}[] \label{}
Seien 
\begin{align*}
w_1= a_{11} v_1 + a_{12} v_2 + ... + a_{nn} v_n \\
w_1= a_{21} v_1 + a_{22} v_2 + ... + a_{nn} v_n \\
\end{align*}
Wir suchen \(\lambda\) (*). Das ist "aquivalent zu:

Dies ist nach linearer Unabhaenig von \ldots{} "Aquivalent zu:

Das heist (*) ist "aquivalent zu einem homogenen LGS aus \(n\) Gleichungen mit \(m\)
Variablen. \(n<m\), also gibt es eine L"osung $\ldots{}$, die ungleich \(0\) ist. \(\implies\)
sind linear unabh"angig.
\end{proof}

\textbf{Korrolar} Je drei Vektoren in \(\mathbb{R}^2\) sind linear unabh"angig, je \(n+1\)
Vektoren in \$\$ sind linear unabh"angig.

\begin{proof}[von Korrolar] \label{}
Seien \(e_i\) die Spalten der Einheitsmatrix sind linear unabh"angig. 

Dies zeigt auch, dass jeder Vektor in \(R\) eine Linearkombination von \(e_i\) ist. 
\end{proof}

\begin{definition}{}{}
Sei V ein \(K-\) Vektorraum, \(U\subseteq V\) eine Teilmenge von V. \(U\) heist
untervektorraum wenn:
\begin{enumerate}
\item \(V\not= \varnothing\)
\item 

\item 
\end{enumerate}

In anderen Worten: Eine Teilmenge von \(V\) die selbst ein Vektorraum ist bzgl.
der von \(V\) vererbten Operationen.
\end{definition}

\begin{notte}[] \label{}
(1) und (3) \(\implies\) \(0\in U\)
\end{notte}

\begin{definition}{}{}
Sei \(S \in V\) eine Teilmenge. Der von \(S\) erzeugte Vektorraum (lineare H"ulle
von \(S\)) \(<S>:=\{\}\) (Menge aller Linearkombinationen von Vektoren in \(S\)).

Alternative Notation: \(<s>=\text{span S}\).
\end{definition}

\begin{notte}[] \label{}
\(<s>\) ist der kleinste Untervektorraum in \(V\), der \(S\) enth"alt. 
\(<\varnothing >:=\{0\}\)
\end{notte}

\begin{exa}[] \label{}
Seien \(v_1, v_2 \in \mathbb{R}^3\).
\(<v1,v2>\) ist eine Gerade wenn \(v_1,v_2\) linear abh. Ist Ebene wenn \(v_1,v_2\)
linear unbh.
\end{exa}

\begin{definition}{}{}
\(S\in V\) heisst Erzeugendensystem wenn \(<S>=V\). (S spannt den Vektorraum auf.)

ist ein Erzeugendensystem: jeder Vektor in \(V\) ist eine Linearkombination von:

\(\lambda_i\) sind nicht unbedingt eindeutig bestimmt, weil nicht linear unabh.
vorrausgesetzt waren.
\end{definition}

\begin{definition}{}{}
Ein Erzeugendensystem \(B\in V\) heisst basis, wenn es linear unabh. ist. Nach dem
Lemma ueber Eindeutigkeit der koeffzienten der Linearkombination gilt: \(B=\{v_1,
..., v_n\}\) ist eine Basis genau dann, wenn f"ur jeden Vektor \(v \in V\) gibt es
 eindeutig bestimmte Zahlen.
\end{definition}

\begin{definition}{}{}
Ein Vektorraum \(V\) heisst endlich dimensional, wenn er ein endliches
erzeugendensystem besitzt. (= wird von endlich vielen Vektoren aufgespannt).
\end{definition}

\begin{theo}{}{}
Jeder endlichedimensionale Vektorraum \(V\)  hat eine Basis, Je zwei Basen von \(V\)
haben gleich viele Elemente.
\end{theo}

\begin{proof}[] \label{}
Sei \(S\) ein endliches Erzeugendensyste von \(V\), Wenn \(S\) lin. unabh. ist, ist es
eine Basis und wir haben gewonnen. Wenn \(S\) linear abh"angig ist $\implies$
(lemma) einer von den Vektoren in \(S\) ist eine Linearkombination von den
anderen. Das entfernen dieses Vektors "andert die Tatsache nicht, das \(S\) den
Vektorraum aufspannt. Jetzt haben wir eine kleinere Menge und fangen von vorne
an. Da \(S\) endlich ist und durch entfernen Vektoren kleiner wird haben wir am
Ende eine Basis.
\(\rightarrow\) Wir haben eine Basis.

Seien \(S, S'\) zwei Basen. Da \(S\) eine Basis ist, ist jedes element von \(S'\) eine
linearkombination in \(S\). Die elemente von \(S\) sind linear unabh. (weil Basis).
Wenn also \(m>n\), dann folgt aus der Proposition, dass \(S'\) linear abh. ist, was
unm"oglich ist, da \(S'\) eine Basis ist. Also \(m \leq n\) und aus Symmetriegr"unden folgt auch \(n \leq m\).
\end{proof}

\begin{definition}{}{}
Sei \(V\) ein endlichdimensionaler Vektorraum. Die Anzahl der Vektoren in einer
(folglich in jeder) Basis von \(V\) heist Dimension von V. \emph{Bezeichung}: \(\dim V\).n
\end{definition}

\begin{exa}[s] \label{}
\(\dim K^{n}=n\) weil \ldots{} eine Basis bilden. 
\end{exa}

\textbf{Frage}: kannn man eine lineare unabh"angige Menge \(S\in V\) zu eine Basis
erweitern?.

\textbf{Proposition} Jede linear unabh"angige Teilmenge \(S\in V\) eines
endlichdimensionalen Vektorraumes \(V\) ist in einer maximalen linear
unabh"angigen Teilmenge enthalten. Eine maximal linear unabh. Teilmenge von \(V\)
st eine Basis von \(V\).

\begin{definition}{Maximal linear unabh"angige Teilmengen}{}
Eine Teilmenge \(S'\in V\) ist maximal linear unabh., wenn aus \(S\)
linear unabh. folgt.
\end{definition}

\begin{proof}[1] \label{}
Sei linear unabh.
Zwei F"alle: entweder ist \(S\) schon maximal (dann sind wir fertig) oder man kann
S erweitern. Wenn wir \(S\) erweitern k"onnen, f"ugen wir neue Vektoren hinzu, bis
wir es nicht mehr tun koennen, ohne lineare unabh. zu verletzen.

Dieser Prozess endet, weil eine linear unabh. Teilmenge h"ochstens von \(V\)
hoechstens \(\dim V\) viele Vektoren enthalten kann. (Prop. "uber lineare unabh.
von vektoren aus linearkombinartionen der Basis.)
\end{proof}

\begin{proof}[2] \label{}
Sei \(S\in V\) maximal linear unabh"angig. Wir habe zu zeigen: \(<S>=V\) (Def. einer
Basis). Wenn $\ldots{}$ d.h. aber, aber \(S\cup {v}\) ist linear unabh. (lemma) $\implies$
\(S\) dann nicht maximal.
\end{proof}

\textbf{Korrolar} Man kann jeder lienar unabh. Teilmenge \(S\in V\) zu einer Basis
erweitern.

\begin{notte}
Wenn \(V=K^n, S\in V\) linear unabh. \(\rightarrow\) man kann zur erweiterung passende
Spalten der Einheitsmatrix.
\end{notte}

\begin{notte}[] \label{}
Man kann bei der obrigen Proposition das Wort "endlichdimensional" fallen
lassen, aber es braucht ein bisschen mehr Mengenlehre (Auswahlaxiom).x
\end{notte}


\begin{theo}{}{}
Sei V ein endlichdimensionaler Vektorraum und \(U\in V\) ein Untervektorraum. Dann
gillt: \(\dim U \leq \dim V\).  \(\dim U = \dim V \iff U=V\)
\end{theo}

\begin{proof}[] \label{}
Sei eine maximale linear unabh. Teilmenge in U. (so eine Teilmenge existiert
weil V endlich ist.)
Nach Proposition (2) ist \ldots{} eine Basis in \(U\), also gilt \(\dim U = k\) Erweitere
\ldots{} zu einer Basis in V \ldots{}. 

(2) \ldots{} trivial \ldots{} Sei \ldots{} eine Basis in U. Erweitere sie zu einer Basis in
\(V\). Diese Basis in V muss aber wegen \ldots{} gleich viele Vektoren haben. \ldots{}. ist
eine Basis in \(V\) \ldots{}
\end{proof}


\begin{definition}{}{}
Sei \(V\) ein Vektorraum \$\$ eine Basis in V. Die Zahlen \((\lambda_1,...\lambda_n)\)
heissen Koordinaten bzgl. des Vektors. Die Spalte \ldots{} heisst Koordinatenspalte
dieses Vektors bzgl. \(B\).
Die Definition einer Basis garantiert, dass hierdurch eine Bijektion \ldots{} entsteht.
\end{definition}

\textbf{Warnung} Diese Korrespondenz kommt auf die Wahl der Basis an.

\begin{exa}[] \label{}

\end{exa}

\begin{exa}[] \label{}
Sei \(Ax=0\) ein h. LGS
\end{exa}

\textbf{Aus Uebungen} \ldots{}

\textbf{Lemma} Die Spalten von \(\Phi\) bilden eine Basis in L.
\begin{proof}[] \label{}
Die Spalten von \(\Phi\) sind linear unabhaenig. (sonst abb. nicht injektiv)

Ferner spannen sie das ganze L auf (Surjektiv). 
\end{proof}

\textbf{Frage} Gegeben Basen \ldots{} in \(V\), und einem Vektor \(v\in V\). Wie rechnet man die
Koordinaten bezgl. B in Koordinaten bzgl. B' um.

Sprachweise ist: B ist alte Basis und B' ist neue Basis.
So gilt:
\begin{relation}
\ldots{} Dr"ucken wir Vektoren von B' bzgl. Vektoren von B aus: \(V\) \ldots{}
wir erhalten \(C\) 
(Die Spalten von C sind Koordinaten der "neuen" Basis bzgl der alten Basis. )

Also gilt: \ldots{}
\(\lambda = G\cdot \lambda'\)
\end{relation}

\textbf{Frage} Ist \(D\) in diesem Fall immer invertierbar? (Ja, aber wir brauchen mehr Theorie.)

\subsection{Lineare Abbildungen zwischen Vektorr"aumen}
\label{sec:orgb4c03c4}
\begin{definition}{Lineare Abbildung}{}
Seien \(V, W\) zwei K-Vektorr"aume. Eine Abbildung. \(f\) heist linear wenn:


(Strukturell kopatiebel.)
\end{definition}

\begin{exa}[] \label{}
\(W=K^n,\; W=K^n\)

Es gilt tats"achlich \(A(x_1+x_2) = A\cdot x_1 + A\cdot x_2\)\ldots{}
\end{exa}

\begin{exa}[] \label{}
\(V=\mathbb{R}[x]_5,\;W=\mathbb{R}[x]_4\) Ableitung ist lineare abbildung.
\end{exa}

\begin{exa}[] \label{}
Relle Funktionen
\end{exa}

\begin{definition}{}{}
Sei eine Lineare Abbildung. Der Kern von \(f\) wird definiert als:
\end{definition}

\begin{exa}[] \label{}

\end{exa}

\textbf{Beobachtung} Kern von \(f\) ist ein Untervektorraum von \(V\): \ldots{}

Errinerung: f"ur djede Abbildung \(f\) existiert ein Bild:

\begin{exa}[] \label{}
Wenn, dann definiert A eine Lineare Abbildung

\begin{proof}[] \label{}
Definitionsgem"ass ist \(f: V\mapsto W\) surjektiv genau dann, wenn \(lm(f)=W\).
\end{proof}
\end{exa}


\textbf{Proposition} Sei \(f:\) linear Es gilt: \(f\) injektiv $\iff$ \(Ker(f)=\{0\}\)

\begin{proof}[] \label{}
\(f\) injektiv $\iff$ f"ur \(v_1\not= v_2 \in V\) gilt \(f(v_1)\not= f(v_2)\)
\end{proof}

\begin{definition}{}{}
Eine Bijektive Lineare Abbilfung \(f:V\mapsto W\) heisst Vektorraum Isomorphismus
zwischen \(V\) und \(V\). \(V\) und \(W\) heissen isomorph, wenn es einen
Vektorraumisomorphismus \(f: V\mapsto W\) gibt.
\end{definition}

\begin{exa}[] \label{}
Sei \(V\) ein Vektorraum, \(S\subseteq V = \{v_1, v_2, ..., v_n\}\). Die
Aufspannabbildung \ldots{} wird definiert als \ldots{}
\end{exa}

\textbf{Korrolar} \$S=\{v\(_{\text{1}}\), \ldots{}, v\(_{\text{n}}\)\} eine Basis $\implies \ldots{}$ ein Isomorphismus

\textbf{Korrolar} \(\dim V = n \iff V\) ismorph \(K^n\). (\ldots{} isomorphe Vektorraume haben
die gleiche Dimension)

\textbf{Beobachung} Wenn ... Isomorphismus $\implies \ldots{}$ ist auch ein Isomorphismus.

\subsubsection{Dimensionsformel}
\label{sec:org9a58004}
\begin{theo}{}{}
Sei \(f\) eine Lineare Abbildung, sei \(V\) endlich dimensional. Dann gilt: 
\end{theo}

\textbf{lemma} sei \(f\) wie oben. Sei \(U \subseteq Ker(f)\)  Dann ist $ \ldots{}$ ein Isomorphismus

\begin{proof}[des Lemmas] \label{}
$\ldots{}$ ist surjektiv nach Konstruktion. Inkektiv $\iff \ldots{}$ Sei $\ldots{}$ . Dann gilt $\ldots{}$. 
\end{proof}

\begin{proof}[der Dimensionsformel] \label{}
W"ahle eine Basis \({e_1, ..., e_k}\) in \ldots{} und erg"anze sie zu einer Basis
\({e_1, ..., e_n}\) in \(V\).

Betrachte jetzt \(U:=<e_{k+1}, ..., e_n> \subseteq V\) Untervektorraum. Es gilt
Lemma. es gilt: \ldots{} weil .. eine Basis im Kern ist. und \ldots{} weil \(u\in U\) also
\ldots{}

Das Lemma sagt jetzt $\ldots{} $ist ein Isomorphismus. Ausserdem gilt f"ur $\ldots{} \implies$
\(f(V)=f(V)\) also $\ldots{} \implies \ldots{} $

Nun gilt nach Konstruktion von \(U\) $ \ldots{} $
\end{proof}

\subsubsection{Summe von Untervektorr"aumen}
\label{sec:org83dfe63}
\begin{definition}{}{}
Sei \ldots{} ein Vektorraum\ldots{}. 
\end{definition}
\begin{definition}{}{}
Die Summe von \ldots{} heisst direkt wenn \ldots{} 
\end{definition}

\textbf{Bemerkung} \ldots{}

In dieser Bezeichnung haben wir im Beseris der Dimensionsformel haben wir den
Ausgangsraum als eine direkte Summe dargestellt. 

\textbf{Bemerkung}: Dimensionsformel ist auch Rangformel.

\begin{definition}{Rang}{}
Sei \ldots{} linear. Der Rang von \(f\) ist \$rk\ldots{}\$
\end{definition}

\textbf{Proposition} Sei \ldots{} linear, endlichdimensional.
Dann gilt:
\begin{itemize}
\item f injekt. \ldots{}
\end{itemize}

Insbesondere gilt: \textbf{Korrolar} Ist $\ldots{}$, so ist f injektiv $\iff $f surjektiv.

\textbf{Proposition} Dimensionformel' \(\dim U_1 +U_2 =\dim..\)
\begin{proof}[] \label{}
ist stehts ein Untervektorraum, wenn Untervektorra"me sind. 

Betrachte die Abbildung .. Hierbei ist \ldots{} der Verktorraum der Paare mit
elementweisen operationen. (auch ''A"ussere Summe'', die Kollision der
Bezeichnunge \ldots{} fuer die direkte Summe zweier Unterr"aume/a"ussere Summe ist harmlos.)

Nun gilt \ldots{} 

Weiterhin gilt eine Basis in eine Basis in \ldots{} eine Basis in \ldots{}
Ferner gilt: \ldots{} 

\(Ker(f)\) \ldots{} (Unterraum)
\end{proof}

Was hat diese ganze Theorie mit Matrizen zu tun? Intuitiv: Abbildungen sind
\``geometrisch\'' und Matrizen sind Koordinatenform dieser geometrischen
Abbildungen.

\begin{exa}[] \label{}
Strecken in richtung von \(l_2\) mit Faktor 2. Wie beschreibt man \(f\) in
Koordinaten?4
\end{exa}

\subsubsection{Abbildunngsmatrix}
\label{sec:org00a1823}
\begin{definition}{}{}
Seien \(V,W\) zwei Vektorraume.
\(Hom_k(v,w)\)  ist selbst ein Vektorraum.
\end{definition}

Seien \(V, W\) endlichdimensional, \ldots{} Sei  Die Abbildungsmatrix ist definiert als
Matrix, deren Spalten die Koordinatenspalten von \ldots{} bzgl. der Basis \(C\) sind. 

\textbf{Vorsicht} H"angt von der Wahl der Basen B und C ab (\(f\) nicht!)

\begin{exa}[] \label{}
Rotation um \(\frac{\pi}{4}\) gegen Urzeigersinn.
\end{exa}

\textbf{Proposition} Seien \(V,W,B,C\) wie oben. Dann entsprechen die Abbildungsmatrizen
 \ldots{}den Abbildungen.  gennauer. Die Abbildung. Ist ein Isomorphismus von
 Vektorra"umen.

\begin{proof}[] \label{}
Aus Definition der \(M_C^B\) folgt sofort: \ldots{} also ist \ldots{} eine lineare
Abbildung:

Ist injektiv: wenn \(f\) dann gilt \(\rightarrow\) Kern ist injektiv.

Ist auch surjektiv: sei gegeben: Definiere eine Abbildung\ldots{} folgendermassen:

Ist linear und es gilt: \ldots{}
\end{proof}

Im Beweis haben wir unter anderem festgestellt:
\begin{relation}
\ldots{}

Das heisst: wenn \(v\)
\end{relation}

\begin{exa}[] \label{}
Wenn \(V=K^n, W=K^M\) dann hat \(V\) eine Basis \ldots{} Sei \(A\in K^{m\times n }\) 
\end{exa}

Seien \(V, W, Z\) drei Vektorraume \ldots{} Dann gilt \ldots{} . Seien \(B,C,D\) Basen in
\(V,W,Z\)

\textbf{Proposition} \ldots{}
\begin{proof}[] \label{}
Sei 
\end{proof}

\textbf{Bemerkung} Wenn \(V\) ein Vektorraum ist, \(B,B'\) zwei Basen, dann Haben wir die
 Basiswechselmatrix. Bezueglich der alten Basis. Es folgt sofort aus den
= Definitionen: \(S+\)
\subsection{Schlagworte:}
\label{sec:orgcf8c685}
\begin{itemize}
\item \(A\cdot B\) Zeilen von \(A\) mal Spalten von \(B\)
\item LGS L"osungen als Vektor!
\begin{itemize}
\item Keine nicht offensichtlich Schritte ueberspringen!
\item Paramatervektor und sine Elemente genau definieren!
\end{itemize}
\item k-te Spalte \((A)_k\)
\end{itemize}
\end{document}