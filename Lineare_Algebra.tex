% Created 2017-11-19 Sun 20:51
% Intended LaTeX compiler: pdflatex
\documentclass[11pt]{article}
\usepackage[utf8]{inputenc}
\usepackage{beton}
\usepackage{euler}
\usepackage{mathpazo} % add possibly `sc` and `osf` options
\usepackage{eulervm}
\usepackage[T1]{fontenc}
\usepackage{graphicx}
\usepackage{grffile}
\usepackage{longtable}
\usepackage{wrapfig}
\usepackage{rotating}
\usepackage[normalem]{ulem}
\usepackage{amsmath}
\usepackage{textcomp}
\usepackage{amssymb}
\usepackage{capt-of}
\usepackage{hyperref}
\usepackage{nicefrac}
\usepackage[a4paper, left=2cm, right=2cm, top=2cm, bottom=2cm]{geometry}
\setlength{\parfillskip}{0pt plus 1fil}
\setlength{\parindent}{0pt}
\usepackage[ngerman]{babel}
\usepackage{fancyhdr}
\usepackage{mathtools} % for xrightarrow
\usepackage{todonotes}
\pagestyle{fancy}
\fancyhead{}
\fancyfoot{}
\fancyhead[L]{\rightmark}
\fancyhead[R]{\thepage}
\renewcommand{\headrulewidth}{0.4pt}
\renewcommand{\footrulewidth}{0pt}
\newcommand{\gq}[1]{\glqq{}#1\grqq{}} % german quotes
\newcommand{\mcolor}[2][red]{\begingroup\color{#1}#2\endgroup }
\DeclareMathOperator{\mdim}{dim}
\DeclareMathOperator{\mKer}{Ker}
\DeclareMathOperator{\mIm}{Im}
\DeclareMathOperator{\mRg}{Rg}
\DeclareMathOperator{\mHom}{Hom}
\DeclareMathOperator{\mId}{id}
\DeclareMathOperator{\mVol}{Vol}
\DeclareMathOperator{\mDiag}{diag}
\DeclareMathOperator{\mEnd}{End}
\DeclareMathOperator{\mDeg}{deg}
\DeclareMathOperator{\Tr}{Tr}
\usepackage{tcolorbox}
\tcbuselibrary{theorems}
\newtcbtheorem[number within=section]{definition}{Definition}%
{colback=green!5,colframe=green!35!black,fonttitle=\bfseries}{th}
\newtcbtheorem[number within=section]{axiom}{Axiom}%
{colback=orange!5,colframe=orange!35!black,fonttitle=\bfseries}{th}
\newtcbtheorem[number within=section]{theo}{Satz}%
{colback=blue!5,colframe=blue!35!black,fonttitle=\bfseries}{th}
\newtcbtheorem[number within=section]{satz}{Satz}%
{colback=orange!5,colframe=orange!35!black,fonttitle=\bfseries}{th}
\newtcolorbox{comm}[1][]
{title=Kommentar,colback=black!5,colframe=black!35!black,fonttitle=\bfseries}
\newtcolorbox{relation}[1][]
{
colframe = blue!25,
colback  = blue!10,
halign = center,
#1,
}
\usepackage{etoolbox}
\usepackage{amsthm}
\usepackage{amssymb}
\usepackage{gauss}
\usepackage{stmaryrd}
\newtheorem{exa}{Beispiel}[section]
\newtheorem{expe}{experiment}[section]
\renewcommand*{\proofname}{\textbf{Beweis}}
\newtheorem{beobachtung}{Beobachtung}
\newtheorem{folgerung}{Folgerung}
\newtheorem*{notte}{Beachte}
\newtheorem*{notation}{Notation}
\newtheorem*{proposition}{Proposition}
\newtheorem*{lemma}{Lemma}
\newtheorem*{korollar}{Korollar}
\newtheorem*{bem}{Bemerkung}
\author{Nebnola, Julius Quasebarth, Valentin Boettcher}
\date{\today}
\title{Lineare Algebra (f"ur Physiker) I}
\hypersetup{
 pdfauthor={Nebnola, Julius Quasebarth, Valentin Boettcher},
 pdftitle={Lineare Algebra (Physiker) I},
 pdfkeywords={},
 pdfsubject={},
 pdflang={Germanq}}
\begin{document}

\maketitle
\tableofcontents

\maketitle
\newpage

\section{Vorwort}
\label{sec:org8d9fa1c}
Ihr habt hier die Mitschriften Valentin Boettchers vor euch. Er teilt eben Diese
"ausserst gern mit euch und freut sich "uber Feedback, Fehlerkorrekturen und
Verbesserungsvorschl"age. Kontaktiert ihn am besten via \href{mailto:valentin.boettcher@mailbox.tu-dresden.de}{Email} :).

Vor allem aber ist es wichtig zu verstehen, dass das Format dieses Skriptes
kein allumfassendes Kompendium ist und nur den Inhalt der Vorlesung abdeckt.
Wenn Valentin einmal ein Paar interessante Gedanken kommen, packt er sie
wohlm"oglich auch hinein, versucht aber immer deren Korrektheit zu
gew"ahrleisten. Auch Kommentare des Lesenden k"onnen Teil dieses Skriptes
werden.

Wie ihr bestimmt bis hierher bemerkt habt, ist Valentins Rechtschreibung
grausig: Also frisch ans Werk und Feedback geben.


Viel Vergn"ugen. \textbf{Mathe ist sch"on.}

\section{Mengenlehre}
\label{sec:orgd4be270}
In der modernen Mathematik fasst man Strukturen (R"aume, Fl"achen,
Zahlensysteme) als \emph{Mengen} und \emph{Abbildungen} auf.

\begin{definition}{Menge}{def-meng}
Eine Zusammenfassung von Objekten die \textbf{Elemente} der Menge heissen. Eine Menge ist
also eindeutig dadurch bestimmt, welche Elemente sie enth"alt.
\end{definition}

\begin{notation}\
\begin{itemize}
\item \(M=\{m_1,m_2,m_3,...\}\) - Aufz"ahlung
\begin{itemize}
\item \(\{...\}\) - Mengenklammern
\end{itemize}
\item \(M=\{x| P(x)\}\) - Eigenschaft
\begin{itemize}
\item Alle \(x\) mit der Eigenschaft \(P(x)\)
\end{itemize}
\end{itemize}
\end{notation}

\begin{exa}\
\begin{itemize}
\item \(n=\{\text{Nat"urliche Zahlen}\} = \{0,1,2,...\}\)
\item \(E=\{x|\text{x hat die Eigenschaft } P(x)\}\)
\end{itemize}
\end{exa}

\subsection{Wichtige Mengen}
\label{sec:orga565e26}
\begin{itemize}
\item \(\mathbb{N}=\{\text{Nat"urliche Zahlen}\} = \{1,2,...\}\)
\item \(\mathbb{Z}=\{\text{Ganze Zahlen}\} = \{...,-2,-1,0,1,2,...\}\)
\item \(\mathbb{Q}=\{\text{Rationale
   Zahlen}\}=\left\{\left.\displaystyle\frac{p}{q}\;\right\vert\begin{array}{c}p \in \mathbb{Z} \\ q \in \mathbb{N} \setminus \{0\}\end{array}\right\}\)
\item \(\mathbb{R}=\{\text{Reelle Zahlen}\}\)
\end{itemize}

\subsection{Beziehungen zwischen Mengen}
\label{sec:orgcffbbc3}
\begin{definition}{Mengenbeziehungen}{def-teilmenge}
Seien \(A,B\) zwei Mengen. 
\begin{enumerate}
\item \(A\) heisst \textbf{Teilmenge} von B, wenn f"ur jedes Element \(a\in A\) gilt: \(a\in B\).
\item Es sei die Menge \(C = \{a|a\in A \text{ und } b\in B\}\), so heisst \(C\) \textbf{Durchschnitt} von \(A\) und \(B\).
\item Es sei die Menge \(C = \{a|a\in A \text{ oder } b\in B\}\), so heisst \(C\) \textbf{Vereinigung} von \(A\) und \(B\).
\end{enumerate}
\end{definition}

\begin{notation}\
\begin{itemize}
\item \(\in\) Element von: \(x\in X\) - \gq{x ist Element von X}
\item \(\subseteq\) Teilmenge: \(A\subseteq B\) - \gq{A ist eine Teilmenge von B}
\item \(\subset\) echte Teilmenge: \(A\subset B \iff A\subseteq B \land A \neq B\)
\item \(\cap\) Durchschnitt: \(A\cap B = \{a|a\in A \text{ und } b\in B\}\)
\item \(\cup\) Vereinigung: \(A\cup B = \{a|a\in A \text{ oder } b\in B\}\)
\item \(\varnothing\) Leere Menge: enth"alt keine Elemente und ist Teilmenge aller Mengen
\item \(\setminus\) Mengendifferenz: \( A\setminus B = \{a|a\in A \land a\notin B \} \)
\item \(A\times B\) Direktes Produkt: \(A \times B = \{(a,b)|a\in A\land b\in B \} \)
\begin{itemize}
	\item \((a,b)\): geordentes Paar mit dem ersten Element \(a\) und dem zweiten
	Element \(b\).
\end{itemize}
\end{itemize}
\end{notation}

\begin{exa}
\(\mathbb{N}\subseteq \mathbb{Z}\), aber \(\mathbb{Q} \nsubseteq \mathbb{Z}\): \(\frac{1}{2} \notin \mathbb{Z}\)
\end{exa}

\begin{exa}
F"ur \(A = \{1,2,3,4,5\}\) und \(B = \{2,3,10\}\):
\begin{itemize}
\item \(A\cap B = \{2,3\}\)
\item \(A\cup B = \{1,2,3,4,5,10\}\)
\end{itemize}
\end{exa}

\begin{definition}{Leere Menge}{}
Die leere Menge \(\varnothing\) ist die (eindeutig bestimmte) Menge, die kein Element enth"alt.
\end{definition}

\begin{exa}
\(\{\pi\} \cap Q = \varnothing\)
\end{exa}

\begin{definition}{Differenz}{}
Die Differenz zweier Mengen \(A, B\) wird definiert als \(A\setminus B = \{a\in A | a\not\in
B\}\) (Elemente aus \(A\), die nicht in \(B\) liegen). 
\end{definition}

\begin{definition}{Direktes/Kartesisches Produkt}{}
Wenn \(A,B\) zwei Mengen sind dann ist die Menge der Paare \((a,b)\) und \(a\in A,
b\in B\) das direkte (kartesische) Produkt von \(A\) und \(B\) (\(A\times B\)).
\end{definition}

Analog gilt: \(A_1\times A_2\times ... \times A_n = \{(a_1,...,a_n)| a_1\in A_1,...,a_n\in A_n\}\)

\begin{exa}
\(\mathbb{R}^n=\mathbb{R}\times ... \times \mathbb{R} =  \{(x_1,...,x_n)| x_1\in \mathbb{R},...,x_n\in \mathbb{R}\}\)
\end{exa}

Geometrie \(m\) der Ebene mit Koordinaten \(=\) Untersuchung von Konstruktionen in
\(\mathbb{R}^2=\mathbb{R}\cdot\mathbb{R}\).

\begin{definition}{Komplemen"armenge}
Seien \(A,M\) Mengen und \(A\subseteq B\) so ist \(A^c = M\setminus A\) und heisst
\textbf{Komplement"armenge} zu \(M\).
\end{definition}

Seien \(A,B,M\) Mengen und \(A\subseteq M\) und \(B\subseteq M\), so gilt:
\begin{relation}
\begin{enumerate}
\item \((A\cup B)^c = A^c \cap B^c\)
\item \((A\cap B)^c = A^c \cup B^c\)
\item \((A^c)^c = A\)
\item \(A\cup A^c = M\)
\end{enumerate}
\end{relation}

\begin{notte}
Es gelten auch alle Identit"aten f"ur Mengen.
\end{notte}


\subsection{Abbildungen zwischen Mengen}
\label{sec:org4ef8946}
\begin{definition}{Abbildung}{}
Seien \(X,Y\) Mengen. Eine Abbildung \(f\) von \(X\) nach \(Y\) (Bez: \(f:X\rightarrow
Y\)) ist eine Vorschrift, die jedem Element \(x\in X\) ein Element von
\(y\in Y\) Zuordnet.
\end{definition}


\begin{notation}
Man schreibt: \(x\mapsto f(x)\) - ''x wird auf \(f(x)\) abgebildet'' = ''dem \(x\in
X\) wird ein \(f(x)\in Y\) zugeordnet.''
\end{notation}

\begin{exa}\
\begin{itemize}
\item \(f(t)=t^2+1\) definiert eine Abbildung \(f: \mathbb{R}\to \mathbb{R}, t\mapsto f(t)=t^2+1\)
\item \(g(t)= \frac{t^2+1}{t-1}\) definiert eine Abbildung \(g: \mathbb{R}\setminus\{
   1\}\to \mathbb{R}, t\mapsto  \frac{t^2+1}{t-1}\)
\item \(h: S=\{\text{Teilnehmer der Vorlesung}\}\to N, s\mapsto Geburtsjahr(s)\)
\end{itemize}
\end{exa}

\subsubsection{Spezielle Abbildungen}
\label{sec:orge512a75}
\begin{relation}
\begin{enumerate}
\item F"ur jede Menge \(X\) ist die \textbf{Indentit"atsabbildung} auf \(X\) definiert durch \(Id_x:X\to X, x\mapsto x\).
\item Gegeben seien Mengen \(A,B\). Die Abbildung \(\pi_A: A\times B \to A, (a,b)
    \mapsto a\) heisst \textbf{Projektionsabbildung} von \(A\times B\) auf \(A\).
\item Seien \(X,Y\) Mengen, sei \(y_0 \in Y\). Dann heisst die Abbildung \(f: X\to
    Y, x\mapsto y_0\) eine \textbf{konstante Abbildung} (mit dem wert \(y_0\)).
\end{enumerate}
\end{relation}

\begin{exa}\
\begin{itemize}
\item Identit"atsabbildung: \(f(x)=x\)
\item konstante Abbildung: \(f(x)=1\)
\item Projektionsabbildung: \(f(x,y)=x\)
\end{itemize}
\end{exa}

\subsubsection{Bild und Urbild}
\label{sec:org006b051}
\begin{definition}{Bild und Urbild einer Funktion}{}
Sei \(f: X\to Y\) eine Abbildung.
\begin{itemize}
\item Sei \(A\subseteq X\). Dann heisst \(f(A):=\{f(a)|a\in A\}\) das Bild von A.
\item Sei \(B\subseteq Y\). Dann heisst \(f^{-1}(B):=\{a\in A|f(a)\in B\}\) das Urbild von \(B\).
\end{itemize}
\end{definition}

\begin{notte}
Das Bild und das Urbild f"ur eine \emph{Menge} einer Funktion ist wieder eine \emph{Menge}.
\end{notte}


\begin{notte}
\(f^{-1}\) ist keine Abbildung, sonder nur ein formales Symbol!s
\end{notte}

\subsubsection{Einige Eigenschaften von Funktionen}
\label{sec:org1909a84}
Seien \(X,Y\) Mengen, \(f: X\to Y\) eine Abbildung. \(f\) heist:
\begin{relation}
\begin{enumerate}
\item \textbf{Injektiv}, wenn f"ur \(x\in X\not = x' \in X\) gilt: \(f(x) \not = f(x')\)
\begin{itemize}
\item Keine Verklebung von Punkten!
\end{itemize}
\item \textbf{Surjektiv}, wenn f"ur \(y\in Y\) ein \(x\in X\) existiert mit \(f(x)=y\).
\begin{itemize}
\item Keine Abbildung auf eine echte Teilmenge von \(Y\)!
\end{itemize}
\item \textbf{Bijektiv}, wenn \(f\) injektiv und surjektiv ist.
\end{enumerate}
\end{relation}

\begin{exa}\
\begin{enumerate}
\item \(f: \mathbb{R} \to \mathbb{R}, t\mapsto t^2\) 
\begin{itemize}
\item ist nicht injektiv:
\end{itemize}
\end{enumerate}
\(-1\mapsto 1\)
\begin{itemize}
\item ist nicht surjektiv: f"ur \(-1\in \mathbb{R}\) gibt es kein \(t\in\mathbb{R}\)
mit \(t^2=-1\)
\end{itemize}
\begin{enumerate}
\item \(g: \mathbb{N}\to\mathbb{Z}, n\mapsto-n\)
\begin{itemize}
\item ist injektiv: \(m\ne n\implies g(m)=-m \ne -n = g(n)\)
\item ist nicht surjektiv: f"ur \(1\in \mathbb{Z}\) gibt es kein \(n\in \mathbb{N}\)
mit \(-n=1\)
\end{itemize}
\item \(h: \mathbb{R}\to\mathbb{R},t\mapsto t^3\) ist Bijektiv ("Ubung)
\end{enumerate}
\end{exa}

\subsubsection{Inverse Abbildung zu einer bijektiven Abbildung}
\label{sec:orgaae7124}
\begin{definition}{Inverse Abbildung}{}
Sei \(f:X\to Y\) bijektiv. Sei \(y\in Y\). Definiere eine Abbildung \(f^{-1}:
Y\to X\) so: \(f^{-1}(y)=x\) mit der Eigenschaft \(f(x)=y\).
\end{definition}

Dies ist wohldefiniert (diese Vorschrift definiert tats"achlich eine Abbildung)
weil:

\begin{relation}
\begin{itemize}
\item Das x mit der gew"unschten Eigenschaft existiert f"ur jedes \(y\in Y\), weil
\(f\) surjectiv ist.
\item F"ur jedes \(y\in Y\) existiert h"ochstens ein \(x\in X\) mit der gew"unschten
Eigenschaft, weil \(f\) injektiv ist.
\end{itemize}
\end{relation}

\begin{notte}
Wenn die Abbildung \(f\) bijektiv ist, hat \(f^{-1}(A)\) f"ur ein \(A\subseteq Y\) a
priori zwei Bedeutungen:
\begin{itemize}
\item Urbild von \(A\) unter f
\item Bild von \(A\) von \(f^{-1}\)
\end{itemize}

Wenn \(f\) bijektiv ist, stimmen aber diese Mengen "uberein. (Bew. "Ubung)

\textbf{Aber}: Wenn \(f\) nicht bijektiv ist, hat \(f^{-1}\) nur einen Sinn: Urbild!
\end{notte}

\subsubsection{Verkn"upfung von Abbildungen}
\label{sec:org540b965}
\begin{definition}{Verkn"upfung}{}
\(f: X\to Y, g: Y\to Z\) ist die verkn"upfung \(g\circ: X\to Z\) definiert
als \(g\circ f(x)=g(f(x))\). Diagramme Siehe V2\(_{\text{1}}\).
\end{definition}


Die Verkn"upfung hat folgende Eigenschaften:
\begin{relation}
\begin{enumerate}
\item Sie ist Assoziativ: \(h\circ (g\circ f) = (h \circ g) \circ f\) f"ur alle Abb. \(f: X\to Y, g:Y\to Z\), \(h:Z\to V\)
\item F"ur jede abbildung \(f: X\to Y\) gilt: \(f\circ id_X=id_Y\circ f = f\).
\item Wenn \(f:X\to Y\) bijektiv ist, dann gilt: \(f\circ f^{-1}=id_Y\):
\begin{itemize}
\item \(f^{-1}\circ f=id_X\) weil: \(f(f^{-1}(y))=y\):
\item \(f^{-1}(f(x))=x'\) mit \(f(x')=f(x)\implies x=x'\) wenn \emph{Bijektiv}
\end{itemize}
\end{enumerate}
\end{relation}

\subsubsection{Kommutative Diagramme}
\label{sec:org429b8d6}
Siehe V2\(_{\text{2}}\): 
\begin{enumerate}
\item Dieses Diagramm heist kommutativ, wenn \(h=g\circ f\).
\item kommutativ wenn \(g\circ f=h\circ k\)
\end{enumerate}

\subsubsection{Eingeschr"ankte Abbildungen}
\label{sec:org60b2559}
\begin{definition}{Einschr"ankung}{}
Sei \(f: X\to Y\) eine Abbildung.\\
Die Einschr"ankung von \(f\) auf eine Teilmenge \(A\subseteq X\) ist die  Abbildung:
\(f|_A:\begin{matrix}A\to Y\\ a\mapsto f(a)\end{matrix}\)
\end{definition}

\begin{exa}
\(f: \mathbb{R}\to \mathbb{R}, t\mapsto t^{2}\) ist nicht injektiv, \(f|_{[0,
\infty)}\) ist injektiv.
\end{exa}

\subsubsection{Quantoren}
\label{sec:orgd2b2557}
\begin{definition}{Quantoren}{}
\begin{itemize}
\item f"ur alle \(x\) in \(X\) - \(\forall x \in X\)
\item es existiert \(x \in X\) - \(\exists x \in X\)
\end{itemize}
\end{definition}

\begin{exa}
\(f:X\to Y\) ist surjektiv, wenn \(\forall y \in Y \exists x\in X\) mit \(f(x)=y\).
\end{exa}

F"ur die Negation der Quantoren gilt:
\begin{relation}
\begin{itemize}
\item \(\neg(\forall x\in X : A(x)) = \exists x\in X : \neq A(x)\)
\item \(\neg(\exists x\in X : A(x)) = \forall x\in X : \neq A(x)\)
\end{itemize}
\end{relation}

\begin{exa}[] \label{} \
\(f: X\to Y\) ist surjektiv \(\iff \forall y\in Y \exists x\in X : f(x)=y\).\\
Also: \(f: X\to Y\) ist \textbf{nicht} surjektiv \(\iff \exists y\in Y \forall x\in X : f(x)\not=y\).
\end{exa}

\subsection{Schlagworte}
\label{sec:org0cab6a2}
\begin{itemize}
\item Venn Diagram - Kreise und Schnittmengen
\item Zeigen von "Aquivalenz zweier Zusammengeseten Mengen:
\begin{itemize}
\item Wahrheitstafel
\item Zur"uckf"uhren auf Aussagenlogik
\end{itemize}
\item Zeigen das \(p,q,r\) "aquivalent sind:
\begin{itemize}
\item \(p\implies q \implies r \implies q\)
\end{itemize}
\item \emph{Injektivit"at} zeigen:
\begin{itemize}
\item nicht I. wenn Gegenbeispiel existiert
\item Zeigen das Funktion streng monoton steigt.
\end{itemize}
\item \emph{Surjektivit"at} zeigen:
\begin{itemize}
\item nicht S. wenn Gegenbeispiel existiert
\item Zeigen das Funktion streng monoton steigt und gegen \(+-\infty\) strebt.
\end{itemize}
\item \(A\setminus (A\setminus B) = A \cap B\)
\item Beweise mit Abbildungen \(M\) sei Menge, \(f\) sei Abbildung:
\begin{itemize}
\item \(y \in f(M) \implies \exists x \in M : f(x)=y\)
\end{itemize}
\end{itemize}

\section{Logik und Beweisf"uhrung}
\label{sec:orgc083250}
Mathematik operiert mit \textbf{Aussagen}.

\begin{definition}{Aussage}{}
Eine Aussage ist eine Behauptung, die Wahr oder Falsch sein kann.
\end{definition}

\begin{notation}\
\begin{description}
\item[{1}] wahr
\item[{0}] falsche
\end{description}
\end{notation}

\(A,B\) seien Aussagen, dann kann man folgende Aussagen betrachten:
\begin{relation}
\begin{itemize}
\item ''nicht \(A\)'': \(\neg A\)
\end{itemize}
\begin{center}
\begin{tabular}{lrr}
\(A\) & 0 & 1\\
\hline
\(\neg A\) & 1 & 0\\
\end{tabular}
\end{center}

\begin{itemize}
\item Vernk"upfungen
\end{itemize}
\begin{center}
\begin{tabular}{rrrrrr}
\(A\) & \(B\) & \(\neg A\) & \(A\wedge B\) & \(A \vee B\) & \(A\implies B\)\\
\hline
0 & 0 & 1 & 0 & 0 & 1\\
0 & 1 & 1 & 0 & 1 & 1\\
1 & 0 & 0 & 0 & 1 & 0\\
1 & 1 & 0 & 1 & 1 & 1\\
\end{tabular}
\end{center}

\begin{itemize}
\item ''A "aquivalent zu B'':  \(A\iff B\)
\end{itemize}
\begin{center}
\begin{tabular}{rrr}
\(A\) & \(B\) & \(\iff A\)\\
\hline
0 & 0 & 1\\
0 & 1 & 0\\
1 & 0 & 0\\
1 & 1 & 1\\
\end{tabular}
\end{center}
\end{relation}

\begin{exa}
F"ur ein Element \(x\in X\) k"onnen wir Aussagen betrachten:
\begin{enumerate}
\item \(A(x)=x\in A\)
\item \(B(x)=x\in B\)
\end{enumerate}
\(A(x)\wedge B(x)=x\in (A\cap B)\)
\end{exa}

\subsection{Identit"aten der Aussagenlogik}
\label{sec:orgd743b6e}
\begin{relation}
\begin{enumerate}
\item Direkter Beweis 
\begin{itemize}
\item \((A\implies B) = (\neg A)\vee B\)
\item Vorraussetzung \(\rightarrow\) logische Aussage \(\rightarrow\) Behauptung
\end{itemize}
\item Beweis in Schritten
\begin{itemize}
\item \(((A\implies B)\wedge (B\implies C))\implies (A\implies C)\) \\
\(\rightarrow\) Konstant \(=1\) (\emph{Tautologie})
\end{itemize}
\item Beweis durch Kontraposition
\begin{itemize}
\item \((A\implies B) \iff (\neg B \implies \neg A)\) - \emph{Tautologie}
\end{itemize}
\end{enumerate}
\end{relation}

\subsection{Widerspruchsbeweis}
\label{sec:org54c9d02}
Wenn wir die Konsequenz aus der Negation der zu beweisenden Aussage und die
Pr"amisse zu einem widerspruch f"uhren so ist die Aussage bewiesen, denn:
\begin{relation}
\[(A\wedge \neg A)=0\]
\end{relation}


Wir wollen \(A\implies B\) zeigen.
Nehmen an \(\neg B\) und leiten her:\\
\begin{relation}
\((\neg B \wedge A)\implies 0\), also \(\neg B\wedge A = 0\), und daher \(A\implies
B\).
\end{relation}

\begin{theo}{Satz von Euklid}{}
Es gibt unendlich viele Primzahlen.
\end{theo}

\begin{proof}\
\begin{enumerate}
\item Nehmen wie an, es gibt nur endlich viele Primzahlen. \(p_1, ..., p_n\).
\item Betrachte \(n=p_1\cdot p_2\cdot ... \cdot p_n + 1\). \(n\) geteilt durch jede
von den Primzahlen \(p_1, ..., p_n\) gibt Rest \(1\).
\item Also ist \(n\) eine Primzahl, aber \(n\not=p_1 ... p_n\) weil gr"osser.
\item Folglich enth"alt die Menge \({p_1,...,p_n}\) nicht alle Primzahlen.
\end{enumerate}
\indent\indent \(\rightarrow\) Das ist ein \textbf{Widerspruch}. (\((A\wedge \neg A) = 0\))
\end{proof}


\begin{exa}
Wir werden die Aussage: wenn \(q\) eine gerade Primzahl ist \(\implies q=2\)
beweisen.

\begin{proof}[Direkter Beweis] \label{} \
\begin{enumerate}
\item \(q\) ist gerade \(\implies q\) ist durch \(2\) Teilbar f"ur \(k\in\mathbb{N}\)
\item \(q\) ist aber eine Primzahl \(\implies\) einer der Faktoren in \(2\cdot k\) ist
gerade \(1\), \(2\not= 1\)
\item \(\implies k=1, q=2\)
\end{enumerate}
\end{proof}

\begin{proof}[Kontraposition] \label{} \
Wir m"ussen zeigen: \(q\not= 2\implies\) (\(q\) ungerade) \(\vee\) (\(q\) keine
Primzahl). Es reicht zu zeigen: (\(q\not=2)\wedge(q\) ist eine Primzahl)
\(\implies q\) ist ungerade!
\begin{enumerate}
\item Wenn \(q\) gerade ist, \(q\cdot 2k\), also ist \(k>1\)
\item also \(q\not= 2\)
\end{enumerate}
\end{proof}

\begin{proof}[Widerspruchsbeweis] \label{} \
Annahme: \(q\) ist gerade, \(q\) ist eine Primzahl, \(q\not= 2\). Wir wollen einen
Widerspruch herleiten.

\begin{enumerate}
\item da \(q\) gerade ist, gilt \(q=2\cdot k\) f"ur ein \(k\in \mathbb{N}\)
\item da \(q\not= 2\), gilt \(k>1\)
\item aber \(q\) ist prim, also kann \(q\) kein Produkt von zwei Zahlen sein! \(\lightning\)
\end{enumerate}
\end{proof}
\end{exa}

\section{Komplexe Zahlen}
\label{sec:org73b0a26}
Idee: Man m"ochte Quadratische Gleichungen ohne reelle Nullstellen trotzdem
l"osen, also erweitert man die reellen Zahlen.

\begin{relation}
Die pototypische Quadratische Gleichungen ohne reelle L"osungen ist: \(x^2+1 =
-1\).\\
Man f"ugt K"unstlich die Zahl \(i\) hinzu mit \(i^2=-1\), m"oglichst unter
Beibehaltung der "ublichen Rechenregeln: man braucht also die Zahlen \(b\cdot i :
b\in \mathbb{R}\) und \(a+b\cdot i :  a,b\in \mathbb{R}\).
\end{relation}

Was passiert, wenn man solche Zahlen miteinander multipliziert ''als ob'' sie
normale Zahlen w"aren: 
\begin{relation}
\((a+bi)\cdot(c+di)=ac+bc\cdot i+ad\cdot i+(-bd)=(ac-bd)+(bc+ad)\cdot i\) f"ur \(a,b,c,d\in \mathbb{R}\)
\end{relation}

Addieren kann man solche Ausdr"ucke auch:
\begin{relation}
\((a+bi)+(c+di)=(a+c)+(b+d)\cdot i\)
\end{relation}

\begin{definition}{Komplexe Zahlen}{}
Die komplexen Zahlen \(\mathbb{C}\) sind die Menge der Paare \((a,b)\in
\mathbb{R}^2\) versehen mit der Addition \((a,b)+(c,d)=(a+c,b+d)\) und der
Multiplikation \((a,b)\cdot (c,d)=(ac-bd, bc+ad)\).
\end{definition}

\begin{notation}[] \label{} \
\begin{itemize}
\item Statt \((a,b)\) schreibt man auch \((a+bi)\in \mathbb{C}\).
\item \(i:=(0,1)=0+1\cdot i\):
\begin{itemize}
\item nach Multiplikation erf"ullt \(i^2=-1\)
\end{itemize}
\end{itemize}
\end{notation}

Man "uberpr"uft, dass die "ublichen Rechenregeln aus \(\mathbb{R}\) weiterhin
gelten (\emph{K"orperaxiome}): F"ur \(z_1, z_2, z_3 \in \mathbb{C}\) gilt, z.B.:
\begin{relation}
\begin{itemize}
\item \(z_1\cdot (z_2+z_3)=z_1z_2+z_1z_3\)
\item \(z_1\cdot (z_2\cdot z_3)=(z_1\cdot z_2)\cdot z_3\)
\item \(z_1 + (z_2 + z_3)=(z_1 + z_2) + z_3\)
\end{itemize}
\end{relation}

\begin{notte}[] \label{}
\((\mathbb{R},+,\cdot)\subsetneq (\mathbb{C},+,\cdot)\) auf nat"urliche Weise als
der der Form \(a+0\cdot i = (a,0)\), \(a\in \mathbb{R}\).
\end{notte}

\begin{definition}{Real- und Imagin"aranteil}{}
F"ur \(z=a+b\cdot i\in \mathbb{C}\) heisst:
\begin{itemize}
\item \(a:=:Re(z)\) Realanteil von \(z\)
\item \(b:=:Im(z)\) Imagin"aranteil von \(z\)
\end{itemize}

Also ist \(z=Re(z)+ Im(z)\cdot i\).
\end{definition}

\begin{definition}{Rein Imagin"are Zahlen}{}
Die Zahlen der Form \(b\cdot i : b\in \mathbb{R}\) heissen \textbf{rein Imagin"ar}.
\end{definition}

F"ur reele Zahlen wissen wir: \(\forall a\in \mathbb{R}\) mit \$a\textlnot{}= 0 \(\exists\)
\(a^{-1} \in \mathbb{R}\) mit \(a*a^{-1}=1\). Gilt das auch in \(\mathbb{C}\) ?

\begin{definition}{Komplexe Konjugation}{}
F"ur \(z\in \mathbb{C}\) heisst die Zahl \(\overline{z}:=a-bi\) die komplex
konjugierte Zahl zu \(a+bi\).
\end{definition}

\begin{exa}[] \label{}
\(\overline{1+i}=1-i\)
\end{exa}

\begin{relation}
\(z*\overline{z}=(a+bi)(a-bi)=a^2+b^2\geq -\) mit Gleichheit genau dann, wenn \(z=0\).
\end{relation}

\begin{definition}{Betrag der Komplexen Zahl}{}
\(|z|:=\sqrt{x\cdot \overline{z}}=\sqrt{a^2+b^2}\) mit \(z=a+bi\).
\end{definition}

\subsection{Inverses zu einer komplexen Zahl}
\label{sec:org0018acd}
Das Inverse zu \(z\not= 0\):
\begin{relation}
\(z\cdot \frac{\overline{z}}{|z|^2}=1\) \\
Also: \(\forall z\not= 0 \exists z^{-1} = \frac{\overline{z}}{|z|^2}\) mit \(z \cdot z^{-1}=1\)
\end{relation}

\begin{exa}[] \label{} \
\((1+i)^{-1}=\frac{1-i}{2}\)
\end{exa}

Mnemonische Rechenregel, Multipliziere mit dem Inversen: 
\begin{relation}
\(\frac{1}{1+i}=\frac{1-i}{(1+i)(1-i)}=\frac{1-i}{2}\)
\end{relation}

\subsection{Geometrische Interpretation von \(\mathbb{C}\)}
\label{sec:org992fe0c}
Siehe Zeichung \(C_1\).

\begin{relation}
\begin{itemize}
\item Addition: als Addition von Vektoren
\item Betrag: L"ange des Vektors
\item \(\varphi\) - Winkel zwischen der reellen Achse und dem Vektor der \(z\) entspricht,
gez"ahlt gegen den Urzeigersinn.
\end{itemize}
\end{relation}

Es folgt:
\begin{relation}
\(a=|z|\cdot \cos(\varphi)\) und \(b=|z|\cdot \sin(\varphi)\)
\end{relation}

\begin{notte}[] \label{}
\(\varphi\) ist nicht eindeutig bestimmt, sondern bis auf Addition von eines
vielfachen von \(2\pi\).
\end{notte}

\begin{exa}[] \label{}
\(\varphi=\frac{\pi}{4}\) und \(\varphi=-\frac{7\pi}{4}\) sind im geometrischen Bild von
\(\mathbb{C}\) "aquivalent.
\end{exa}

\begin{definition}{}{}
Der wert von \(\varphi\), welcher in \([0, 2\pi)\) liegt, heisst Hauptargument von \(z\),
\(arg(z)=\varphi\).\\
Das Argument von \(z\) ist die Menge von allen \(\varphi \in R\),4
\(z=|z|(\cos(\varphi)+i\cdot \sin(\varphi))\), \(Arg\, z = {\varphi \in R : |z|(\cos(\varphi)+i\cdot \sin(\varphi))}\).
\end{definition}

\begin{notte}[] \label{}
\(Arg\, z= {arg(z)+2\pi\cdot k : k\in \mathbb{Z}}\)
\end{notte}

\begin{exa}[] \label{} \
Seien \(z_1=|z_1|\cdot \cos(\varphi_1)+i\cdot \sin(\varphi_1)\), \(z_2=|z_2|\cdot
\cos(\varphi_2)+i\cdot \sin(\varphi_2)\) zwei komplexe Zahlen.\\

So gilt: \(z_1\cdot z_2 = |z_1|\cdot |z_2|(\cos(\varphi_1+\varphi_2)+i\cdot \sin(\varphi_1 +
\varphi_2))\)
\end{exa}

\begin{relation}
Bei der Multiplikation von komplexen Zahlen multiplizieren sich die Betr"age,
und die Argumente addieren sich.
\end{relation}

F"ur geometrische Interpretation: Siehe \(C_2\).

Besonders n"utzlich ist dies f"ur die Multiplikation einer komplexen Zahl vom
Betrag \(1\):
\begin{align*}
|z|=1\iff z=\cos(\varphi)+i\cdot \sin(\varphi) \text{f"ur ein} \varphi \in \mathbb{R}
\end{align*}

\begin{relation}
Es liegen \(\{z\in \mathbb{C} : |z|=1\}\) auf dem Einheitskreis.
Die Multiplikation mit von komplexen Zahlen Zahlen mit dem Betrag 1 entspricht
also der Rotation gegen den Urzeigersinn um \(\varphi\).
\end{relation}

\subsection{Exponentialform der komplexen Zahlen}
\label{sec:orgb4d9f14}
\begin{notation}[] \label{} \
\begin{itemize}
\item Exponentialform: \(\cos(\varphi)+i\cdot \sin(\varphi):=e^{i\cdot \varphi}\)
\item es gilt \(e^{i(\varphi_k)}, k\in\mathbb{R}\) sind die Zahlen auf dem Einheitskreis
\end{itemize}
\end{notation}

\begin{definition}{Exponentialform der komplexen Zahlen}{}
Die Exponentialform f"ur jede komplexe Zahl \(z\in\mathbb{C}\) lautet \(z=|z|e^{i\cdot arg\,z}\).
\end{definition}

Mit dieser Notation folgt:
\begin{relation}
\((e^{i\varphi})^n=(\cos(\varphi)+i\cdot\sin(\varphi))^2=e^{n\cdot i\cdot
 \varphi}=\cos(n\varphi)+i\cdot\sin(n\varphi)\) f"ur alle \(n\in\mathbb{N}\)
\end{relation}

\begin{exa}[] \label{}\
\begin{equation*}
%\begin{split}
(\cos(\varphi)+i\cdot \sin(\varphi))^2 =\cos^2(\varphi)-\sin^2(\varphi)+2\cdot\sin(\varphi)\cdot\cos(\varphi) 
 = \cos(2\varphi) + 2\sin(2\varphi) 
 \implies 
\begin{cases}
\cos(2\varphi)=\cos^2(\varphi)-\sin^2(\varphi) \\
\sin(2\varphi)=2\cdot\sin(\varphi)\cdot\cos(\varphi)
\end{cases}
%\end{split}
\end{equation*}
\end{exa}

\subsection{Einscheitswurzeln}
\label{sec:org0755105}
Sei die gleichung \(x^n=a\) "uber \(\mathbb{R}\) gegeben. Je nach Vorzeichen von
\(a\) und Parit"at von \(n\), gibt es Varianten f"ur die Anzahl der L"osungen.
\begin{relation}
In \(\mathbb{C}\) hat aber die Gleichung \(z^n=a\) f"ur ein \(a\in
\mathbb{C}\setminus \{0\}\) immer genau \(n\) L"osungen.
\end{relation}

Sei \(w\in \mathbb{C}\) mit \(w^n=a\). Dann gilt \((\frac{z}{w})^n=1\) f"ur jedes
\(z\in \mathbb{C}\) mit \(z^n=a\). \textbf{Also} l"osen wir erst mal die Gleichung \(z^n=1\),
und dann reduzieren wir den allgemeinen Fall darauf.

\begin{definition}{Einheitswurzel}{}
Eine Zahl \(z\in \mathbb{C}\) heisst \(n\text{-te}\) Einheitswurzel, wenn \(z^n=1\).
\end{definition}

\begin{proposition}[] \label{}
F"ur jedes \(n\geq, n\in\mathbb{N}\) existieren genau \(n\)
Einheitswurzeln in \(\mathbb{C}\). Sie sind durch die Formel
\(z_k=e^{\frac{2\pi\cdot k\cdot i}{n}},\quad k=0,1,...,n-1\) gegeben.
\end{proposition}

\begin{proof}[] \label{} \
\(z_k\) sind \(n\text{-te}\) Einheitswurzeln denn:
\begin{align*}
z_k^n & = (e^{\frac{2\cdot\pi\cdot k}{n}})^n \\ 
& = e^{2\pi\cdot k} \\
& = 1
\end{align*}


Wir m"ussen noch zeigen, dass jede \(n\text{-te}\) Einheitswurzel von dieser Form
ist. \\

Sei \(z\in\mathbb{C}\) mit \(z^n=1\). Es gilt:

\begin{align*}
|z|^n & =|z^n|=1 \\
& \implies |z|=1  \\
& \implies z=e^{i\cdot\varphi} \tag*{f"ur ein $\varphi\in[0, 2\pi)$}  \\ 
& \implies 1 = z^n \\
& = (e^{i\varphi})^n=e^{i\varphi\cdot n} \\
& =\cos(n\varphi)+i\cdot \sin(n\varphi)
\end{align*}

Also folgt:
\begin{gather*}
\cos(n\varphi)=1,\;\sin(n\varphi)=0 \\
\implies  n\cdot\varphi = 2\pi\cdot k \tag*{f"ur ein $k\in \mathbb{Z}$} \\ 
 \implies \varphi = \frac{2\pi\cdot k}{n} \tag*{f"ur ein $k\in \mathbb{Z}$}
\end{gather*}


Da \(\varphi\) in \([0,2\pi)\implies 0\leq k < n\).
\end{proof}

Wenn wir jetzt also eine Gleichung \(z^n=a\) l"osen wollen, reicht es, eine
L"osung \(w\) zu finden, die anderen L"osungen bekommt man als \(w\cdot z_k,\;
k=0,...,n-1\) mit \(z_k\), der \(n\text{-ten}\) Einheitswurzeln: \(z^n=a\iff
(\frac{z}{w})^n=1\).\\

Eine L"osung \(w\) kann man folgendermassen finden:
\begin{relation}


\begin{align*} 
\text{Schreiben wir a}\; & =|a|\cdot e^{i\cdot \psi}\; \text{f"ur ein $\psi\in \mathbb{R}$} \\
\text{Dann gilt: }
w & =\sqrt[n]{|a|}\cdot e^{\frac{i\cdot\psi}{n}} \text{ l"ost $w^n=a$} \\
& \\
\left(\sqrt[n]{|a|}\cdot e^{\frac{i\cdot\psi}{n}}\right)^n & = \sqrt[n]{|a|}\cdot e^{\frac{i\cdot\psi}{n}\cdot n} \\
& = |a|\cdot e^{i\cdot \psi} \\ 
& = a
\end{align*}
\end{relation}

Gemetrische Interpretation: regul"ares \(n\text{-Eck}\).

\newpage

\section{Lineare Gleichungsysteme}
\label{sec:org4c2c71d}
Wir werden die Bezeichung \(K\) f"ur \(\mathbb{R}\) oder \(\mathbb{C}\) verwenden.

\begin{definition}{Lineare Gleichung}{}
Eine Lineare Gleichung "uber \(K\) ist eine Gleichung der Form
\(a_1x_1+a_2x_2+...+a_nx_n=b\).\\
Hierbei sind \(x_1,...,x_n\) die Variablen und \(a_1,...,a_n,b \in K\), die Koeffizienten.
\end{definition}

\begin{definition}{Lineares Gleichunssystem}{}
Ein Lineares Gleichungsystem ist eine endliche Menge von Gleichungen:
\[{\displaystyle {\begin{matrix}a_{11}x_{1}+a_{12}x_{2}\,+&\cdots
&+\,a_{1n}x_{n}&=&b_{1}\\a_{21}x_{1}+a_{22}x_{2}\,+&\cdots
&+\,a_{2n}x_{n}&=&b_{2}\\&&&\vdots &\\a_{m1}x_{1}+a_{m2}x_{2}\,+&\cdots
&+\,a_{mn}x_{n}&=&b_{m}\\\end{matrix}}}\]
\end{definition}

Ein L"osung von diesem Gleichungssystem ist ein \[n\text{-Tupel }
\left( \begin{matrix} x_{1}\\ \vdots\\ x_{n}\end{matrix} \right) \in K^{n} \]
dass jede Gleichung erf"ullt. Ein lineares Gleichungssystem (LGS) zu l"osen,
heisst, alle L"osungen zu finden.

\begin{relation}
\textbf{Idee}: Man formt das LGS durch Operationen um, die die Menge der L"osungen nicht
ver"andern. Solche Operationen heissen "Aquivalenzumformungen. Diese sind unter
anderem: 
\begin{enumerate}
\item Multiplikation einer Gleichung mit einer zahl \(\alpha\in K\setminus \{0\}\)
\item Addierung von einer Gleichung zu der anderen (z.B. Ersetzen der zweiten
Gleichung durch die Summe der ersten und zweiten.)
\item Vertauschen von zwei Gleichungen; dies kann man auf Operationen von Typ eins
und Zwei zur"ukf"uhren
\end{enumerate}
\end{relation}

Wir werden ein LGS umformen, um es auf eine Form zu bringen, wo die L"osung
offensichtlich ist.

Wir beobachten:
\begin{relation}
Es ist "uberflu"ssig, die Variablen mitzuschleppen. Man k"onnte statdessen die
''Tabellen'' von Koeffizienten umformen.
\end{relation}

\begin{definition}{}{}
Eine \(M\times N\) Matrix \(A\) ist eine Tabelle der Gr"osse \(m\times n\), gef"ullt
mit Elementen aus \(K\).  
\[A=(a_{ij})_{\substack{i=1,\cdots,m \\ j=1,\cdots,n}}\]
\end{definition}

\begin{exa}[] \label{}
\[
  A=\left( \begin{matrix} 1& 1\\ 2& -3\end{matrix} \right)
\]

Wobei \(a_{11} = 1\), \(a_{21} = 2\), \(a_{12}=1\) und \(a_{22}=-3\).
\end{exa}

\begin{relation}
Gegeben ein LGS (\(*\)), k"onnen wir eine Matrix \[ A=\left( \begin{matrix}
a_{11}& \cdots & a_{1n}\\ \vdots & & \vdots \\ a_{n1}& \ldots &
a_{nn}\end{matrix} \right) \] aufstellen. Sie heisst Koeffizientenmatrix des
LGS. Auch stellen wir \[b=\left( \begin{matrix} b_{1}\\ \vdots
\\ b_{n}\end{matrix} \right)\]
eine \(m\times 1\) Matrix (Spalte) auf. (Sie
heisst rechter Teil des LGS). Die Matrix \(A'=(A\mid b)\) heisst erweiterte
Koeffizientenmatrix des LGS (\(*\)).
\end{relation}


\begin{definition}{Elementare Zeilenumforumungen}{}
Die "Aquivalenzumformungen des LGS, die wir vorhin betrachtet haben, entsprechen
dann folgenden Umformungen von der erweiterten Koeffizientenmatrix: 
\begin{itemize}
 \item[1'.] Multiplikation einer Zeile mit $\alpha \in K$
 \item[2'.] Addieren von einer Zeile zu der anderen.
\end{itemize}
Wir werden dann versuchen, die (erweiterten koeffzienten-) Matrizen durch diese
Umformungen auf eine Form zu bringen, in der man die L"osung leicht ablesen
kann.

\(1'\) und \(2'\) heissen elementare Zeilenumforumungen.
\end{definition}

Weitere Zeilenumformungen, die man aus diesen erhalten kann:
\begin{relation}
\begin{itemize}
\item Vertauschen Zweier Zeilen
\item Addieren einer Zeile, Multipliziert mit \(\alpha \not= 0\)
\end{itemize}
\end{relation}

Ziel ist eine gegebe erweiterte Koeffizientenmatrix \((A\mid b)\), durch
Zeilenumformungen zu einer Matrix umzuformen, aus der man die L"osung leicht
ablesen kann.

\begin{definition}{Pivotelement}{}
Gegeben einer Zeile \(Z=(a_1,...,a_n)\in K^n\), nennen wir das erste Element
\(a\not= 0\) das Pivotelement.
Wenn \(Z=(0,...,0)\) ist dann gibt es kein Pivotelement.
\end{definition}

\begin{definition}{Zeilenstufenform}{}
Eine Matrix \(A\) hat Zeilenstufenform, wenn folgendes gilt:
\begin{enumerate}
\item Die Nummern von Pivotlementen der Zeilen von \(A\) bilden eine aufsteigende
Folge.
\item Die Nullzeilen, falls existent, stehen am Ende.
\end{enumerate}
\end{definition}

\begin{exa}[] \label{} \
\[
\begin{pmatrix}
 0 & a_{12} & a_{13} \\
 0 & 0 & a_{23} \\
 0 & 0 & 0 \\
\end{pmatrix}
\]
\end{exa}

\begin{theo}{Gauss}{}
Jede Matrix kann durch elementare Zeilenumformungen auf die Stufenform gebracht
werden.
\end{theo}

\begin{proof}[] \label{}
Sei \(A=\begin{matrix}a_{11}&...&a_{nn}\end{matrix}\). \\
Wenn \(A=0\) - Bewiesen. \\
Wenn \(A\not=0\), dann gibt es eine Spalte \(\not= 0\). Sei
\(j_1\) die Nummer dieser Spalte. Durch vertausche von Zeilen erreichen wir
zun"achst \(a_{1j_1}\not= 0\). Multiplaktion der ersten Zeule mit
\(\frac{1}{j_1}\). Jetzt Subtrahiere von jeder Zeile ab der Zweiten die erste
Zeile multipliziert mit \(a_{kj_1}\) (\(k=\) Nummer der Zeile). \\

Wir erhalten dann Restmatrix \(A_1<A\) und wir wenden das selbe Verfahren auf
\(A_1\) an. Da \(A_1\) weniger Zeilen hat, stoppt der ganze Prozess.

\begin{notte}
Nach diesem Verfahren gilt sogar: Pivotelemente sind alle \$=1\$1
\end{notte}
\end{proof}


\begin{exa}
\begin{align*}
  & \begin{gmatrix}[p]
      1 & 2 \\
      3 & 4
      \rowops
      \add[-3]{0}{1}
    \end{gmatrix} \\
  \Rightarrow & \begin{gmatrix}[p]
      1 & 2 \\
      0 & -6
      \rowops
      \mult{1}{\scriptstyle\cdot-\frac{1}{6}}
    \end{gmatrix} \\
  \Rightarrow & \begin{gmatrix}[p]
      1 & 2 \\
      0 & 1
      \rowops
      \add[-2]{1}{0}
    \end{gmatrix} \\
  \Rightarrow & \begin{gmatrix}[p]
      1 & 0 \\
      0 & 1
    \end{gmatrix}
\end{align*}
\end{exa}

\begin{definition}{Reduzierte Zeilenstufenform}{}
Nachdem wir die Zeilenstufenform mit Pivotelementen \(=1\) erreicht haben, k"onnen
wir durch weitere Zeilenumformungen die eintr"age zu Null f"uhren, die oberhalb
von Pivotelementen stehen; Die Finalform heisst dann \textbf{reduzierte
Zeilenstufenform}.
\end{definition}

Das entsprechende Verfahren zum L"osen von LGS sieht so aus:
\begin{relation}
\begin{enumerate}
\item Bringe die erweiterte Koeffizientenmatrix auf die reduzierte
Zeilenstufenform: \\
Die Spalten mit den Pivotelementen in dieser reduzierten Zeilenstufenform
nennen wir Basispalten.
\item Zwei F"alle:
\begin{enumerate}
\item Letzte Spalte des ist eine Basispalte - in diesem Fall hat das LGS keine
L"osungen, da eine Gleichung \(0=1\) entsteht.
\item Die letzte Spalte ist keine Basisspalte: \\
Das LGS in der reduzierten Zeilenstufenform dr"uckt die Variablen, die zu
Basisspalten geh"oren , durch die restlichen (freien) Variablen und den
rechten Teil des LGS aus. Alle L"osungen werden dadurch erhalten, dass
man f"ur die freien Variablen beliebige Werte in \(K\) ausw"ahlt. Die
Basisvariablen werden dann durch Freie Variablen ausgedr"uckt.
\end{enumerate}
\end{enumerate}
\end{relation}

In unserem Beispiel l"asst sich die L"osung so aufschreiben: 


Errinnerung: ein LGS hatte die erweiterte Koeffizientenmatrix \((A|b)\). Das LGS4
l"asst sich dann auch so aufschreiben:$\backslash$\ \(:=A\cdot x\), wobei \(x=\)


\subsection{Matrizenrechnung}
\label{sec:org0f3e63e}
\begin{definition}{Matrix-Spaltenvektor Produkt}{}
Das Produkt von einer \(m\times n\) Matrix \(A\) und einer Spalte (in dieser
Reihenfolge) wird definiert durch \(A\cdot x =\). In dieser Spalte wird das LGS
\(A\cdot b\).
\end{definition}

Die Menge von Matrizen der Gr"osse \(m\times n\) mit Eintr"agen in \(K\) wird durch
\(M(m\times n, k)\) oder \(K^{m\times n}\) bezeichnet. Matrizen der Gr"osse \(1\times
n\) heissen Spalten der L"ange \(n\). Matrizen der Gr"osse \(n\times 1\) heissen
Zeilen der L"ange \(n\).

\begin{definition}{Addition}{}
Matrizen gleicher Gr"osse kann man eintragsweise Addieren: \$A,B \(\in\) K\(^{\text{m}\texttimes{}\text{
n}}\) \(\rightarrow\) (A+B)\(_{\text{ij}}\):=\$A\(_{\text{ij}}\)+B\(_{\text{ij}}\)\$\$
\end{definition}

\begin{definition}{Multiplikation}{}
Matrizen kann man mit Zahlen multiplizieren (indem man jeden eintrag mit dieser
Zahl multipliziert).

\((\lambda \cdot A)_{ij}:=\lambda \cdot A_{ij}\).
\end{definition}

\begin{definition}{Produkt}{}
Wenn die Breite von \(A\) mit der H"ohe von \(B\) "ubereinstimmt, kann man das
Produkt \(A\cdot B\) definieren: \\
\(A\cdot B:=(A\cdot b_1\; \cdots \; A\cdot b_n)\) mit \(B=(b_1\; \cdots\; b_n)\) (Spalten)
mit \(A\cdot B \in K^{p\times n}\)
\end{definition}

\subsection{Eigenschaften der Matrix-Multiplikation}
\label{sec:org8eaaee0}
\begin{notation}[] \label{}
\begin{itemize}
\item wenn \(\alpha_1,...,\alpha_n\in K\) dann notieren wir \(\alpha_1+...+\alpha_n :=
   \sum_{i=1}^{n}{\alpha_i}\)
\item analog \(\alpha_1\cdot ...\cdot\alpha_n := \Pi_{i=1}^{n}{\alpha_i}\)
\end{itemize}
\end{notation}

\begin{relation}
Es gilt dann mit \(A=(a_{ij})_{\substack{i=1,p}}\) : \((A\cdot
x))_i=\sum_{j=1}^{m}{a_{ij}\cdot x_j,\, i=1,p}\) \\

Insbesondere gilt: \((A\cdot b_k)_i\) Aber \((A\cdot b_k)_i = (A\cdot B)_{ik}\) und \((b_k)_j=b_jk\)
\end{relation}

\begin{relation}
Matrixmultiplikation ist \emph{linear}: \(A\cdot (\lambda B_1 + \lambda_2 A B_2)\)
Analog:
\end{relation}

\begin{proof}[] \label{}
Sei \(C=A\cdot (\lambda_1 B_1 + \lambda_2 B_2)\)
\end{proof}

\begin{relation}
Die Matrixmultiplikation ist assoziativ: \(A\cdot (B\cdot C)=(A\cdot B)\cdot C\)
\end{relation}

\begin{proof}[] \label{}

\end{proof}

\begin{definition}{Einheitsmatrix}{}
Die Einheitsmatrix der gr"osse \(r\) ist die Matrix, die auf der Hauptdiagonale
(links-oben nach rechts unten) Einsen und sonnst Nullen hat. Beizeichnung \(E_r\)
oder \(1_r\).
\end{definition}

\begin{definition}{Kronecker-Symbol}{}
Das Kronecker-Symbol ist definiert als: 

Also gilt: \((Er)_{ij}=\delta_{ij}\).
\end{definition}

\begin{theo}{}{} \
F"ir alle \(A\in K^{p\times m}\) gilt:
\begin{itemize}
\item \(E_p\cdot A=A\)
\item \(A\cdot E_m =A\)
\end{itemize}
\end{theo}

\begin{proof}[] \label{}

\end{proof}


\begin{notation}[] \label{Vorsicht!}
Die Matrix Multiplikation ist nicht Konjunktiv: \(A\cdot B\not= B\cdot A\) im
Allgemeinen, selbst wenn beide Produkte definiert sind und die gleiche Gr"osse
haben.
\end{notation}

\begin{exa}[] \label{}

\end{exa}

\begin{notation}[] \label{}
Die \(i\text{te}\) Spalten der Einheitsmatrix wird durch \(e_i=()\) bezeichnet.
\end{notation}

\begin{definition}{Transposition}{}
Sei \(A\in K^{m\times n}\). Die transponierte Matrix \(A^{T}\in K^{n\times n}\) ist
definiert durch \((A^T)_{ij}:=A_{ji}\). Also ist die i-te Zeile der Einheitsmatrix \((e_i)^T\)
\end{definition}

Wie l"ost man nun das LGS \(A\cdot x=b\)? Man bringt die erweiterte
Koeffizientenmatrix in die reduzierten Zeilenstufenform.

\begin{relation}
Die Basisspalten in der reduzierten Zeilenstufenform sind von der Form, wo \(e_i\)
die i-te Spalte der Einheitsmatrix \(1_r\) ist. \(r <= m\).
\end{relation}

Wenn die ersten \(r\) Spalten Basisspalten sind, dann sieht die reduzierte
Zeilenstufenform so aus:


Dann sehen die L"osungen so aus:
\begin{relation}
\begin{enumerate}
\item Es gibt keine \(\iff\) \(b''\not= 0\)
\item Wenn \(b''=0\), dann sehen die L"osungen so aus: \[x=+\]
\end{enumerate}
\end{relation}

Proposition: Sei \(A\in k^{m\times n}\). Das homogene LGS der Form \(L=\{\phi
t \}\) fuer ein \(r\geq 0, \phi\)

\(\rightarrow\) es gibt \(n-r\) freie Parameter, die die Loesungsmenge beschreiben.

\textbf{Anmerkung} Ein homogenes LGS \(A\cdot x=0\) mit  hat immer eine L"osung \(x\not=
0\) (es gibt mindestens eine freie Variable).


\begin{exa}[] \label{}
Finde ein reelles Polynom von Grad 2.

Die Frage ist aequivalent zu dem LGS:
\end{exa}

\begin{definition}{}{}
Die Menge der Polynome vom Grad h"ochstens \(n\) mit Koeffizienten in \(K\) ist
durch \(K[t]_n\) berechnet.
\end{definition}

\section{Vektorra"ume}
\label{sec:org4906e00}
\begin{definition}{Vektorraum}{}
Ein \(k\) - Vektorraum \(V\) ist eine Menge zusammen mit den Operationen und mit
folgenden Eigenschaften:
\begin{itemize}
\item Addition \(+:\, V\times V \to V, (v_1,v_2)\mapsto v_1+v_2\)
\begin{enumerate}
\item kommutativ
\item assoziativ
\item \(\exists 0 \in V\) mit \(0+v=v+0=v\) \(v \in V\)
\end{enumerate}
\item Skalarmultiplikation \(+:\, V\times V \to V, (v_1,v_2)\mapsto v_1+v_2\)
\begin{enumerate}
\item assoziativ
\item distributiv bez. addition
\item \(1\cdot v = v\), \(v\in V\)
\end{enumerate}
\end{itemize}
\end{definition}

\begin{exa}[] \label{}
\begin{enumerate}
\item \(K\) ist selbst ein Vektorraum mit \(+\) und \(\cdot\)
\item \(K^{n}:=K^{n\times 1}\) ist ein K-Vektorraum mit:
\begin{enumerate}
\item Addition
\item Skalarmultiplikation
\end{enumerate}
\item \(K^{m \times n}\), eine Matrix der Gr"osse \(m\times n\) mit Eintr"agen in K,
ist ein K-Vektorraum mit Addition und Skalarmultiplikation von Matrizen:
\item \(K[t]_n\) ist ein K-Vektorraum:
\item \(K[t]:=\{a_n\cdot \}\) - alle Polynome mit Koeffizienten in \(K\) bilden einen
K-Vektorraum mit gleichen Operatoren.
\item Sei \(X\) X eine Menge (z.B. \(X=\mathbb{R}\)) \(Fun(X,K)=\{\}\) ist ein K-Vektorraum:
\begin{enumerate}
\item Addition \((f_1 + f_2)(x):= f_1(x)+ f_2(x)\), \(x\in X\)
\item Miltiplikation
\end{enumerate}
\item Sei \(A\in K^{m\times n}\). Die L"osungen von dem homogenen LGS bilden einen
Vektorraum:
\begin{itemize}
\item Wenn \(x_1,x_2\) L"osungen sind, dann gilt: Also ist die Menge der
L"osungen auch ein Vekorraum bzgl. der Operatoren aus \(K^n\)
\end{itemize}
\end{enumerate}
\end{exa}

\begin{notte}[] \label{}
Bei der Entwicklung der Vektorraumtheorie ist es oft n"utzlich, an das Beispiel
\(V=K^n\) zu denken.
\end{notte}

\subsection{Vektorraumtheorie}
\label{sec:org432e282}
Sei \(V\) ein K-Vektorraum.

\begin{definition}{Linearkombination}{}
Seien \(v_1, v_2\). Die Linearkombination mit Koeffizienten ist der Vektor.
\end{definition}

\begin{definition}{Triviale Linearkombination}{}
Eine Linearkombination heist trivial wenn \(\lambda_1 = \lambda_2 = ... =
\lambda_n = 0\). (\emph{Nichttrivial} wenn mindestens ein \(\lambda_i\not= 0\)).
\end{definition}


\begin{definition}{}{}
Die Menge der Vektoren heist linear Unabh"angig wenn: \$\$ (Nur die Triviale
linearkombination ergibt 0). Andernfalls heist die Menge linear abh"angig.
\end{definition}

\begin{exa}[] \label{}
\(\begin{pmatrix}1\\0\end{pmatrix}\begin{pmatrix}0\\1\end{pmatrix}\)
\end{exa}

\begin{exa}[] \label{}
\(\{v_1,v_2\}\) sind linear abhaengig wenn sie Proportional sind.

\begin{proof}[] \label{}

\end{proof}
\end{exa}

\textbf{Lemma} Die Menge ist linear abh"angig. \(v_i\) ist eine Linearkombination von \$\$

\begin{proof}[] \label{}
Wenn \(v_i=\lambda_1 v_1\), dann \(0=\) Denn \(-1\) ist ein nicht-trivialer
Linearfaktor.

\((\implies)\) Nach Definition gibt es eine nichttriviale Linearkombination: als
\(\exists i : \lambda_i \not= 0\) Also gilt  folglich 
\end{proof}

\begin{notte}[] \label{}
Eine L"osung des LGS \(Ax=b\) ist eine Spalte \$\$ mit 
Deis heisst, das LGS \(Ax=b\) zu l"osen ist genau linearkombinationen von spalten
zu finden, welche \(b\) ergeben.
\end{notte}

\textbf{Lemma} Seien die Vektoren linear unabh"angig. Ein Vektor \(v\) ist genau dann
eine Linearkombination von \(v_1,...,v_n\) wenn linear abh"angig ist.

\begin{proof}[] \label{}
(\(\implies\)) Sei Dann gilt , also ist linear ab"angig.
Sei .. linear abh"angig  Dann 

Es gilt: \(\lambda \not= 0\) (wenn  .. linearunabh"angig.) Also gilt \(v=-\lambda_1\)
\end{proof}

\textbf{Lemma} Sei \(v=\lambda\) eine Linearkombination von \(v_1,...,v_n\). Diese
Darstellung ist eindeutig ganau dann, wenn linear unabh"angig sind.

\begin{proof}[] \label{}
\((\implies)\) Sei die Darstellung eindeutig \$v=..\$ Wenn jetzt \$\$, dan gilt \(v=\)
Eindeutigkeit der Darstellung ergibt:

Seien \(v_1,..,v_n\) linear unabh"angig, sei 
Dann gilt: \(\rightarrow\) lineare Unabhaengigkeit von erzwingt Korollar: Wenn die
Spalten von \(A\) linear unabhaenig sind, hat das LGS \(Ax=b\) h"ochstens eine
L"osung, folglich hat \(Ax=0\) genau eine L"osung x=0.
\end{proof}

\subsection{Geometrische Deutung der linearen Abh"angigket}
\label{sec:org95b9a1d}
\begin{notte}[zu geometrischer Interpretation] \label{}
Wichitige Beispiele von Vektorr"aumen sind: \(V=\mathbb{R}^2\) (Ebene),
\(V=\mathbb{R}^3\) (3D-Raum).
\end{notte}

Seien \(v_1, v_2\) nicht proportional.
In drei Dimensionen:
\begin{relation}
\begin{itemize}
\item wenn \(v_3\) in \(E\) liegt, dann ist \(v_3\) eine Linearkombination \(v_1, v_2\)
\item wenn nicht, dann linear Unabhaenig (sonst in Ebene.)
\end{itemize}
\end{relation}

\subsection{Lineare unabhangigkeit in R"aumen}
\label{sec:orgea5b4b9}
\textbf{Proposition} Seien \(v_1,...,v_n \in \mathbb{V}\) linear unabhaenig, seien \(W_1,
..., W_n \in \mathbb{V}\) so dass jedes \(w_i\) eine Linearkombination von
\(v_1,...,v_n\) ist. Wenn \(m>n\), dann sind \(w_1,...,w_n\) linear abhaengig.

\begin{proof}[] \label{}
Seien 
\begin{align*}
w_1= a_{11} v_1 + a_{12} v_2 + ... + a_{nn} v_n \\
w_1= a_{21} v_1 + a_{22} v_2 + ... + a_{nn} v_n \\
\end{align*}
Wir suchen \(\lambda\) (*). Das ist "aquivalent zu:

Dies ist nach linearer Unabhaenig von \ldots{} "Aquivalent zu:

Das heist (*) ist "aquivalent zu einem homogenen LGS aus \(n\) Gleichungen mit \(m\)
Variablen. \(n<m\), also gibt es eine L"osung $\ldots{}$, die ungleich \(0\) ist. \(\implies\)
sind linear unabh"angig.
\end{proof}

\textbf{Korollar} Je drei Vektoren in \(\mathbb{R}^2\) sind linear unabh"angig, je \(n+1\)
Vektoren in \$\$ sind linear unabh"angig.

\begin{proof}[von Korollar] \label{}
Seien \(e_i\) die Spalten der Einheitsmatrix sind linear unabh"angig. 

Dies zeigt auch, dass jeder Vektor in \(R\) eine Linearkombination von \(e_i\) ist. 
\end{proof}

\begin{definition}{}{}
Sei V ein \(K-\) Vektorraum, \(U\subseteq V\) eine Teilmenge von V. \(U\) heist
untervektorraum wenn:
\begin{enumerate}
\item \(V\not= \varnothing\)
\item 

\item 
\end{enumerate}

In anderen Worten: Eine Teilmenge von \(V\) die selbst ein Vektorraum ist bzgl.
der von \(V\) vererbten Operationen.
\end{definition}

\begin{notte}[] \label{}
(1) und (3) \(\implies\) \(0\in U\)
\end{notte}

\begin{definition}{}{}
Sei \(S \in V\) eine Teilmenge. Der von \(S\) erzeugte Vektorraum (lineare H"ulle
von \(S\)) \(<S>:=\{\}\) (Menge aller Linearkombinationen von Vektoren in \(S\)).

Alternative Notation: \(<s>=\text{span S}\).
\end{definition}

\begin{notte}[] \label{}
\(<s>\) ist der kleinste Untervektorraum in \(V\), der \(S\) enth"alt. 
\(<\varnothing >:=\{0\}\)
\end{notte}

\begin{exa}[] \label{}
Seien \(v_1, v_2 \in \mathbb{R}^3\).
\(<v1,v2>\) ist eine Gerade wenn \(v_1,v_2\) linear abh. Ist Ebene wenn \(v_1,v_2\)
linear unbh.
\end{exa}

\begin{definition}{}{}
\(S\in V\) heisst Erzeugendensystem wenn \(<S>=V\). (S spannt den Vektorraum auf.)

ist ein Erzeugendensystem: jeder Vektor in \(V\) ist eine Linearkombination von:

\(\lambda_i\) sind nicht unbedingt eindeutig bestimmt, weil nicht linear unabh.
vorrausgesetzt waren.
\end{definition}

\begin{definition}{}{}
Ein Erzeugendensystem \(B\in V\) heisst basis, wenn es linear unabh. ist. Nach dem
Lemma ueber Eindeutigkeit der koeffzienten der Linearkombination gilt: \(B=\{v_1,
..., v_n\}\) ist eine Basis genau dann, wenn f"ur jeden Vektor \(v \in V\) gibt es
 eindeutig bestimmte Zahlen.
\end{definition}

\begin{definition}{}{}
Ein Vektorraum \(V\) heisst endlich dimensional, wenn er ein endliches
erzeugendensystem besitzt. (= wird von endlich vielen Vektoren aufgespannt).
\end{definition}

\begin{theo}{}{}
Jeder endlichedimensionale Vektorraum \(V\)  hat eine Basis, Je zwei Basen von \(V\)
haben gleich viele Elemente.
\end{theo}

\begin{proof}[] \label{}
Sei \(S\) ein endliches Erzeugendensyste von \(V\), Wenn \(S\) lin. unabh. ist, ist es
eine Basis und wir haben gewonnen. Wenn \(S\) linear abh"angig ist $\implies$
(Lemma) einer von den Vektoren in \(S\) ist eine Linearkombination von den
anderen. Das entfernen dieses Vektors "andert die Tatsache nicht, das \(S\) den
Vektorraum aufspannt. Jetzt haben wir eine kleinere Menge und fangen von vorne
an. Da \(S\) endlich ist und durch entfernen Vektoren kleiner wird haben wir am
Ende eine Basis.
\(\rightarrow\) Wir haben eine Basis.

Seien \(S, S'\) zwei Basen. Da \(S\) eine Basis ist, ist jedes element von \(S'\) eine
linearkombination in \(S\). Die elemente von \(S\) sind linear unabh. (weil Basis).
Wenn also \(m>n\), dann folgt aus der Proposition, dass \(S'\) linear abh. ist, was
unm"oglich ist, da \(S'\) eine Basis ist. Also \(m \leq n\) und aus Symmetriegr"unden folgt auch \(n \leq m\).
\end{proof}

\begin{definition}{}{}
Sei \(V\) ein endlichdimensionaler Vektorraum. Die Anzahl der Vektoren in einer
(folglich in jeder) Basis von \(V\) heist Dimension von V. \emph{Bezeichung}: \(\dim V\).n
\end{definition}

\begin{exa}[s] \label{}
\(\dim K^{n}=n\) weil \ldots{} eine Basis bilden. 
\end{exa}

\textbf{Frage}: kannn man eine lineare unabh"angige Menge \(S\in V\) zu eine Basis
erweitern?.

\textbf{Proposition} Jede linear unabh"angige Teilmenge \(S\in V\) eines
endlichdimensionalen Vektorraumes \(V\) ist in einer maximalen linear
unabh"angigen Teilmenge enthalten. Eine maximal linear unabh. Teilmenge von \(V\)
st eine Basis von \(V\).

\begin{definition}{Maximal linear unabh"angige Teilmengen}{}
Eine Teilmenge \(S'\in V\) ist maximal linear unabh., wenn aus \(S\)
linear unabh. folgt.
\end{definition}

\begin{proof}[1] \label{}
Sei linear unabh.
Zwei F"alle: entweder ist \(S\) schon maximal (dann sind wir fertig) oder man kann
S erweitern. Wenn wir \(S\) erweitern k"onnen, f"ugen wir neue Vektoren hinzu, bis
wir es nicht mehr tun koennen, ohne lineare unabh. zu verletzen.

Dieser Prozess endet, weil eine linear unabh. Teilmenge h"ochstens von \(V\)
hoechstens \(\dim V\) viele Vektoren enthalten kann. (Prop. "uber lineare unabh.
von vektoren aus linearkombinartionen der Basis.)
\end{proof}

\begin{proof}[2] \label{}
Sei \(S\in V\) maximal linear unabh"angig. Wir habe zu zeigen: \(<S>=V\) (Def. einer
Basis). Wenn $\ldots{}$ d.h. aber, aber \(S\cup {v}\) ist linear unabh. (lemma) $\implies$
\(S\) dann nicht maximal.
\end{proof}

\textbf{Korollar} Man kann jeder lienar unabh. Teilmenge \(S\in V\) zu einer Basis
erweitern.

\begin{notte}
Wenn \(V=K^n, S\in V\) linear unabh. \(\rightarrow\) man kann zur erweiterung passende
Spalten der Einheitsmatrix.
\end{notte}

\begin{notte}[] \label{}
Man kann bei der obrigen Proposition das Wort "endlichdimensional" fallen
lassen, aber es braucht ein bisschen mehr Mengenlehre (Auswahlaxiom).x
\end{notte}


\begin{theo}{}{}
Sei V ein endlichdimensionaler Vektorraum und \(U\in V\) ein Untervektorraum. Dann
gillt: \(\dim U \leq \dim V\).  \(\dim U = \dim V \iff U=V\)
\end{theo}

\begin{proof}[] \label{}
Sei eine maximale linear unabh. Teilmenge in U. (so eine Teilmenge existiert
weil V endlich ist.)
Nach Proposition (2) ist \ldots{} eine Basis in \(U\), also gilt \(\dim U = k\) Erweitere
\ldots{} zu einer Basis in V \ldots{}. 

(2) \ldots{} trivial \ldots{} Sei \ldots{} eine Basis in U. Erweitere sie zu einer Basis in
\(V\). Diese Basis in V muss aber wegen \ldots{} gleich viele Vektoren haben. \ldots{}. ist
eine Basis in \(V\) \ldots{}
\end{proof}


\begin{definition}{}{}
Sei \(V\) ein Vektorraum \$\$ eine Basis in V. Die Zahlen \((\lambda_1,...\lambda_n)\)
heissen Koordinaten bzgl. des Vektors. Die Spalte \ldots{} heisst Koordinatenspalte
dieses Vektors bzgl. \(B\).
Die Definition einer Basis garantiert, dass hierdurch eine Bijektion \ldots{} entsteht.
\end{definition}

\textbf{Warnung} Diese Korrespondenz kommt auf die Wahl der Basis an.

\begin{exa}[] \label{}

\end{exa}

\begin{exa}[] \label{}
Sei \(Ax=0\) ein h. LGS
\end{exa}

\textbf{Aus Uebungen} \ldots{}

\textbf{Lemma} Die Spalten von \(\Phi\) bilden eine Basis in L.
\begin{proof}[] \label{}
Die Spalten von \(\Phi\) sind linear unabhaenig. (sonst abb. nicht injektiv)

Ferner spannen sie das ganze L auf (Surjektiv). 
\end{proof}

\textbf{Frage} Gegeben Basen \ldots{} in \(V\), und einem Vektor \(v\in V\). Wie rechnet man die
Koordinaten bezgl. B in Koordinaten bzgl. B' um.

Sprachweise ist: B ist alte Basis und B' ist neue Basis.
So gilt:
\begin{relation}
\ldots{} Dr"ucken wir Vektoren von B' bzgl. Vektoren von B aus: \(V\) \ldots{}
wir erhalten \(C\) 
(Die Spalten von C sind Koordinaten der "neuen" Basis bzgl der alten Basis. )

Also gilt: \ldots{}
\(\lambda = G\cdot \lambda'\)
\end{relation}

\textbf{Frage} Ist \(D\) in diesem Fall immer invertierbar? (Ja, aber wir brauchen mehr Theorie.)

\subsection{Lineare Abbildungen zwischen Vektorr"aumen}
\label{sec:orgb4c03c4}
\begin{definition}{Lineare Abbildung}{}
Seien \(V, W\) zwei K-Vektorr"aume. Eine Abbildung \(f: V\rightarrow W\) heißt linear, wenn:

\begin{enumerate}
	\item \(\forall v,v' \in V: f(v+v')=f(v)+f(v')\)
	\item \(\forall v\in V, \lambda \in K: f(\lambda v)=\lambda f(v)\)
\end{enumerate}

\end{definition}

\begin{exa}[] \label{}
$f_{A}: K^n\rightarrow K^m, x\mapsto A\cdot x$, für $A \in K^{m\times n}$ ist eine lineare Abbildung.
\end{exa}

\begin{exa}[] \label{}
\(V=\mathbb{R}[x]_5,\;W=\mathbb{R}[x]_4\). $f: V\rightarrow W, p \mapsto \frac{dp}{dx}$ ist eine lineare Abbildung, die jedem Polynom seine Ableitung zuordnet.
\end{exa}

\begin{exa}[] \label{}
$V=W=\{f:\mathbb{R}\rightarrow\mathbb{R}\}$ die Menge aller reellen, reellwertiger Funktionen. Für eine gegebene Abbildung $g:\mathbb{R}\rightarrow\mathbb{R}$ ist die Abbildung $mg: V\rightarrow W, f\mapsto g\cdot f, (g\cdot f)(x)=g(x)\cdot f(x)$ linear.
\end{exa}

\begin{definition}{}{}
Sei $f:V\rightarrow W$eine Lineare Abbildung. Der Kern von \(f\) wird definiert als \[Ker\: f := f^{-1}(0) = \{v\in V|f(v)=0\}\] Der Kern der Abbildung f ist also die Menge aller Elemente in V, die auf das Nullelement von W abgebildet werden.
\end{definition}

\begin{exa}[] \label{}
$f=f_A: K^n\rightarrow K^m \quad x\mapsto A\cdot x$\\
$Ker (f_A) = \{x\in K^n|A\cdot x = 0\}$, also die Lösung des homogenen LGS mit der Koeffizientenmatrix A.
\end{exa}

\textbf{Beobachtung} Der Kern von \(f\) ist ein Untervektorraum von \(V\):
\[v_1 , v_2 \in Ker(f) \Rightarrow f(v_1 + v_2)=f(v_1 )+f(v_2 )=0 \Rightarrow v_1 + v_2 \in Ker(f) \]
\[v \in Ker(f), \lambda \in K \Rightarrow f(\lambda v)=\lambda f(v)=\lambda \cdot 0 = 0 \Rightarrow \lambda v \in Ker(f) \]

\textbf{Beobachtung} Das Bild von f $Im(f) \subseteq W$ ist Untervektorraum, wenn:
\[w_1 =f(v_1 ), w_2 = f(v_2 )\in Im(f) \Rightarrow w_1 +w_2 = f(v_1 )+f(v_2 )= f(v_1 +v_2) \Rightarrow w_1 +w_2 \in Im(f) \]
\[w=f(v)\in Im(f), \lambda \in K \Rightarrow \lambda w = \lambda f(v)=f(\lambda v) \in Im(f) \]
$\Rightarrow$ Das Bild einer linearen Abbildung ist ein Untervektorraum.

\begin{exa}[] \label{}
$Im(f_A )=\{b\in K^m \:|\: \exists x \in K^n: A\cdot x=b \}=\{b\in K^m\:|\:LGS Ax=b\; l"osbar\} $
\end{exa}

\textbf{Beobachtung} Definitionsgemäß ist $f:V\rightarrow W$ genau dann surjektiv, wenn $Im(f) = W$.

\textbf{Proposition} Sei \(f:V\rightarrow W\) eine lineare Abbildung. Es gilt: \(f\) injektiv $\iff$ \(Ker(f)=\{0\}\)

\begin{proof}[] \label{}
f injektiv $\iff v_1 \neq v_2 \in V: f(v_1)\neq f(v_2) \iff f(v_1)-f(v_2)\neq 0 \iff f(v_1 -v_2)\neq 0 \iff v_1,v_2 \in V, v_1-v_2\neq 0: f(v_1 -v_2)\neq 0 \iff v\neq 0: f(v)\neq 0 \iff Ker(f) = \{0\}$
\end{proof}

\begin{definition}{}{}
Eine bijektive Lineare Abbildung \(f:V\rightarrow W\) heißt Vektorraumisomorphismus
zwischen \(V\) und \(W\). \(V\) und \(W\) heißen isomorph, wenn es einen
Vektorraumisomorphismus \(f: V\rightarrow W\) gibt.
\end{definition}

\begin{exa}[] \label{}
Sei \(V\) ein Vektorraum, \(S\subseteq V = \{v_1, v_2, ..., v_n\}\). Die
Aufspannabbildung $\varphi_S: K^n \rightarrow V$ wird definiert als $\varphi_S(\lambda)=\sum_{i=1}^{n}\lambda_i v_i$

\begin{itemize}
	\item \(\varphi_s\) ist linear.
	\item \(\varphi_s\) ist bijektiv \(\iff\) S ist Basis.
\end{itemize}
\end{exa}

\begin{korollar}
	\(S=\{v_1,\ldots,v_n\}\) ist eine Basis von V $\implies \varphi_S: K^n \rightarrow V$ ist Isomorphismus ($K^n\cong V$)
\end{korollar}

\begin{korollar}
	\(\dim V = n \iff V\cong K^n\)
\end{korollar}

\begin{beobachtung}
	Wenn \(f:V\rightarrow W\) Isomorphismus $\implies f^{-1}:W\rightarrow V $ ist auch ein Isomorphismus.
\end{beobachtung}

\subsubsection{Dimensionsformel}
\label{sec:org9a58004}
\begin{theo}{}{}
Sei \(f:V\to W\) eine Lineare Abbildung, sei \(V\) endlich dimensional. Dann gilt: \(\mdim\mKer f + \mdim\mIm f = \mdim V\).
\end{theo}

\begin{lemma} sei \(f\) wie oben. Sei \(U \subseteq \mKer(f)\) ein Untervektorraum mit \(U \cap \mKer(f) = \{0\}\).  Dann ist \(f|_U:U\to f(U)\) ein Isomorphismus.
\end{lemma}

\begin{proof}[Beweis des Lemmas] \label{}
\(f|_U:U\to f(U)\) ist surjektiv nach Konstruktion. Zu zeigen: \(f|_U\) injkektiv \(\iff \mKer f|_U = \{0\}\). Sei \(u\in \mKer f|_U \). Dann gilt \(u\in U \land u \in \mKer(f) \implies u \in U \cap \mKer f \implies u = 0\)
\end{proof}

\begin{proof}[Beweis der Dimensionsformel] \label{}
W"ahle eine Basis \({e_1, ..., e_k}\) in \(\mKer f\) und erg"anze sie zu einer Basis
\({e_1, ..., e_n}\) in \(V\).

Betrachte jetzt \(U:=\langle e_{k+1}, ..., e_n\rangle \subseteq V\) Untervektorraum. Es gilt $U \cap \ker(f) = \{0\}$, weil für $u\in U \cap \ker(f)$ gilt: \(u = \sum\limits_{i=1}^k \lambda_ie_i \), weil \(e_1, \dots e_k\) eine Basis im Kern ist, und \(u = \sum\limits_{i=k+1}^{n} \lambda_i e_i\) weil \(u\in U\). Also: \(\sum\limits_{i=1}^k \lambda_ie_i - \sum\limits_{i=k+1}^{n} \lambda_ie_i = 0 \xRightarrow{e_1,\dots, e_n \text{Basis}} \text{alle }\lambda_i = 0 \implies u =0\).

Das Lemma sagt jetzt: \(f|_U\) ist ein Isomorphismus. Außerdem gilt f"ur \(v =\sum\limits_{i=1}^{n}\lambda_ie_i \in V:\)
\begin{align*}
f(v) = f\left(\underbrace{\sum_{i=1}^{k}\lambda_ie_i}_{\in \mKer(f)}\right) + f\left(\sum_{i=k+1}^{n}\lambda_ie_i\right) = f\left(\underbrace{\sum_{i=k+1}^{n} \lambda_ie_i}_{\in U}\right)
\end{align*}
\(\implies f(V)=f(U)\) also \(f|_U : U\to f(V) = \mIm(f) \) ist ein Isomorphismus \(\implies \mdim \mIm f = \mdim U \)

Nun gilt nach Konstruktion von \(U\): \[\underbrace{\mdim \mKer(f)}_{k} + \underbrace{\mdim U}_{n-k} = \underbrace{\mdim V}_{n} \implies \mdim\mKer f + \mdim\mIm f = \mdim V \]
\end{proof}

\subsubsection{Summe von Untervektorr"aumen}
\label{sec:org83dfe63}
\begin{definition}{}{}
	Sei V ein Vektorraum, $U_1, U_2 \subseteq V$ Untervektorräume.
	\[U_1 + U_2 := \{u_1+u_2 | u_1 \in U_1, u_2 \in U_2\}\]
	heißt Summe von $U_1$ und $U_2$.
	$U_1 + U_2$ ist Untervektorraum.
	
\end{definition}
\begin{definition}{}{}
Die Summe von $U_1$ und $U_2$ heißt direkt, wenn $U_1 \cap U_2 = \{0\}$. Bezeichnung: $U_1 + U_2 = U_1 \oplus U_2$
\end{definition}

\textbf{Bemerkung} \ldots{}

In dieser Bezeichnung haben wir im Beseris der Dimensionsformel haben wir den
Ausgangsraum als eine direkte Summe dargestellt. 
\todo{Das bitte ergänzen, da habe ich nicht gut mitgeschrieben}
\textbf{Bemerkung}: Dimensionsformel ist auch Rangformel.

\begin{definition}{Rang}{}
Sei $f:V\to W$ linear. Der Rang von \(f\) ist $\mRg(f) = \text{rk}(f) := \dim\mIm(f)$
\end{definition}

\begin{proposition} Sei \(f:V\to W\) linear, endlichdimensional.
Dann gilt:
\begin{align*}
	f injektiv &
	\\ \mKer(f) = \{0\}  &
	\\ \mRg(f) = \dim V &
\end{align*}
Andererseits gilt:
\begin{align*}
	\text{f surjektiv}
	\\ \mIm(f) = W  &
	\\ \mRg(f) = \dim W &
\end{align*}
\end{proposition}

\begin{korollar} Ist \(\mdim V = \mdim W \), so ist \(f\) injektiv $\iff $ \(f\) surjektiv.\end{korollar}
\begin{beobachtung}
	Das ist analog zu Abbildungen endlicher Mengen: Sind \(X,Y\) endliche Mengen, \(h: X\to Y\) eine Abbildung, so gilt:
	\begin{itemize}
		\item \(h\text{ injektiv} \iff |h(X)| = |X| \)
		\item \(h\text{ surjektiv} \iff |h(X)| = |Y| \)
		\item \(h\text{ bijektiv} \iff |X|=|h(X)|=|Y| \)
	\end{itemize}
\end{beobachtung}

\begin{proposition}[Dimensionsformel]
Sind \(U_1, U_2 \subset V\) Untervektorr"aume, dann gilt: \(\dim(U_1 +U_2) =\dim U_1 + \dim U_2 - \dim (U_1\cap U_2)\)
\end{proposition}
\begin{bem}
\(U_1\cap U_2\) ist stets ein Untervektorraum, wenn \(U_1\) und \(U_2\) Untervektorr"aume sind.
\end{bem}
\begin{proof}[Dimensionsformel] 

Betrachte die Abbildung \(f: U_1 \times U_2 \to U_1 + U_2, (u_1, u_2) \mapsto u_1 + u_2 \). Hierbei ist \(U_1\times U_2 = \{(u_1, u_2) | u_1 \in U_1, u_2 \in U_2 \} \) der Verktorraum der Paare \((u_1, u_2)\) mit
elementweisen Operationen:
\begin{align*}
(u_1, u_2) + (u_1', u_2') &= (u_1 + u_1', u_2 + u_2') \\
\lambda (u_1, u_2) &= (\lambda u_1, \lambda u_2)
\end{align*}
Oft nennt man \(U_1 \times U_2\) auch \gq{"au"sere direkte Summe} von \(U_1\) und \(U_2\) mit Bezeichnung \(U_1 \oplus U_2\).
\begin{bem}Die Kollision der
Bezeichnungen \(U_1\oplus U_2\) für die direkte Summe zweier Untervektorr"aume / "außere Summe ist harmlos ("Ubg).
\end{bem}

Nun gilt \(\dim \mKer(f) + \dim\mIm(f) = \dim(U_1\times U_2) \)

Es gilt weiterhin: \(\dim(U_1 \times U_2) = \dim U_1 + \dim U_2 \) Wenn \(e_1, \dots, e_k \) eine Basis in \(U_1\) und \(b_1, \dots, b_l \) eine Basis in \(U_2\) \(\implies\) \(e_1, \dots, e_k, b_1, \dots, b_l\) Basis in \(U_1\times U_2\)

Ferner gilt: \(\mIm(f) = U_1 + U_2\) per Definition. Nun ist:
\begin{align*}
\mKer(f) &= \{(u_1, u_2) | u_1 \in U_1, u_2\in U_2, u_1 + u_2 = 0\} = \{(u, -u)| u \in U_1 \cap U_2 \}\\
&\implies \dim\mKer (f) = \dim(U_1 \cap U_2)\\
&\implies \underbrace{\dim(U_1 \cap U_2)}_{\dim\mKer(f)} + \underbrace{\dim(U_1 + U_2)}_{\dim\mIm(f)} = \underbrace{\dim(U_1) + \dim(U_2)}_{\dim (U_1\times U_2)}
\end{align*}
\end{proof}

Was hat diese ganze Theorie mit Matrizen zu tun? Intuitiv: Abbildungen sind
\gq{geometrisch} und Matrizen sind Koordinatenformen dieser geometrischen
Abbildungen.

\begin{exa}
Strecken in Richtung von \(l_2\) mit Faktor 2. Wie beschreibt man \(f\) in
Koordinaten?
\end{exa}

\subsubsection{Abbildunngsmatrix}
\begin{definition}{}{}
Seien \(V,W\) zwei (\(\mathbb{K}\)-) Vektorr"aume. Dann nennen wir
\(\mHom_\mathbb{K}(V,W) = \{f: V\to W | f \text{ist }\mathbb{K}\text{-linear}\}\).
\end{definition}
\begin{proposition}
\(\mHom_\mathbb{K}(V,W)\) ist auch ein Vektorraum.
\end{proposition}
\begin{proof}
Wenn \(f_1, f_2 \in \mHom_\mathbb{K}(V,W)\):
\begin{align*}
(f_1 + f_2)(v) &= f_1(v) + f_2(v) \quad\text{ist linear} \\
(\lambda f_1)(v) &= \lambda\cdot f_1(v)\quad\text{ist auch linear}
\end{align*}
\end{proof}

Seien \(V, W\) endlichdimensional, \(B = \{b_1, \dots, b_n\}\) Basis in \(V\), \(C = \{c_1, \dots, c_m\}\) Basis in \(W\). Sei \(f \in \mHom_\mathbb{K}(V,W)\). Die Abbildungsmatrix \(F= M_C^B(f)\) in \(f\) bzgl. \(B\) und \(C\) ist definiert als
Matrix, deren Spalten die Koordinatenspalten von \(f(b_1),\dots, f(b_n)\) bzgl. der Basis \(C\) sind:
\begin{align*}
f(b_j) = \sum_{i=1}^mF_{ij}c_i \implies F \in \mathbb{K}^{m\times n}
\end{align*}

\textbf{Vorsicht!} \(M_C^B(f)\) h"angt von der Wahl der Basen B und C ab (\(f\) jedoch nicht!).

\begin{exa}
Rotation um \(\frac{\pi}{4}\) gegen Urzeigersinn.
\begin{align*}
M_B^B(f) = \begin{pmatrix}
\nicefrac{\sqrt{2}}{2} & -\nicefrac{\sqrt{2}}{2}\\
\nicefrac{\sqrt{2}}{2} & \nicefrac{\sqrt{2}}{2}
\end{pmatrix}
\end{align*}
\end{exa}

\begin{proposition} Seien \(V,W,B,C\) wie oben. Dann entsprechen die Abbildungsmatrizen
 \(F = M_C^B(f)\) den Abbildungen.  Genauer: Die Abbildung \(M_C^B: \mHom_\mathbb{K}(V,W)\to\mathbb{K}^{m\times n}\), \(f \mapsto M_C^B(f)\) ist ein Isomorphismus von
 Vektorra"umen.
\end{proposition}

\begin{proof}
Aus Definition von \(M_C^B(f)\) folgt sofort:
\begin{align*}
M^B_C(f_1 + f_2) &= M_C^B(f_1) + M_C^B(f_2)\\
M^B_C(\lambda\cdot f) &= \lambda\cdot M_C^B(f)
\end{align*}
Also ist \(M_C^B(f)\) eine lineare
Abbildung.

\(M_C^B\) ist injektiv: wenn \(f\) mit \(M^B_C(f) = 0\), dann gilt \(f(b_j) = 0\quad\forall j= 1,\dots n\); Da jeder Vektor \(v \in V\) eine Linearkombination von \(b_j\)s ist, folgt \(f(v) = 0 \quad\forall v \in V\) \(\implies\) der Kern ist null

\(M^B_C\) ist auch surjektiv: sei \(F\in \mathbb{K}^{m\times n}\) gegeben. Definiere eine Abbildung \(f: V\to W\) folgenderma"sen:
\begin{align*}
f\left(\sum_{j=1}^n\lambda_jb_j\right) = \sum_{j=1}^n\lambda_j\cdot\left(\sum_{i=1}^mF_{ij}c_i\right)=\sum_{i=1}^mc_i\sum_{j=1}^nF_{ij}\lambda_j
\end{align*}

\(f\) ist linear und es gilt:
\begin{align*}
f(b_j) = \sum_{i=1}^mF_{ij}c_i \implies M^B_C(f) = F
\end{align*}
\end{proof}

Im Beweis haben wir unter anderem festgestellt:
\begin{relation}
\begin{align*}
f\left(\sum_{j=1}^n\lambda_jb_j\right)=\sum_{i=1}^mc_i\underbrace{\sum_{j=1}^nF_{ij}\lambda_j}_{\mu_i}
\end{align*}

Das hei"st: Wenn \(v=\sum_{j=1}^n\lambda_jb_j\in V\) mit Koordinatenspalte \(\lambda = (\lambda_1,\dots,\lambda_n)^\top\) (bez"uglich \(B\)), \(w = f(v)\), dann hat \(w\) folgende Koordinatenspalte bez"uglich \(C\):
\begin{align}
\mu = F\cdot \lambda
\end{align}
\end{relation}

\begin{exa}
Wenn \(V=K^n, W=K^M\) dann hat \(V\) eine Basis \(\mathcal{E}_n = (e_1, \dots, e_n)\), wobei \(e_i\) der Vektor ist, der an der \(i\)-ten Stelle 1 ist und sonst 0. Und \(W\) hat die Basis \(\mathcal{E}_m = (e_1, \dots, e_m)\).
Sei \(A\in K^{m\times n }\), \(f_A: \mathbb{K}^n\to \mathbb{K}^m\), \(x \mapsto A\cdot x\). Dann gilt:
\begin{align*}
M^{\mathcal{E}_n}_{\mathcal{E}_m}(f_A) &= A\\
f_A(e_j) &= A\cdot e_j = j\text{-te Spalte von } A = a_j
\end{align*}
\end{exa}

Seien \(V, W, Z\) drei \(\mathbb{K}\)-Vektorr"aume, \(f \in \mHom_\mathbb{K}(V, W)\), \(g \in \mHom_\mathbb{K}(W,Z)\). Dann gilt: \(g\circ f \in \mHom_\mathbb{K}(V,Z)\).

\begin{proposition}
	 Seien \(B,C,D\) Basen in
	\(V,W,Z\). Dann gilt: \(M^B_D(g\circ f) = M^C_D(g)\cdot M^B_C(f)\).
\end{proposition}
\begin{proof}
 Sei \(F = M^B_C(f)\), \(G = M^C_D(g)\). Dann:
 \begin{align*}
 (g\circ f)(b_j) &= g(f(b_j)) = g\left(\sum_{i=1}^m F_{ij}c_i\right) = \sum_{i=1}^mF_{ij}g(c_i) \\&= \sum_{i=1}^m F_{ij}\sum_{k=1}^lG_{ki}d_k = \sum_{k=1}^l\underbrace{\left(\sum_{i=1}^mG_{ki}F_{ij}\right)}_{M^B_D(g\circ f)}d_k
 \end{align*}
\end{proof}

Wenn \(V\) ein Vektorraum ist, \(B,B'\) zwei Basen, dann erhalten wir die
 Basiswechselmatrix \(S\), deren Spalten die Koordinaten von den neuen
 Basisvektoren \((b_1', \dots, b_n')\) bez"uglich der alten Basis \(B = (b_1,
 \dots, b_n)\) sind. Es folgt sofort aus den Definitionen:
 \(S=M^{B'}_B(\mId_V)\)
 
\begin{lemma}[Basiswechselmatrizen sind invertierbar]
	\(S = M^{B'}_B(\mId_V)\) ist invertierbar f"ur je zwei Basen \(B, B' \subset V\), und \(S^{-1} = M^B_{B'}(\mId_V)\).
\end{lemma}
\begin{proof}
	\(M^B_B(\mId_V)=1_n\), daher ist
	\[\underbrace{M^{B'}_B(\mId_V)}_{S}\cdot M^B_{B'}(\mId_V) = M^B_B(\mId_V) = 1_n \]
\end{proof}


\subsubsection{Basen und Abbildungsmatrizen (Zusammenfassung)}
\label{sec:baab}
Jede Basis in V definiert eine Bijektion $\phi_B^{-1}V\rightarrow K^n, v\mapsto C$
\begin{relation}
  $\phi_B$ ist hierbei die Aufspannabbildung.  
\end{relation}

\textbf{Warnung} Unsere Konvention f"ur die Basiswechselmatrix: ''von $B$ zu
$B'$'' ist $M^{B'}_B$, also dr"uckt die meue Basis in der alten aus und ist
nicht die einzige!

\subsubsection{Physikerdefinition eines Vektors}
\label{sec:phyv}

Ein (kontravarianter) Vektor ist ein Objekt, welches durch ein n-Tuperl \(\lambda\)
von Zahlen beschrieben wird, die sich Koordinatentransformationenen so
verh"alt: Ist $S$ die Basiswechselmatrix der Koordinatentransformation, so gilt
in den neuen koordinaten $\lambda'=S^{-1}\lambda$

\begin{notte}
  Diese Transformationsformel haben wir gesehen: ist $v\in V$ mit der
  Koordinatenspalte $\lambda$ bzgl. $B$ so gilt
  $\lambda=M^{B'}_{Basiswechselmatrix}\cdot \lambda ' $
\end{notte}

\textbf{Lemma}(Verhalten von Abbildungsmatrizen unter Basiswechsel)
Sei .. eine Lineare Abbildung, ..., dann gilt:
\[
  M^{B'}_{C'}(f)=
\] In Anderen Worten: wenn $S$ die Basiswechselmatrix von ... so gilt

\begin{proof}
  $M^{B'}_{C'}(f)=$
\end{proof}

\begin{notte}[zur Basiswechselmatrix]
  $ M^{B'}_{B}$ ist die Basiswechselmatrix von $B$ zu $B'$: sie enth"alt die
  Koordinaten von Vektoren aus $B'$ bzgl. $B$. Wie verhalten sich die
  Kootdinaten von Vektoren bzgl. $B$ und $B'$ Antwort: $\lambda =
  S^{-1}\lambda$.
\end{notte}

\begin{exa}
  $V=R^{2}$
\end{exa}

\begin{proposition}[Rangform einer linearen Abbildung]
  Sei ... eine Lineare Abbildung und $V,W$ endlichdimensional. Dann existieren
  Basen ... , so dass ...
\end{proposition}

\begin{korollar}
  F"ur jede Matrix ... gibt es invertierbare Matrizen ... , so dass
\end{korollar}

\begin{proof}
  Sei ... w"ahle eine Basis ... im $\text{Ker}(f)$  und erweitere sie zu einer
  Basis ... in Basis. Setze: ... Beim Beweis der Dimensionsformal haben wir
  gesehen ... . Erweitere jetzt ... zu einer Basis ... in $W$. Es gilt. 
\end{proof}

\subsubsection{Konsequenzen f"ur Matrizen}
\label{sec:konsma}

Sei ...
$Rg(A)=Rg(f_A))=dim<a_1,...,a_n>$ ...

\begin{korollar}[Kriterien f"ur Invertierbarkeit]
  Sei ... Folgende bedingungen sin ... :
  \begin{enumerate}
  \item Wenn ..., dann ist $f$ ein Isomorphismus
  \item A ist invertierbar.
  \item $A^T$ ist Invertierbar
  \item $\exists B\in K^n$ mit (Ist ''Rechtsinvertierbar'')
  \item ... (Ist ''Linksinvertierbar'')
  \item ... A hat vollen Rang
  \item Kern ist die leere Menge
  \item Das LGS $A\cdot x = 0$ hat nur die triviale L"osung $x=0$
  \item Das LGS $A\cdot x = n$ hat f"ur jedes $b\in $ genau eine L"osung
  \item A ist darstellbar als das Produkt von Elementarmatrizen
  \end{enumerate}

  Ausserdem gilt: wenn ... wie in ..., so ist ... 
\end{korollar}

Aus dem Satz ergibt sich folgendes Verfahren zur Berechnung von $A^{-1}$: ...
Das stimmt denn: dieses Verfahren l"ost gleichzeitig die LGS ...

\begin{lemma}[Injektivit"at und Surjektivit"at bei Verkn"upfungen]
  Seien $X,Y,Z$ Mengen, ... Dann gilt:  
\end{lemma}

\begin{proof}[Beweisschema]
  \begin{enumerate}
  \item (0) $ \iff $ (1)
  \item (1) $ \iff $ (2): $B:=A^{-1}$
  \item (1) $ \iff $ (3): $C:=A^{-1}$
  \end{enumerate}
\end{proof}

\begin{proof}{des Satzes}\leavevmode
  \begin{itemize}
  \item[(0) $\implies$ (1)]  $A=M^{\varepsilon}_{\varepsilon}(f_A)$, woberi $f_A:$
  \item[(1) $\implies$ (0)] Sei $A$ invertierbar, $A=$ Betrachten wir die
    Abbildung $g$ mit ..., es gilt: 
  \item[(1) $\iff$ ($1^T$)] ...
  \item[(4) $ \iff $ (5) $ \iff $ (6)] nach dem Lemma
  \item[(7) $ \iff $ (5)]
  \item[(8) $ \iff $ ((5) $ \iff $ (6))]
  \item[(2) $ \implies $ (6)] Wenn $A\cdot B=1_n$, dann muss nach dem Lemmma die
    Abbildung $x\mapsto A\cdot x$ surjektiv sein $\implies$ (6)
  \item[(3) $ \implies $ (5)] $C\cdot A = 1_n$
  \item[(9) $ \implies $ (1)] Elementarmatrizen sind invertierbar, und wenn...
  \item[(7) $ \implies $ (9)] L"osen wir das LGS $A\cdot x = 0$ durch
    Zeilenumformungen (Gauss-Jordan): also machen wir Zeilenumformungen. Wir
    haben keine freien Variablen, denn wir haben nur eine L"osung $\implies$
    $A$ wird zur $1_n$ durch Zeilenumformungen transformiert. Wenn wir den
    Prozess invertieren, so transformieren wir $1_n$ zu $A$ durch
    Zeilenumformungen. \\
    Jede Zeilenumformung entspricht der Multiplikation mit einer
    Elementarmatrix von links: \[A=T_1 \cdot \dots \cdot 1_n = T_1 \cdot \cdots
      \cdot T_e\]
    Nun wenn $A\cdot B = 1_n$ wie in (2) $\implies$, \\
    Wenn wir ... \\
    Analog
  \end{itemize}
\end{proof}

\subsubsection{Determinanten}
\label{sec:det}

Motivation: Wir haben viele Kriterien f"ur invertierbarkeit, aber bislang kein
Kriterium, welches effizient berechenbar w"ahre. Ausserdem fehlt uns zur Zeit
eine geometrische Interpretation f"ur invertierbarkeit.

\begin{relation}
\begin{trivlist}
  \item Wir haben gesehen: $A\in K^{n\times n}$ ist genau dann invertierbar, wenn ihre Spalten
  (Zeilen) linear unabh"angig sind.
 
  \item  Wenn $A\in \mathbb{K}^{2\times 2}$, dann sind ihre
  Spalten linear unabh"angig genau dann, wenn die entsprechenden Vektoren $a_1,
  a_2 \in \mathbb{R}^2$ ein ''nicht ausgeartetes'' Parallelogramm aufspannen. \\

  \item Analog sind $a_1, a_2, a_3$ linear unabh"angig in $\mathbb{R}^3$.
\end{trivlist}
\end{relation}

Wenn wir diese Ideen in einen ''abstrakten'' Vektorraum uebertragen wollen,
m"ussen wir also Volumen invertieren, indem wir nur die Struktur des
Vektorraums benutzen.\\

Welche Eigenschaften hat das Volumen bzgl. der Vektorraumstruktur in
$\mathbb{R}^2$ bzw. $\mathbb{R}^3$?
\begin{itemize}
\item $\mVol(a_1,a_2)=\mVol(a_1, a_2 + \lambda \cdot a_1)\; \forall \lambda \in
  \mathbb{R}$
\item $\mVol(a_1,a_1)=0$
\item Bei Streckungen von einem Vektore multipliziert das Volumen mit dem
  entsprechenden Faktor.
  \begin{align*}
    Vol(\lambda\cdot a_1, a_2)=\lambda\cdot \mVol(a_1,a_2) \tag{$\lambda \geq 0$}
    \end{align*}
\end{itemize}

\begin{definition}{Volumenform}
  Eine \textbf{Volumenform} $\omega$ auf $V$ ist eine Abbildung ... mit
  folgenden Eigenschaften:
  \begin{enumerate}
  \item $\omega$ ist linear in jeder Variable: ...
  \item $\omega$ ist alternierend: ...
  \end{enumerate}
\end{definition}

\begin{notte}[zu (2)]
  ...
\end{notte}

\begin{proof}[zur Bemerkung]
  Wir lassen die Variablen ausser ... fest und rechnen.
\end{proof}

\begin{notte}
  Genau wie bei Volumina von Quadern in $\mathbb{R}^3$ die Wahl einer
  Masseinheit das Volumen eindeutig.
\end{notte}

\begin{lemma}
  Sei $\omega: V^n\mapsto K$ eine Volumenform und sei $B=(b_1m, ... , b_n)$
  eine Basis in $V$. Der Wert ....  ..
\end{lemma}

\begin{proof}[des Lemmas]
  Stelle $v_i=\sum{\lambda_{ij}b_j}$ dar ($B$ ist ja eine Basis) und benutze
  Linearit"at von $\omega$.
  \begin{align*}
    w(v_1, ..., v_n)=\text{Summe von Termen der Form } (...)
  \end{align*}

  Es ueberleben nur Terme, wo $b_{j1},...,b_{ji}$ in anderer Reihenfolge sind
  $\implies$ dann ist ... $\implies$ ... Summe von Termen der Form
\end{proof}
\begin{proposition}
  Sei ... eine Volumenform auf V. Dann gilt:
  \begin{enumerate}
  \item $\omega(v_1,...,v_n)$ ...
  \item Sind ... linear unabh"angig, so gilt
  \end{enumerate}
\end{proposition}

\begin{proof}
  \begin{enumerate}
  \item Linearit"at: 
  \item Sind ... linear abh"angig, dann ist ein $V_J=\sum$
  \end{enumerate}
\end{proof}

Wir wollen die Abh"angigkeit aus dem lemma explizit aufschreiben

\subsubsection{Permutationen}
\label{sec:perm}

\begin{definition}{Permutationen}{}
  Sei $X$ eine Menge. Eine Bijektion $\sigma: X \to X$ heißt Permutation
  von X. \newline
  Für \(X = \{1,...,n\}\) heißt die Menge aller Permutationen \(\sigma: \{1,...,n\}\to\{1,...,n\}\) die symmetrische Gruppe auf n Elementen. Notation: \(S_n = \{ \sigma: \{1,...,n\}\to\{1,...,n\}|\sigma\text{ bijektiv}\}\)
\end{definition}

\begin{exa}
  Ein Elelement \(\sigma \in S_n\) schreibt man h"aufig als Tabelle:
  
  \(\left(\begin{array}{cccc}
  	1 & 2 & ... & n\\
  	\sigma(1) & \sigma(2) & ... & \sigma(n)\\
  \end{array}\right)\)
\end{exa}

\begin{notte}
  $\tau\circ\sigma \not= \sigma\circ\tau$
\end{notte}

Da Verknüpfungen von Bijektionen bijektiv sind, definiert die Verknüpfung eine Abbildung \(\circ: S_n \times S_n \to S_n \) (Multiplikation von Permutationen). Permutationen kann man auch invertieren: \(\sigma \in S_n \implies \sigma^{-1} \in S_n \)

\begin{definition}{Halbysystem}{}
  Sei \(P=\{(i,j) \in \{1,...,n\}^2\mid i\neq j \} \) Ein Halbsystem \(H \subseteq P \) ist eine Teilmenge mit der Eigenschaft: Von den Paaren \((i,j) \) und \((j,i) \) ist immer genau eines in P enthalten. Formal:
  \(\forall (i,j) \in P: ((i,j) \in H \land (j,i) \notin H) \lor ((j,i) \in H \land (i,j) \notin H )\)
\end{definition}

\begin{exa}
  Für \(n=3 \) ist \(\{(1,2), (1,3), (2,3)\} \) ein Halbsystem.
\end{exa}

\begin{definition}{Vorzeichen}{}
  Sei \(\sigma \in S_n. \quad \varepsilon(\sigma):= \prod\limits_{(i,j)\in H} \frac{\sigma(i)-\sigma(j)}{i-j} \) heißt Vorzeichen von \(\sigma \).
\end{definition}

\(\varepsilon(\sigma) \) ist unabhängig von der Wahl von H, denn: Wenn \(H^\prime \) ein anderes Halbsystem ist, muss ggf. \((j,i) \) statt \((i,j) \) genommen werden, aber \(\frac{\sigma(j)-\sigma(i)}{j-i} = \frac{\sigma(i)-\sigma(j)}{i-j} \).
Das heißt, wir können uns einfach ein Halbsystem aussuchen:
\(\varepsilon(\sigma) = \prod\limits_{i<j}\frac{\sigma(i)-\sigma(j)}{i-j} \)

\textbf{Interpretation für \(\varepsilon(\sigma) \)}:
\[\varepsilon(\sigma) = \prod\limits_{i<j}\frac{\sigma(i)-\sigma(j)}{i-j} = \prod_{i<j} sgn(\sigma(i)-\sigma(j)) = \prod_{\substack{i<j\\ \sigma(i)>\sigma(j)}} (-1)\]\[ = \text{Minus eins hoch Anzahl der „Reihenfolgenverstöße“ von }\sigma\]


\begin{definition}{}{}
  Die Permutationen ... heissen gerade ... ungerade.
  Sei $i\not=j$ ... Die Transposition von ... ist die Permutation: ..
  (Vertauscht $i$ und $j$, alle andere bleibt gleicht)
\end{definition}


\begin{proposition}[das Vorzeichen ist multiplikativ]
  $\varepsilon(\sigma\circ \tau)= \varepsilon (\sigma) \cdot \varepsilon (\tau)$
\end{proposition}


\begin{relation}
  Sei $\tau_{ij}$ eine Transposition: $\varepsilon(\tau_{ij})=-1$.
\end{relation}

\begin{proposition}
  Wenn ... ein Produkt von $n$ Transpositionen, dann gilt: ...
\end{proposition}

\begin{notte}
  Jede Permutation ist als Produkt von Transpositionen darstellbar ("Ubung)
\end{notte}


Sei \(V\) ein Vektorraum, \(b_1, ..., b_n\) eine Basis, \(v_1, ..., v_n \in
V\) mit Darstellungen \(v_i = \sum_{i=1}^n{\lambda_{ij}\cdot b_j}\).

\todo{Das stand so nicht an der Tafel, aber \((-1)^\varepsilon\) kam mir spanisch vor}
\begin{align*}
    \omega(v_1,\dots,v_n) &= \omega\left(\sum_{j_1=1}^n\lambda_{1,j_1}b_{j_1},\dots,\sum_{j_n=1}^n\lambda_{n,j_n}b_{j_n}\right) \\
    &= \sum_{j_1,\dots,j_n=1}^n\left(\underbrace{\lambda_{1,j_1}\dots\lambda_{n,j_n}}_{\text{Produkt}}\cdot\omega(b_{j_1},\dots,b_{j_n})\right) \\
    &= \sum_{\sigma\in S_n}\left(\lambda_{1,\sigma(1)}\dots\lambda_{n,\sigma(n)}\omega(b_{\sigma(1)},\dots,b_{\sigma(n)})\right) \\
    &= \left(\sum_{\sigma\in S_n}{\varepsilon(\sigma)}\lambda_{1,\sigma(1)}\dots\lambda_{n,\sigma(n)}\right)\omega(b_1,\dots,b_n)
\end{align*}


 Wenn \(V=K^n, (b_1,\dots,b_n) = (e_1,\dots,e_n)\) die Standardbasis in \(K^n\), und die Volumenform \(\omega:
 V^n\to K\) so gew"ahlt ist, dass \(\omega(e_1,...,e_n)=1\), dann bekommt man
 folgende Definition f"ur eine Matrix \(A\):

 \begin{definition}{}{}
 	Sei \(A\in K^{n\times n}\) eine Matrix mit Eintr"agen \(a_{ij}\). Wenn man die Zeilen von \(A\) (sagen wir \(\alpha_1,\dots, \alpha_n\)) als
   Vektoren in \(K^n\) auffast, dann gilt: \[\omega(\alpha_1,\dots,\alpha_n) = \sum_{\sigma\in S_n}{\varepsilon(\sigma)a_{1,\sigma(1)}\dots a_{n,\sigma(n)}} =: \det A \]
   Das nennen wir ab jetzt auch Leibniz-Formel.
 \end{definition}

 \begin{relation}
   Geometrische Bedeutung: \(\det A\) ist das (orientierte) Volumen des Quaders / Parallelotops aufgespannt durch Zeilen von \(A\) (wenn man das Volumen des \gq{Standardquaders} aufgespannt durch Standardzeilen \(e_1^T,\dots,e_n^T\) gleich 1 setzt)
 \end{relation}
\begin{proposition}
	Sei \(V\) ein \(n\)-dimensionaler Vektorraum. Es gibt eine bis auf ein Vielfaches eindeutige Volumenform \(\omega \neq 0\) auf \(V\).
\end{proposition}
\begin{proof}
	W"ahle eine Basis \(B= (b_1,\dots,b_n) \in V\), definiere \(\omega(v_1,\dots,v_n)\) durch die Leibniz-Formel mit \(\omega(b_1,\dots,b_n):=1\). Dann ist \(\omega\neq 0\) nach Konstruktion. \(\omega\) ist linear in jeder Variable, weil \(\det\Delta \) definiert durch die Leibniz-Formel linear in jeder Zeile der Matrix \(\Delta\) ist. Wir m"ussen noch zeigen, dass es alternierend ist: \[\omega(v_1,\dots,v_{i-1},v_i,v_{i+1},\dots,v_{j-1},v_j,v_{j_1},\dots,v_n) = \det\Delta \cdot\omega(b_1,\dots,b_n) \] Wobei die \(i\)-te und \(j\)-te Zeile von \(\Delta\) "ubereinstimmen (es gilt also \(\lambda_{i,x}=\lambda_{j,x}\)). Es gilt:
	\begin{align*}
	\det\Delta &= \sum_{\sigma\in S_n}\varepsilon(\sigma)\lambda_{1,\sigma(1)}\dots\lambda_{n,\sigma(n)} \\
	&= \sum_{\sigma = \sigma'\circ\tau_{ij}\in S_n}\varepsilon(\sigma)\lambda_{1,\sigma(1)}\dots\lambda_{i,\sigma(i)}\dots\lambda_{j,\sigma(j)}\dots\lambda_{n,\sigma(n)} \\
	&= \sum_{\sigma\in S_n}\varepsilon(\sigma'\circ\tau_{ij})\cdot\lambda_{1,\sigma'(1)}\dots\overbrace{\lambda_{i,\sigma'(j)}}^{=\lambda_{i,\sigma(i)}}\dots\overbrace{\lambda_{j,\sigma'(i)}}^{=\lambda_{j,\sigma(j)}}\dots \lambda_{n,\sigma'(n)} \\
	&= \mcolor{-} \sum_{\sigma'\in S_n} \mcolor{\varepsilon(\sigma')} \lambda_{1,\sigma'(1)}\dots\lambda_{i,\sigma'(i)}\dots\lambda_{j,\sigma'(j)}\dots\lambda_{n,\sigma'(n)} = -\det A \\
	&\implies \det A = 0
	\end{align*}
	\(\implies\omega \) alternierend.
\end{proof}

\(\omega\) ist eindeutig durch den Wert (=1) auf \(B\) bestimmt \(\implies\)  wenn \(\omega'\) eine Volumenform auf \(V\) ist, so gilt \(\omega' = \underbrace{\omega'(b_1,\dots,b_n)}_{\in K}\cdot\omega\)

\begin{definition}{Determinante}
	Sei \(V\) ein \(n\)-dimensionaler Vektorraum, \(f: V\to V\) linear. Die Determinante von \(f\) ist
	\[\det(f):= \frac{\omega(f(b_1),\dots,f(b_n)}{\omega(b_1,\dots,b_n)}\]
	wobei \(\omega \neq 0\) eine Volumenform auf \(V\) ist.
\end{definition}

\begin{beobachtung}
	Die obige Proposition zeigt, dass \(\det(f)\) wohldefiniert ist (= auf die Wahl von \(\omega\neq 0\) nicht ankommt). Geometrisch: \(\det(f)\) ist der Verzerrungsfaktor von dem Volumen (unter \(f\)).
\end{beobachtung}

\begin{lemma}
	\(\det\Delta = \det\Delta^T\quad\forall \Delta\in K^{n\times n}\)
\end{lemma}
\begin{proof}
	Es gilt: \(\varepsilon(\sigma) = \varepsilon(\sigma^{-1})\quad\forall \sigma\in S_n \) (folgt z.B. aus \(\varepsilon(\sigma\circ\sigma^{-1})=1=\varepsilon(\sigma)\varepsilon(\sigma^{-1})\))
	\begin{align*}
	\det\Delta^T = \sum_{\sigma\in S_n}\varepsilon(\sigma)\lambda_{\sigma(1),1}\dots\lambda_{\sigma(n),n} = \sum_{\sigma^{-1}\in S_n}\varepsilon(\sigma^{-1})\lambda_{1,\sigma^{-1}(1)}\dots\lambda_{n,\sigma^{-1}(n)} = \det \Delta
	\end{align*}
\end{proof}
\begin{proposition}
	\begin{enumerate}
		\item \(f, g: V\to V \) linear \(\implies\det(g\circ f) = \det(g)\det(f) \)
		\item \(f\) ist genau dann invertierbar, wenn \(\det(f)\neq 0 \)
	\end{enumerate}
\end{proposition}
\begin{korollar}
	Sei \(B\) eine Basis in \(V\), \(f:V\to V \) linear. Dann gilt: \(\det(f) = \det M^B_B(f) \)
\end{korollar}
\begin{proof}
	\begin{align*}
	\det(f) = \frac{\omega(f(b_1),\dots,f(b_n)}{\omega(b_1,\dots,b_n)} \stackrel{Def. Abb. + Leibniz}{=} \det M^B_B(f)^T \stackrel{Lemma}{=} \det M_B^B(f)
	\end{align*}
\end{proof}
\begin{korollar}
	Eine Matrix \(A\in K^{n\times n} \) ist genau dann invertierbar, wenn \(\det A \neq 0 \)
\end{korollar}
\begin{bem}
	Es gibt eine Reihe von Begriffen f"ur Matrizen, die nach unseren Invertierbarkeitskriterien alle "aquivalent dazu sind, dass \(A \) invertierbar ist, z.B. nicht ausgeartet, regul"ar etc.
\end{bem}
\begin{korollar}
Wie berechnet man Determinanten?

Die Leibniz-Foreml ist zu ineffizient (sie hat \(n!\) Summanden). Die Berechnungen macht man normalerweise unter Benutzung folgender Eigenschaften der Determinante (die daraus folgen, dass \(\det : (K^n)^n \to K\) eine Volumenform ist):
\begin{itemize}
	\item Wenn man zu einer Zeile / Spalte der Matrix ein Vielfaches einer anderen Zeile bzw. Spalte addiert, so "andert sich die Determinante nicht;
	\item \(\det \) ist linear in jeder Zeile bzw. Spalte.
\end{itemize}
\end{korollar}
\begin{bem}
	F"ur \(\varepsilon(\sigma)\), das Vorzeichen einer Permutation, gibt's auch die Physiker-Notation: \(\varepsilon_{1,\dots,n}=1\) und es ver"andert das Vorzeichen, wenn man zwei Indizes vertauscht.
\end{bem}
\begin{bem}[Orientierung von Vektorr"aumen]
	Volumenformen sind laut unserer Definition linear \(\implies\) selbst wenn \(V\) ein \(\mathbb{R}\)-Vektorraum ist, kann es passieren, dass \(\omega(v_1,\dots,v_n)< 0\). Die Wahl von einer Volumenform \(\omega\) definiert die sogenannte \emph{Orientierung} von \(V\) (\(V\) ein \(mathbb{R}\)-VR). Wenn wir \(\omega\) fixieren, dann entstehen zwei Klassen von Basen in \(V\): \(B = (b_1,\dots,b_n)\) hei"st positiv (negativ) orientiert, wenn \(\omega(b_1,\dots,b_n) > (<) 0\). 
	
	In diesem Sinne ist \(\omega(v_1,\dots,v_n)\) das \emph{orientierte} Volumen von dem Quader aufgespannt durch \(v_1,\dots,v_n\): wenn \(v_1,\dots,v_n\) positiv orientiert sind, ist \(\omega(v_1,\dots,v_n) = \) \(+\)Volumen, andernfalls \(-\)Volumen.
\end{bem}

\begin{proposition}
	Sei \(V\) ein \(n\)-dimensionaler Vektorraum. Es gibt eine bis auf ein Vielfaches eindeutige Volumenform \(\omega \neq 0\) auf \(V\).
\end{proposition}
\begin{proof}
	W"ahle eine Basis \(B= (b_1,\dots,b_n) \in V\), definiere \(\omega(v_1,\dots,v_n)\) durch die Leibniz-Formel mit \(\omega(b_1,\dots,b_n):=1\). Dann ist \(\omega\neq 0\) nach Konstruktion. \(\omega\) ist linear in jeder Variable, weil \(\det\Delta \) definiert durch die Leibniz-Formel linear in jeder Zeile der Matrix \(\Delta\) ist. Wir m"ussen noch zeigen, dass es alternierend ist: \[\omega(v_1,\dots,v_{i-1},v_i,v_{i+1},\dots,v_{j-1},v_j,v_{j_1},\dots,v_n) = \det\Delta \cdot\omega(b_1,\dots,b_n) \] Wobei die \(i\)-te und \(j\)-te Zeile von \(\Delta\) "ubereinstimmen (es gilt also \(\lambda_{i,x}=\lambda_{j,x}\)). Es gilt:
	\begin{align*}
	\det\Delta &= \sum_{\sigma\in S_n}\varepsilon(\sigma)\lambda_{1,\sigma(1)}\dots\lambda_{n,\sigma(n)} \\
	&= \sum_{\sigma = \sigma'\circ\tau_{ij}\in S_n}\varepsilon(\sigma)\lambda_{1,\sigma(1)}\dots\lambda_{i,\sigma(i)}\dots\lambda_{j,\sigma(j)}\dots\lambda_{n,\sigma(n)} \\
	&= \sum_{\sigma\in S_n}\varepsilon(\sigma'\circ\tau_{ij})\cdot\lambda_{1,\sigma'(1)}\dots\overbrace{\lambda_{i,\sigma'(j)}}^{=\lambda_{i,\sigma(i)}}\dots\overbrace{\lambda_{j,\sigma'(i)}}^{=\lambda_{j,\sigma(j)}}\dots \lambda_{n,\sigma'(n)} \\
	&= \mcolor{-} \sum_{\sigma'\in S_n} \mcolor{\varepsilon(\sigma')} \lambda_{1,\sigma'(1)}\dots\lambda_{i,\sigma'(i)}\dots\lambda_{j,\sigma'(j)}\dots\lambda_{n,\sigma'(n)} = -\det A \\
	&\implies \det A = 0
	\end{align*}
	\(\implies\omega \) alternierend.
\end{proof}

\(\omega\) ist eindeutig durch den Wert (=1) auf \(B\) bestimmt \(\implies\)  wenn \(\omega'\) eine Volumenform auf \(V\) ist, so gilt \(\omega' = \underbrace{\omega'(b_1,\dots,b_n)}_{\in K}\cdot\omega\)

\begin{definition}{Determinante}{}
	Sei \(V\) ein \(n\)-dimensionaler Vektorraum, \(f: V\to V\) linear. Die Determinante von \(f\) ist
	\[\det(f):= \frac{\omega(f(b_1),\dots,f(b_n)}{\omega(b_1,\dots,b_n)}\]
	wobei \(\omega \neq 0\) eine Volumenform auf \(V\) ist.
\end{definition}

\begin{beobachtung}
	Die obige Proposition zeigt, dass \(\det(f)\) wohldefiniert ist (= auf die Wahl von \(\omega\neq 0\) nicht ankommt). Geometrisch: \(\det(f)\) ist der Verzerrungsfaktor von dem Volumen (unter \(f\)).
\end{beobachtung}

\begin{lemma}
	\(\det\Delta = \det\Delta^T\quad\forall \Delta\in K^{n\times n}\)
\end{lemma}
\begin{proof}
	Es gilt: \(\varepsilon(\sigma) = \varepsilon(\sigma^{-1})\quad\forall \sigma\in S_n \) (folgt z.B. aus \(\varepsilon(\sigma\circ\sigma^{-1})=1=\varepsilon(\sigma)\varepsilon(\sigma^{-1})\))
	\begin{align*}
	\det\Delta^T = \sum_{\sigma\in S_n}\varepsilon(\sigma)\lambda_{\sigma(1),1}\dots\lambda_{\sigma(n),n} = \sum_{\sigma^{-1}\in S_n}\varepsilon(\sigma^{-1})\lambda_{1,\sigma^{-1}(1)}\dots\lambda_{n,\sigma^{-1}(n)} = \det \Delta
	\end{align*}
\end{proof}
\begin{proposition}
	\begin{enumerate}
		\item \(f, g: V\to V \) linear \(\implies\det(g\circ f) = \det(g)\det(f) \)
		\item \(f\) ist genau dann invertierbar, wenn \(\det(f)\neq 0 \)
	\end{enumerate}
\end{proposition}
\begin{korollar}
	Sei \(B\) eine Basis in \(V\), \(f:V\to V \) linear. Dann gilt: \(\det(f) = \det M^B_B(f) \)
\end{korollar}
\begin{proof}
	\begin{align*}
	\det(f) = \frac{\omega(f(b_1),\dots,f(b_n)}{\omega(b_1,\dots,b_n)} \stackrel{Def. Abb. + Leibniz}{=} \det M^B_B(f)^T \stackrel{Lemma}{=} \det M_B^B(f)
	\end{align*}
\end{proof}
\begin{korollar}
	Eine Matrix \(A\in K^{n\times n} \) ist genau dann invertierbar, wenn \(\det A \neq 0 \)
\end{korollar}
\begin{bem}
	Es gibt eine Reihe von Begriffen f"ur Matrizen, die nach unseren Invertierbarkeitskriterien alle "aquivalent dazu sind, dass \(A \) invertierbar ist, z.B. nicht ausgeartet, regul"ar etc.
\end{bem}
\begin{korollar}
Wie berechnet man Determinanten?

Die Leibniz-Foreml ist zu ineffizient (sie hat \(n!\) Summanden). Die Berechnungen macht man normalerweise unter Benutzung folgender Eigenschaften der Determinante (die daraus folgen, dass \(\det : (K^n)^n \to K\) eine Volumenform ist):
\begin{itemize}
	\item Wenn man zu einer Zeile / Spalte der Matrix ein Vielfaches einer anderen Zeile bzw. Spalte addiert, so "andert sich die Determinante nicht;
	\item \(\det \) ist linear in jeder Zeile bzw. Spalte.
\end{itemize}
\end{korollar}
\begin{bem}
	F"ur \(\varepsilon(\sigma)\), das Vorzeichen einer Permutation, gibt's auch die Physiker-Notation: \(\varepsilon_{1,\dots,n}=1\) und es ver"andert das Vorzeichen, wenn man zwei Indizes vertauscht.
\end{bem}
\begin{bem}[Orientierung von Vektorr"aumen]
	Volumenformen sind laut unserer Definition linear \(\implies\) selbst wenn \(V\) ein \(\mathbb{R}\)-Vektorraum ist, kann es passieren, dass \(\omega(v_1,\dots,v_n)< 0\). Die Wahl von einer Volumenform \(\omega\) definiert die sogenannte \emph{Orientierung} von \(V\) (\(V\) ein \(mathbb{R}\)-VR). Wenn wir \(\omega\) fixieren, dann entstehen zwei Klassen von Basen in \(V\): \(B = (b_1,\dots,b_n)\) hei"st positiv (negativ) orientiert, wenn \(\omega(b_1,\dots,b_n) > (<) 0\). 
	
	In diesem Sinne ist \(\omega(v_1,\dots,v_n)\) das \emph{orientierte} Volumen von dem Quader aufgespannt durch \(v_1,\dots,v_n\): wenn \(v_1,\dots,v_n\) positiv orientiert sind, ist \(\omega(v_1,\dots,v_n) = \) \(+\)Volumen, andernfalls \(-\)Volumen.
\end{bem}

\subsubsection{Eigenvektoren, Eigenwerte, Diagonalisierbarkeit}

Wir haben gesehen: wenn \(f: V\to W\) linear, \(V, W\) endlichdimensionale Vektorr"aume \(\implies \exists B\subset V, C\subset W \) Basen so dass
\[M^B_C(f) = \begin{pmatrix}
1_r & 0\\
0 & 0
\end{pmatrix}, r = \mRg f\]

\begin{definition}{Endomorphismus, lineare Operatoren}{}
	Ein Endomorphismus von \(V\) ist eine lineare Abbildung \(f: V\to V\). Bezeichnung: \(f\in \text{End}_K(V) = \text{Hom}_K(V,V) \)	
	
	Endomorphismus hei"sen auch lineare Operatoren auf \(V\). Die Wahl einer Basis \(B\subset V \) gibt uns eine Matrix \(M^B_B(f)\).
\end{definition}
\begin{korollar}[Hauptfrage]
	Wenn \(f\) gegeben ist, wie findet man eine Basis \(B\), so dass \(M^B_B(f)\) eine besonders einfache Form hat?
	
	"Aquivalent, in Termen von Matrizen: Gegeben eine Matrix \(A\in K^{n\times n} \), finde eine invertierbare Matrix \(S\) so dass \(S^{-1}AS\) besonders einfache Form hat.
\end{korollar}
  Sei \(V\) ein \(\mathbb{R}\)-Vektorraum, sei \(B\) eine Basis mit \[M^B_B(f) =
    \begin{pmatrix}
      1 & 0 & 0 \\
      0 & 2 & 0 \\
      0 & 0 & 3
    \end{pmatrix} = \mDiag(1,2,3) \]
  Das hei"st, \(f\) ist in diesem Fall eine Streckung mit Koeffizienten 1, 2, oder 3 in Richtung von \(b_1\), \(b_2\) oder \(b_{3}\).

  \begin{definition}{Eigenvektor, Eigenwert}{}
    Sei \(f: V\to V\) eine lineare Abbildung. Ein Eigenvektor von \(f\) ist ein Vektor \(v\neq 0\), so dass \(f(v) = \lambda v\) f"ur ein \(\lambda \in K\), \(\lambda\) hei"st \emph{Eigenwert} von \(f\). Man sagt, dass \(v\) Eigenvektor zum Eigenwert \(\lambda\) ist. Geometrisch gesehen: \(f\) streckt \(v\) mit Koeffizienten \(\lambda\).
  \end{definition}
  \begin{lemma}
    Sei \(f: V\to V\) ein Endomorphismus. Wenn \(B = (b_{1},\dots,b_{n})\) eine Basis in \(V\) aus Eigenvektoren zu Eigenwerten \(\lambda_{1},\dots,\lambda_{n}\) ist, so gilt:
    \[M^{B}_{B}(f)=
      \begin{pmatrix}
        \lambda_{1} & & 0 \\
        & \ddots & \\
        0 & & \lambda_{n}
      \end{pmatrix}
      = \mDiag(\lambda_{1},\dots,\lambda_{n})
    \]
  (\(B\) hei"st auch Eigenbasis.)
\end{lemma}
\begin{proof}
  Es gilt laut Definition \(f(b_i) = \lambda_ib_i\) wobei \(i = 1, \dots, n\) \(\implies\) die Form von \(M^B_B(f)\) folgt aus den Definitionen.
\end{proof}
Es folgt: Wenn wir die Diagonalform von \(M^B_B(f)\) errichen m"ochten, m"ussen wir Eigenwerte und Eigenvektoren suchen!
\begin{definition}{Diagonalisierbarkeit}{}
  Ein Endomorphismus \(f\) hei"st diagonalisierbar, wenn er eine Eigenbasis hat.
\end{definition}
\begin{proposition}
  Sei \(f\in \mEnd_K(V), \lambda\in K\).
  \begin{enumerate}
  \item \(v\in V\) ist ein Eigenvektor von \(f\) zum Eigenwert \(\lambda \iff v\in\mKer(\lambda\mId_V - f)\setminus\{0\}\)
  \item \(\lambda\) ist ein Eigenwert von \(f\) (d.h., es existieren Eigenvektoren zum Eigenwert \(\lambda\iff\det(\lambda\mId_V-f)=0\))
  \end{enumerate}
\end{proposition}
\begin{proof}
  \begin{enumerate}
  \item Sei \(v\in V\) ein Eigenvektor (Eigenvektoren sind per Definition \(\neq 0\)) zum Eigenwert \(\lambda\). Nach Definition gilt:
    \begin{align*}
      f(v) &= \lambda v \iff f(v) - \lambda v = 0 \iff (f - \lambda\mId_V)(v) = 0 \\
      &\iff v\in \mKer(\lambda\mId_V - f)
    \end{align*}
    Andersherum: Wenn \(v \in \mKer(\lambda\mId_V -f)\setminus\{0\}\), dann ist \((\lambda\mId_V -f)(v)=0\land v\neq 0 \implies v\) ist Eigenvektor zum Eigenwert \(\lambda\).
  \item Nach (oben) ist \(\lambda\) ein Eigenwert \(\iff\mKer(\lambda\mId_V -f)\neq 0\)\(\iff \lambda\mId_V -f\) ist nicht invertierbar \(\iff\det(\lambda\mId_V - f) = 0\)
  \end{enumerate}
\end{proof}
F"ur die Suche nach Eigenvektoren bedeutet das:
\begin{enumerate}
\item Wenn wir einen Eigenwert \(\lambda\) kennen, k"onnen wir Eigenvektoren ganz einfach bestimmen, indem wir eine Basis von \(\mKer(\lambda\mId_V -f)\) finden (dazu muss man nur ein LGS l"osen).
  \item Um Eigenwerte zu finden, m"ussen wir die Gleichung \(\det(\lambda\mId_V - f) = 0\) nach \(\lambda\) l"osen.
  \end{enumerate}
\begin{definition}{Charakteristisches Polynom}{}
  Sei \(f\in\mEnd_K(V)\). Das Polynom \[\chi_f(\lambda) := \det(\lambda\mId_V-f)\] hei"st charakteristisches Polynom von \(f\). Die Gleichung \[\chi_f(\lambda) = 0\] hei"st charakteristische Gleichung von \(f\).
\end{definition}
Aus obigen "Uberlegungen folgt: Die Nullstellen von \(\chi_f(\lambda)\) sind genau die Eigenwerte von \(f\). Wenn wir \(f\) also diagonalisieren wollen, m"ussen wir folgendes tun:
\begin{enumerate}
\item charakteristische Gleichung l"osen, um die Eigenwerte zu bestimmen
\item Basen von \(\mKer(\lambda\mId_V-f)\) zu jedem Eigenwert \(\lambda\) finden
\item hoffen, dass Vereinigung von diesen Basen eine Basis in \(V\) ergibt
\end{enumerate}
\begin{definition}{Eigenraum}{}
  \(\mKer(\lambda\mId_V-f)\) hei"st Eigenraum zum Eigenwert \(\lambda\).
\end{definition}
\begin{lemma}
  Die Eigenvektoren zu verschiedenen Eigenwerten sind linear unabh"angig, genauer: \[\lambda_1\neq\lambda_2\in K\implies\mKer(\lambda_1\mId_V-f)\cap\mKer(\lambda_2\mId_V-f)=\{0\}\]
  Insbesondere ist die Summe verschiedener Eigenr"aume immer direkt.
\end{lemma}
\begin{proof}
  Sei \(\lambda_1\neq\lambda_2\), sei weiterhin \(v\in\mKer(\lambda_1\mId_V-f)\cap\mKer(\lambda_2\mId_V-f)\). Dann gilt \(f(v)=\lambda_1v=\lambda_2v\) \(\implies(\lambda_1-\lambda_2)v = 0\implies v = 0\).
(Nachtrag). Ausf"uhrlicherer Beweis dieser Aussage: Seien \(v_1,\dots,v_n\) Eigenvektoren zu Eigenwerten \(\lambda_1,\dots,\lambda_n\) (diese sollen paarweise verschieden sein). Sei \(\alpha_1 v_1+\dots + \alpha_nv_n = 0\). Wir definieren \(g_i\):
  \begin{align*}
    g_i := (f-\lambda_1\mId_V)\dots(f-\lambda_{i-1}\mId_V)(f-\lambda_{i+1}\mId_V)\dots(f-\lambda_n\mId_V)
  \end{align*}
  (Beachte hierbei: Ob dieser Ausdruck ein Produkt oder eine Funktionsverkettung ist, ist egal.)
  \begin{align*}
    g_i(v_k) &= (\lambda_k-\lambda_1)(\lambda_k-\lambda_2)\dots(\lambda_k-\lambda_{i-1})(\lambda_k-\lambda_{i+1})\dots(\lambda_k-\lambda_n)v_k =
               \begin{cases}
                 0, & k\neq i \\
                 \beta_iv_i, & k = i
               \end{cases} \quad ,\beta_i\neq 0
  \end{align*}
  \todo{Beweis vervollst"andigen, ich verstehe den letzten Schritt nicht :(}
\end{proof}
Insbesondere folgt f"ur (3), dass die Vereinigung von Basen in Eigenr"aumen immer eine linear unabh"angige Menge ist, aber manchmal gibt es einfach \gq{zu wenige} Eigenvektoren, damit sie zu einer Basis werden.
\begin{exa}[Eigenwertberechnung]
  \[M^B_B(f) =
    \begin{pmatrix}
      9 & -6 \\
      2 & 2
    \end{pmatrix}\]
  Nun suchen wir eine Eigenbasis von \(f\).
  \begin{enumerate}
  \item Charakteristische Gleichung: \(\det(\lambda\mId_V-f)=0\)
    \[\chi_f(\lambda) = \det\begin{pmatrix}9-\lambda & -6 \\ 2 & 2-\lambda\end{pmatrix} = (\lambda - 9)(\lambda - 2) + 12 = \lambda^2 - 11\lambda + 30 = (\lambda-5)(\lambda-6)\]
  \item Eigenwerte: \(\lambda_1 = 5, \lambda_2 = 6\)
  \item Eigenvektoren:
    \begin{enumerate}
    \item zu \(\lambda_1 = 5\): Finde Basis zu \(\mKer(f - \lambda\mId_V)\) \(\implies\) homogenes LGS mit Matrix \(
        \begin{pmatrix}
          4 & -6 \\
          2 & -3
        \end{pmatrix}
      \) l"osen, Basen finden \(\implies\begin{pmatrix}3\\2\end{pmatrix}\) ist L"osung
    \item f"ur \(\lambda_2 = 6\): L"ose homogenes LGS mit Matrix \(
        \begin{pmatrix}
          3 & -6 \\
          2 & -4
        \end{pmatrix}
\) \(\implies\begin{pmatrix}2\\1\end{pmatrix}\) ist L"osung
\end{enumerate}
\item Eigenbasis: \(B = (v_1, v_2)\), setze nun \(S = M^{B'}_B=\begin{pmatrix}3 & 2\\ 2 & 1\end{pmatrix}\)
  \[\implies M^{B'}_{B'}=S^{-1}\begin{pmatrix}9 & -6 \\ 2 & 2 \end{pmatrix}S = \begin{pmatrix}5 & 0 \\ 0 & 6 \end{pmatrix}\]
  \end{enumerate}
\end{exa}
Was sind m"ogliche Hindernisse zu Diagonalisierbarkeit?
Es kann passieren, dass es \gq{zu wenig} Eigenwerte gibt.
\begin{exa}[Rotation]
  Rotation in \(\mathbb{R}^2\) um \(\frac{\pi}{2}\): \(R_{\frac{\pi}{2}}\) hat die Matrix \(
  \begin{pmatrix}
    0 & -1 \\
    1 & 0
  \end{pmatrix}
  \).
  \[\chi_{R_{\frac{\pi}{2}}}(\lambda) = \det
    \begin{pmatrix}
      \lambda & 1\\
      -1 & \lambda
    \end{pmatrix}
    = \lambda^2 + 1
\] \(\implies\) keine reellen Nullstellen \(\implies\) keine Eigenwerte, keine Eigenvektoren. Hier scheitern wir daran, dass das Polynom \(\lambda^2+1\) nicht genug Nullstellen in \(\mathbb{R}\) hat.
\end{exa}

\subsection{Nullstellen von Polynomen}
\begin{definition}{Nullstelle}{}
  Sei \(p\in K[\lambda] = \left\{\sum_{i=0}^n\alpha_i\lambda^i|\alpha_i\in K, n\in\mathbb{N} \right\}\) ein Polynom. Eine Nullstelle \(\mu\) von \(p\) ist ein Element \(\mu\in K\) so dass \(p(\mu) = 0\).
\end{definition}
\begin{lemma}[Bezout]
  \(\mu\in K\) ist eine Nullstelle von \(p \iff p(\lambda) = (\lambda-\mu)\cdot p_1(\lambda)\), wobei \(p_1\) ein Polynom ist.
\end{lemma}
\begin{proof}[Bezout, Skizze]
  \((\impliedby)\): offensichtlich \\
  \((\implies)\): folgt durch Division mit Rest: Wenn man \(p\) durch \((\lambda-\mu)\) mit Rest dividiert, bekommt man \(p(\lambda) = (\lambda-\mu)p_1(\lambda) + r(\lambda)\). Es ist (siehe auch Analysis) \(\mDeg(r) < \mDeg(\lambda-\mu) = 1\) \(\implies r(\lambda)=r\) ist Konstante. \(0 = p(\mu) = 0 + r \implies r = 0\).
\end{proof}
\begin{beobachtung}
  Ein Polynom von Grad \(d\) hat h"ochstens \(d\) Nullstellen.
\end{beobachtung}
\begin{theo}{Fundamentalsatz der Algebra}
  Jedes Polynom \(p\in\mathbb{C}[\lambda]\) zerf"allt in Linearfaktoren:
  \[p(\lambda) = (\lambda - \lambda_1)(\lambda-\lambda_2)\dots(\lambda-\lambda_d) \qquad d = \mDeg p\]
  "Aquivalent: Ein Polynom \(p\in\mathbb{C}[\lambda]\) von Grad \(d\) hat genau \(d\) Nullstellen in \(\mathbb{C}\), wenn man sie mit Vielfachheiten z"ahlt (das hei"st z.B. \(p(\lambda) = (\lambda-1)^3(\lambda-4)^5\) hat eine dreifache Nullstelle 1, eine f"unffache Nullstelle 4 \(\implies\) insgesamt 8 Nullstellen)
\end{theo}

\begin{folgerung}
  Sei $V$ ein $\mathbf{C}$ Vektorraum, $f: V \mapsto V$ linear. Dann hat $f$
  einen Eigenvektor. \\
  Ideen:
  \begin{itemize}
  \item Wenn $V$ nun ein $\mathbb{R}$ Vektorraum ist, k"onnen wir es zu einem
    $\mathbb{C}$ Vektorraum aufr"usten.
  \end{itemize}
\end{folgerung}


\begin{definition}
  Sei V ein $\mathbb{R}$ Vektorraum. Die komplexifizierung $V_{\mathbb{C}}$ von
  $V$ ist der folgende $\mathbb{C}$ Vektorraum. \[V_{\mathbb{C}}=\{(v,w) |
    v,w\in V\}\] als Menge, Addition komplexweise, .... 
  Man denkt an $(v,w)$ als ''$v+i\cdot w$'' 
\end{definition}

\begin{lemma}
  Sei $f: V\mapsto V$ $\mathbb{R}$ -  Linear, dann ist .... eine
  komlexifizierung von $f$
\end{lemma}
\begin{proof}
  $f_{\mathbb{c}}$ ... 
\end{proof}

\begin{lemma}
  Wenn $B$ eine Basis in $V$ ist, so ist ...
\end{lemma}

\begin{proof}
  $(b_1,0),...,(b_n,0)$ Spannen $V_{\mathbb{C}}$ weil die reellend
  Linearkombinationen von ... den Raum ... aufspannenn, und ... den rest
  aufspannen. Sie sind auch linear unabh"angig "uber $\mathbb{C}$ denn, die
  Aufspannabbildung: ... ist die Komplexifizierung der reellen
  Aufspannabbildung. ... war injektiv $\implies$ ... inkektiv. 
\end{proof}

\begin{lemma}
  Wenn $f: V\mapsto V$ ... linear ist und ... die Komplexifizierung, $B$ Basis
  in $V$, dann gilt: ... (Wir nehmen reelle Matrizen als Complexe war.)
\end{lemma}

\begin{proof}
  Inspektion der Abbildungsmatrix. \[f_(\mathbf{C})(b_i,0) = (f(b_i), 0)\]
\end{proof}
\begin{exa}
  ...
\end{exa}

\begin{definition}
  Sei $f: V \mapsto V$ eine $K$ - Lineare Abbildung. Ein Untervektorraum
  $U\subseteq V$ heisst invariant, wenn $f(U) \subseteq U$.
\end{definition}

\begin{proposition}
  Sei $f: V\mapsto V$ eine ... Abbildung. Eim Vektor ... ist genau dann eine
  Eigenvektor von ... mit dem Eogenwert ..., wenn .... invariant unter $f$ ist
  mit ...
\end{proposition}

\begin{proof}
  ... Diese Matrix hat Eigenwerte $\alpha + i\cdot\beta$ und hat keine
  Eigenvektoren.
\end{proof}

Zusammen mit der Existenz von Eigenvektoren f"ur $f_{\mathbb{C}}$ haben wir:

\begin{proposition}
  ... ein linearer Operator auf einem $\mathbb{R}$ VR. Dann hat $f$ einen 1,
  oder 2 dimensionalen invarianten Unterraum.
\end{proposition}

Geometrisch:
\begin{relation}
  \begin{itemize}
  \item[1 - Dimensional] = Eigenraum
  \item[2 - Dimensional] = Drehstreckung
  \end{itemize}
\end{relation}

Diese Proposition beantwortet auf die bestm"ogliche Weise die Frage: ''Was
passiert, wenn ein reeller Operator Eigenwerte mit imagin"arem Teil hat''

\begin{exa}
  ...
  Es ist $f$ nicht diagonalisierbar.
\end{exa}

\begin{definition}
  Seien $p,q\in K[\lambda]$ zwei Polynome. $p$ teilt $q$ ($p|q$), wenn $\exists
  Es ist $f$ nicht diagonalisierbarp_1 \in K[t]$ s.d. $q= p\cdot p_1$. 
\end{definition}

\begin{proposition}
  Sei ... linearer Operator $U\subseteq V$ ein invarianter Untervektorraum. Dann
  Es ist $f$ nicht diagonalisierbargilt $\Xi_{f} | $... 
\end{proposition}
\begin{proof}
  Nehmen wir eine Basis $B_U$ in $U$ und erweitern sie zu einer Basis $B$ in $V$
\end{proof}

\begin{folgerung}
  Die Dimension des Eigenraumes $Ker...$ ist kleiner gleich der Vielfachheit von
  Es ist $f$ nicht diagonalisierbar $\lambda$ als Nullstelle $\Xi_f$.
\end{folgerung}

\begin{proof}
  .... ist invariant, 
\end{proof}

\begin{definition}
  Sei ... linear $\lambda$ ein Eigenwert von $f$. Die algebraische Vielfachheit
  Es ist $f$ nicht diagonalisierbar von $\lambda$, $\mu_{alg}(\lambda)$ ist die
  Es ist $f$ nicht diagonalisierbarVielfachheit von $\lambda$ als Nullstelle von $\Xi_f$.
\end{definition}

\begin{definition}
  Sei ... linear, $\lambda$ ein Eigenwert von $f$. Die geometrische Vielfachheit
  Es ist $f$ nicht diagonalisierbarvon $\lambda$ ist ...
  Die Behauptung oben ist ...
\end{definition}

\begin{theo}
  Sei ... linear. Dann ist $f$ diagonalisierbar genau dann, wenn:
  \begin{enumerate}
  \item alle Nullstellen von $\Xi_f \in K$ (Automatisch erf"ullt f"ur die
  Es ist $f$ nicht diagonalisierbarkomplexen Zahlen)
  \item f"ur jede Nullstelle $\lambda$ von $\Xi_f$ gilt: $\mu_{geo}(\lambda)=\mu_{alg}(\lambda)$
  \end{enumerate}
\end{theo}

\begin{proof}
  ... ist ein Polynom vom Grad $\dim V \rightarrow \Xi_f$ hat h"ochstens $n$
  Es ist $f$ nicht diagonalisierbarNullstellen in $K$, gez"ahlt mit
  Es ist $f$ nicht diagonalisierbarVielfachheiten. $\implies$ ....

  Andererseits ist $f$ genau dann diagonalisierbar, wenn es eine Basis aus
  Es ist $f$ nicht diagonalisierbarEigenvektoren gibt, also ..
\end{proof}

\begin{notte}
  \begin{proof}[Tips und Tricks]
    Der obige Satz ist nur interessant, wenn (1) erf"ullt ist und mehrfache
    Nullstellen existieren, weil f"ur jeden Eigenwert $\lambda$ von $f$ gilt ja
    nach Definition $\mu_{geo}(\lambda)\geq 1$. D.h., wenn alle Nullstellen in
    $K$ liegen und ..., f"ur alle Eigenwerte $\lambda$ dann gilt: ... f"ur alle
    Eigenwerte $\lambda$, und $f$ ist dann diagonalisierbar.

    $\implies$ Jede Komplexe Matrix ist bis auf eine beliebeig kleine
    Pertubation diagonalisierbar. In diesen Sinne sind ''die meissten'' Matrizen diagonalisierbar.
  \end{proof}
\end{notte}

Zum charakteristischen Polynom:
\begin{relation}
  \begin{enumerate}
  \item $\Xi_A(0) = \det (-A) = (-1)^n \det A$
  \item $\Xi_A (\lambda) = \lambda^n + ...$, weil der Einzige Summand in ...
  \item Der Koeffizient vor ... in ... $=\Tr (A)$
  \end{enumerate}
\end{relation}

Folglich kann man $\Xi_1$ f"ur $2\times 2$ Matrizen direkt ablesen.

\begin{definition}
  Sei V ein $K$ Vektorraum, ... Die Spalten von $f$ .... F"ur $\dim V < \infty$
  giltL ... 
\end{definition}
Dieses Mathematische Spektrum hat f"ur viele Physikalisch motivierte Operatoren
tats"achliche Bedeutung.

\begin{notte}
  Wir habengesehen, dass es nicht diagonalisierbare Matrizen gibt. Es gibt die
  nahelegende Fragem was ist f"ur solche allgemeinen Matrizen/Abbildungen die
  ''bestm"ogliche'' Form.
\end{notte}
\begin{relation}
  ....

  (Diagonalisierbarkeit entspricht der Bedingung, dass  alle Bl"ocke Gr"osse 1 haben.)
\end{relation}

\section{Bilineare und Quadratische Formen}
\label{sec:bili}
\textbf{Motivation}: Bislang haben wir Vektorr"aume ohne geometrische Strukturen
studiert; Speziell: wir konnten den Vektoren in Vektorra"umen keine
l"ange/Winkel zuordnen.

Eine Zusatzstruktur auf $V$, die das erm"oglicht, ist das Skalarprodukt. z.B.
im $\mathbb{R^n}$ gibt es das Standartskalarprodukt...) Dieses Skalarpordukt ist
linear in jeder Variable $\rightarrow$ ''Bilinearform''

In der Physik ist die folgene Bilinearform von Bedeutung: ...

\begin{definition}
  Sei $V$ ein $K$ - Vektorraum. Eine Bilinearform $b$ auf V ist eine Abbildung
  $b: V\times V \mapsto K$, die linear in jeder Variable ist.

  Die Zugeh"orige Quadratische Form: $q(v):=b(v,v)$
\end{definition}



\begin{definition}
  Sei $V$ ein $K$-Vektorraum, $B\subset V$ eine Basis, ... eine Biliniearform.
  Die MAtrix $M_B(b)$ der Bilinearform $b$ bzgl. der Baiss $B$ ist definiert
  durch die Eigenschaft $()M_B(b))_{ij}=b(b_i, b_j)$
\end{definition}
\begin{exa}
  ...
\end{exa}
Wenn $x,y$ die Koordinatenspalten von $v$ bzw. $w$ bzgl. $B$ sind, so haben wir
\[b(v,w)=b(...)=.... \implies b(v,w)\] ...

Wenn $B'$ eine andere in Basis $V$ ist und $x'$, $y'$ Koordinatenspalten von $v$
bzw. $w$ bzgl. $B'$, so haben wir: $x=M...$ ...  $\implies b(v,w)= ...$ Es
folgt, dass $M_{B'}= (M^{B'}_{B})^T\cdot M_B(b)\cdot (M^{B'}_{B}))$

Wie bei linearen Abbildungen stellt sich die Frage: ''Gibt es eine Basis
$M_B(b)$ besonders einfach ist?'' Diese Form ist f"ur unterschiedliche
Bilinearformen unterschiedlich. (Hier nur f"ur symetrische Formen.)

\begin{definition}
  Sei $U\subset V$ ein Untervektorraum $b$ eine Bilinearform aif $V$. Das
  orthogonale Komplement von $U$ bzgl. $b$ ist der Untervektorraum $U={v\in
    V|b(u,v)=0 \forall u\in U}$

  Der Kern/Annulator von $b$ ist der Untervektorraum $V....$ Die Bilinearform
  $b$ heisst nicht ausgeartet, wenn ...
\end{definition}

\begin{lemma}
  Sei $V$ ein $K$ Vektorraum, $B\subset V$ eine Basis, $\dim V < \infty$, $b$
  eine Bilinearform. Es gilt: $b$ nicht ausgeartet $\iff$ $M_B(b)$ nicht
  ausgeartet (Invertierbar)
\end{lemma}

\begin{proof}
  Wenn ..
  Die Bedingungen ... sind "Aquivalent zum LGS $M_B(x) \cdot x = 0 $  auf die
  Koordinatenspalte $x$ von $v$. Dieses LGS hat genau dann nur die Null"osung,
  wenn $M_B(b)$ nicht ausgeartet ist. Der Beweis Zeigt auf $\dim V... = \dim
  \{x \mid  M_B(b)= \dim V - \mRg M_B(b)\}$ insbesondere ist die Zahl $\mRg M_B(b)$
  unabh"angig von der Basis $B$
\end{proof}

\begin{definition}
  $\mRg (b):=\dim V - \dim V = \mRg M_B(b)$
\end{definition}

\begin{exa}
  Skalarprodukt ist nicht ausgeartet. Geometrisch wissen wir, wenn $U$ in
  $\mathbb{R}^3$ eine Gerade ist, dann ist, ... eine Ebene.
\end{exa}

\begin{proposition}
  Sei $V$ ein $K$ - Vektorraum, $b$ eine nicht ausgeartete Bilinearform auf V.
  Dann gilt f"ur jeden UVR ....
\end{proposition}

\begin{proof}
  ... 
\end{proof}

\begin{korollar}
  Wenn $b$ nicht ausgeartet ist, so gilt $(...)$ f"ur Jeden Untervektorraum
  $U\subset  V$
\end{korollar}
\begin{proof}
  .... nach dem Prinzip ''Paul, wie heisst du'' Ausserdem gilt ...
\end{proof}
  

\subsection{Schlagworte:}
\label{sec:orgcf8c685}
\begin{itemize}
\item \(A\cdot B\) Zeilen von \(A\) mal Spalten von \(B\)
\item LGS L"osungen als Vektor!
\begin{itemize}
\item Keine nicht offensichtlich Schritte ueberspringen!
\item Paramatervektor und sine Elemente genau definieren!
\end{itemize}
\item Transposition vertauscht faktoren.
\item k-te Spalte \((A)_k\)
\end{itemize}
\end{document}

%%% Local Variables:
%%% mode: latex
%%% TeX-master: t
%%% End:
